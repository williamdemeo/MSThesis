\documentclass{article}

\begin{document}

\begin{center} 
Lecture Notes on Monte Carlo Methods \\
Fall Semester, 2005 \\
Courant Institute of Mathematical Sciences, NYU \\
Jonathan Goodman, goodman@cims.nyu.edu \\ 
\vspace{.6cm}
\large 
Chapter 2:  Dynamic sampling and Markov chain Monte Carlo.
\normalsize
\end{center}

A variety of techniques collectively 
called\footnote{The practice predates the name by several decades.} 
{\em Markov chain Monte Carlo} (MCMC) or {\em dynamic sampling} allow sampling of 
comlex high dimensional distributions not accessablt by simple samplers.
With technical ideas to follow, the rough idea is that if $f$ is the invariant law 
of a nondegenerate Markov chain $X(t)$, then the law of $X(t)$ converges to $f$ as 
$t \to \infty$.
We do not have a direct simple way to sample $f$, so we choose a starting state 
$X(0)$ in an arbitrary way and {\em run} the chain.
Regardless of the distribution of $X(0)$, the law of $X(t)$ will converge to $f$
as $t\to\infty$.
Moreover, for large enough $s$, the samples $X(t)$ and $X(t+s)$ ``decorrelate''
(become independent, actually) so as to become effectively distinct samples of $f$.
Therefore we can estimate $A = E_f[V(X)]$ using
\begin{equation}
\widehat{A} = \frac{1}{n} \sum_{t=1}^n V(X(t)) \; .
\end{equation}
Regardless of the choice of $X(0)$, we have $\widehat{A} \to A$ as $n \to\infty$.
This is called the {\em ergodic theorem} for Markov chains, in anology to ergodic
theorems in statistical mechanics.
It is remarkable that there are many distributions that are impractical to sample
directly but easy (or at least possible) to sample with MCMC.

A rejection sampler is inefficient if its acceptence probability is low.
A dynamic sampler can be inefficient because of large long lasting correlations
between $X(t)$.
The {\em autocorrelation time} is the minimum $s$ before $X(t)$ and $X(t+s)$ 
are effectively independent samples from the point of view of error bars.
A {\em rapidly mixing} chain has small autocorrelation time.
There are tricks for inventing rapidly mixing chains in various situations
(e.g. Multigrid Monte Carlo, many more in Alan Sokal's notes or Jun Liu's book).
We will discuss error bars and autocorrelation time estimates for dynamic Monte Carlo.



\end{document}
