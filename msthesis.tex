%% NYU PhD thesis format. Created by Jos Koiller 2007--2008.

%% Use the first of the following lines during production to
%% easily spot "overfull boxes" in the output. Use the second
%% line for the final version.
%\documentclass[12pt,draft,letterpaper]{report}
\documentclass[12pt,letterpaper]{report}

%% Replace the title, name, advisor name, graduation date and dedication below with
%% your own. Graduation months must be January, May or September.
\newcommand{\thesistitle}{A Lanczos Procedure for Approximating Eigenvalues of Large Stochastic Matrices}
\newcommand{\thesisauthor}{William J. DeMeo}
\newcommand{\thesisadvisor}{Professor Jonathan Goodman}
\newcommand{\graddate}{January 1999}

%\urladdr{williamdemeo@gmail.com}
%% \urladdr{http://williamdemeo.org}
%% \address{Department of Mathematics\\
%% University of South Carolina\\Columbia 29208\\USA}
%% If you do not want a dedication, scroll down and comment out
%% the appropriate lines in this file.
\newcommand{\thesisdedication}{To my parents, with love and appreciation.}

%% The following makes chapters and sections, but not subsections,
%% appear in the TOC (table of contents). Increase to 2 or 3 to
%% make subsections or subsubsections appear, respectively. It seems
%% to be usual to use the "1" setting, however.
\setcounter{tocdepth}{1}

%% Sectional units up to subsubsections are numbered. To number
%% subsections, but not subsubsections, decrease this counter to 2.
\setcounter{secnumdepth}{3}

%% Page layout (customized to letter paper and NYU requirements):
\setlength{\oddsidemargin}{.6in}
\setlength{\textwidth}{5.8in}
\setlength{\topmargin}{.1in}
\setlength{\headheight}{0in}
\setlength{\headsep}{0in}
\setlength{\textheight}{8.3in}
\setlength{\footskip}{.5in}

%% Use the following commands, if desired, during production.
%% Comment them out for final version.
%\usepackage{layout} % defines the \layout command, see below
%\setlength{\hoffset}{-.75in} % creates a large right margin for notes and \showlabels

%% Controls spacing between lines (\doublespacing, \onehalfspacing, etc.):
\usepackage{setspace}

%% Use the line below for official NYU version, which requires
%% double line spacing. For all other uses, this is unnecessary,
%% so the line can be commented out.
\doublespacing % requires package setspace, invoked above

%% Each of the following lines defines the \com command, which produces
%% a comment (notes for yourself, for instance) in the output file.
%% Example:    \com{this will appear as a comment in the output}
%% Choose (uncomment) only one of the three forms:
%\newcommand{\com}[1]{[/// {#1} ///]}       % between [/// and ///].
\newcommand{\com}[1]{\marginpar{\tiny #1}} % as (tiny) margin notes
%\newcommand{\com}[1]{}                     % suppress all comments.

%% This inputs your auxiliary file with \usepackage's and \newcommand's:
%% It is assumed that that file is called "definitions.tex".
%%
%% \usepackage{enumerate,amsmath,amssymb,fancyhdr,mathrsfs,amsthm,url,stmaryrd}
\usepackage[colorlinks=true,urlcolor=black,linkcolor=black,citecolor=black]{hyperref}
%\usepackage{dcolumn}
\usepackage{relsize}
%\usepackage{booktabs}
%\usepackage{multicol}
\usepackage{marvosym}
%\usepackage{makeidx}
\usepackage{graphicx}
\usepackage{enumerate} 
\usepackage{tikz}
\usetikzlibrary{calc}
\usepackage{scalefnt}
\usepackage{verbatim}
\usepackage{color}
\usepackage{algorithm}
\usepackage{algorithmic}
\usepackage{ifthen}
\theoremstyle{plain}
\newtheorem{theorem}{Theorem}[chapter]
\newtheorem{corollary}[theorem]{Corollary}
\newtheorem{lemma}[theorem]{Lemma}
\newtheorem{prop}[theorem]{Proposition}
\newtheorem{assumption}[theorem]{Assumption}
\theoremstyle{definition}
\newtheorem{definition}[theorem]{Definition}
\newtheorem{example}[theorem]{Example}
\newtheorem{question}[theorem]{Question}
\newcounter{claim}
\newtheorem{claim}[claim]{Claim}
\newcounter{conjecture}
\newtheorem{conjecture}[conjecture]{Conjecture}
\newtheorem*{problem}{Problem}
\newtheorem*{fact}{Fact}
\newtheorem{case}{Case}
\theoremstyle{remark}
\newtheorem*{remark}{Remark}
\newtheorem*{remarks}{Remarks}
\newtheorem*{notation}{Notation}
\numberwithin{theorem}{chapter}
\numberwithin{claim}{chapter}
\numberwithin{equation}{chapter}
\numberwithin{conjecture}{chapter}
\renewcommand\S{\ensuremath{\mathcal{S}}}
\newcommand\sK{\ensuremath{\mathcal{K}}}
\newcommand\F{\ensuremath{\mathcal{F}}}
\newcommand\R{\ensuremath{\mathbb{R}}}
\newcommand\bC{\ensuremath{\mathbf{C}}}


%% \newcommand\C{\ensuremath{\operatorname{C}}}
%% \renewcommand\P{\ensuremath{\operatorname{P}}}
%% \newcommand\M{\ensuremath{\operatorname{M}}}
%% \renewcommand\H{\ensuremath{\operatorname{H}}}
%% \newcommand\T{\ensuremath{\operatorname{T}}}
%% \newcommand\I{\ensuremath{\operatorname{I}}}
%% \newcommand\Q{\ensuremath{\operatorname{Q}}}
%% \newcommand\mR{\ensuremath{\operatorname{R}}}
%% \newcommand\Z{\ensuremath{\operatorname{Z}}}
%% \newcommand\E{\ensuremath{\operatorname{E}}}
%% \newcommand\U{\ensuremath{\operatorname{U}}}
%% \newcommand\V{\ensuremath{\operatorname{V}}}

% Changing to nicer matrix fonts
\newcommand\C{\ensuremath{\mathrm{C}}}
\renewcommand\P{\ensuremath{\mathrm{P}}}
\newcommand\bP{\ensuremath{\mathbf{P}}}
\newcommand\M{\ensuremath{\mathrm{M}}}
\renewcommand\H{\ensuremath{\mathrm{H}}}
\newcommand\T{\ensuremath{\mathrm{T}}}
\newcommand\I{\ensuremath{\mathrm{I}}}
\newcommand\K{\ensuremath{\mathrm{K}}}
\newcommand\Q{\ensuremath{\mathrm{Q}}}
\newcommand\mR{\ensuremath{\mathrm{R}}}
\newcommand\Z{\ensuremath{\mathrm{Z}}}
\newcommand\E{\ensuremath{\mathrm{E}}}
\newcommand\U{\ensuremath{\mathrm{U}}}
\newcommand\V{\ensuremath{\mathrm{V}}}

\newcommand\bE{\ensuremath{\mathbf{E}}}
\newcommand\Var{\ensuremath{\mathbf{Var}}}
\newcommand\ran{\ensuremath{\operatorname{ran}}}
\newcommand\psip[1]{\ensuremath{\psi^{(#1)}}}
\newcommand\pij{\ensuremath{p_{ij}}}
\newcommand\pji{\ensuremath{p_{ji}}}
%\renewcommand\phi{\ensuremath{\varphi}}
\newcommand\<{\ensuremath{\langle}}
\renewcommand\>{\ensuremath{\rangle}}
\newcommand\diag{\ensuremath{\operatorname{diag}}}
\newcommand\one{\ensuremath{\mathbf{1}}}
\newcommand\lambdamax{\ensuremath{\lambda_{\max}(\M)}}

\usepackage{enumerate,amsmath,amssymb,fancyhdr,mathrsfs,amsthm,url,stmaryrd}
\usepackage[colorlinks=true,urlcolor=black,linkcolor=black,citecolor=black]{hyperref}
\usepackage{relsize}
\usepackage{marvosym}
\usepackage{graphicx}
\usepackage{enumerate} 
\usepackage{tikz}
\usetikzlibrary{calc}
\usepackage{scalefnt}
\usepackage{verbatim}
\usepackage{color}
\usepackage{algorithm}
\usepackage{algorithmic}
\usepackage{ifthen}
\theoremstyle{plain}
\newtheorem{theorem}{Theorem}[chapter]
\newtheorem{corollary}[theorem]{Corollary}
\newtheorem{lemma}[theorem]{Lemma}
\newtheorem{prop}[theorem]{Proposition}
\newtheorem{assumption}[theorem]{Assumption}
\theoremstyle{definition}
\newtheorem{definition}[theorem]{Definition}
\newtheorem{example}[theorem]{Example}
\newtheorem{question}[theorem]{Question}
\newcounter{claim}
\newtheorem{claim}[claim]{Claim}
\newcounter{conjecture}
\newtheorem{conjecture}[conjecture]{Conjecture}
\newtheorem*{problem}{Problem}
\newtheorem{fact}{Fact}[chapter]
\newtheorem{case}{Case}
\theoremstyle{remark}
\newtheorem*{remark}{Remark}
\newtheorem*{remarks}{Remarks}
\newtheorem*{notation}{Notation}
\numberwithin{theorem}{chapter}
\numberwithin{claim}{chapter}
\numberwithin{equation}{chapter}
\numberwithin{conjecture}{chapter}
\renewcommand\S{\ensuremath{\mathcal{S}}}
\newcommand\sK{\ensuremath{\mathcal{K}}}
\newcommand\F{\ensuremath{\mathcal{F}}}
\newcommand\R{\ensuremath{\mathbb{R}}}
\newcommand\bC{\ensuremath{\mathbf{C}}}

\newcommand\C{\ensuremath{\mathrm{C}}}
\renewcommand\P{\ensuremath{\mathrm{P}}}
\newcommand\bP{\ensuremath{\mathbf{P}}}
\newcommand\M{\ensuremath{\mathrm{M}}}
\renewcommand\H{\ensuremath{\mathrm{H}}}
\newcommand\T{\ensuremath{\mathrm{T}}}
\newcommand\I{\ensuremath{\mathrm{I}}}
\newcommand\K{\ensuremath{\mathrm{K}}}
\newcommand\Q{\ensuremath{\mathrm{Q}}}
\newcommand\mR{\ensuremath{\mathrm{R}}}
\newcommand\Z{\ensuremath{\mathrm{Z}}}
\newcommand\E{\ensuremath{\mathrm{E}}}
\newcommand\U{\ensuremath{\mathrm{U}}}
\newcommand\V{\ensuremath{\mathrm{V}}}

\newcommand\bE{\ensuremath{\mathbf{E}}}
\newcommand\Var{\ensuremath{\mathbf{Var}}}
\newcommand\ran{\ensuremath{\operatorname{ran}}}
\newcommand\psip[1]{\ensuremath{\psi^{(#1)}}}
\newcommand\pij{\ensuremath{p_{ij}}}
\newcommand\pji{\ensuremath{p_{ji}}}
\newcommand\<{\ensuremath{\langle}}
\renewcommand\>{\ensuremath{\rangle}}
\newcommand\diag{\ensuremath{\operatorname{diag}}}
\newcommand\one{\ensuremath{\mathbf{1}}}
\newcommand\lambdamax{\ensuremath{\lambda_{\max}(\M)}}


%% Cross-referencing utilities. Use one or the other--whichever you prefer--
%% but comment out both lines for final version.
%\usepackage{showlabels}
%\usepackage{showkeys}


\begin{document}
%% Produces a test "layout" page, for "debugging" purposes only.
%% Comment out for final version.
%\layout % requires package layout (see above, on this same file)

%%%%%% Title page %%%%%%%%%%%
%% Sets page numbering to "roman style" i, ii, iii, iv, etc:
\pagenumbering{roman}
%
%% No numbering in the title page:
\thispagestyle{empty}
%
\begin{center}
  {\large\textbf{\thesistitle}}

  \vspace{5mm}

  by

  \vspace{5mm}

  \thesisauthor

  \vfill

%% \begin{doublespace}
  A dissertation submitted in partial satisfaction of the\\[-4pt]
  requirements for the degree of\\[-4pt]
  Master of Science\\[6pt]
  in\\[6pt]
  Mathematics\\[6pt]
  in the\\[6pt]
  GRADUATE DIVISION\\[-4pt]
  of the \\[-4pt]
  NEW YORK UNIVERSITY\\
%% \end{doublespace}
\end{center}
Committee in charge:\\
\phantom{XXX} Professor Jonathan Goodman, Chair\\[-4pt]
\phantom{XXX} Professor Leslie Greengard
\begin{center}
  \graddate
\end{center}
\vfill

\noindent\makebox[\textwidth]{\hfill\makebox[2.5in]{\hrulefill}}\\
\makebox[\textwidth]{\hfill\makebox[2.5in]{\hfill\thesisadvisor\hfill}}
\newpage
%%%%%%%%%%%%% Blank page %%%%%%%%%%%%%%%%%%
\thispagestyle{empty}
\vspace*{0in}
\newpage

%%%%%%%%%%%%%% Dedication %%%%%%%%%%%%%%%%%
%% Comment out the following lines if you do not want to dedicate
%% this to anyone...
\vspace*{\fill}
\begin{center}
  \thesisdedication\addcontentsline{toc}{section}{Dedication}
\end{center}
\vfill
\newpage
%%%%%%%%%%%%%% Acknowledgements %%%%%%%%%%%%
%% Comment out the following lines if you do not want to acknowledge
%% anyone's help...
\section*{Acknowledgments}\addcontentsline{toc}{section}{Acknowledgments}
%%% 
%%% First, I thank my advisor, Professor Jonathan Goodman, for giving me the opportunity to work on
this problem, and helping me arrive at the following exposition. Next, I wish to thank Professor
Helena Frydman for first sparking my interest in Markov chains by giving lucid descriptions of their
many interesting properties and applications. I thank Professors James Demmel and Beresford
Parlett, for answering many questions pertaining to this problem, and Professor Leslie Greengard
for agreeing to review this paper.

Finally, I would like to thank my family, for their support of my interests in research (and
everything else!), and especially my parents, Barbara and Ted Terry, and Benita and Bill De Meo.

Without them, this paper would not have been written.


%%%
First, I thank my advisor, Professor Jonathan Goodman, for giving me the opportunity to work on
this problem, and helping me arrive at the following exposition. Next, I wish to thank Professor
Helena Frydman for first sparking my interest in Markov chains by giving lucid descriptions of their
many interesting properties and applications. I thank Professors James Demmel and Beresford
Parlett, for answering many questions pertaining to this problem, and Professor Leslie Greengard
for agreeing to review this paper.

Finally, I would like to thank my family, for their support of my interests in research (and
everything else!), and especially my parents, Barbara and Ted Terry, and Benita and Bill DeMeo.
Needless to say, without them this paper would not have been written.


\newpage
%%%% Abstract %%%%%%%%%%%%%%%%%%
\section*{Abstract}\addcontentsline{toc}{section}{Abstract}
%%%----------------------------------------------------------------
%%%
%%% \begin{center}
\newcommand\skipsize{6pt}
A Lanczos Procedure for Approximating Eigenvalues of Large Stochastic Matrices\\[\skipsize]
by\\[\skipsize]
William J. DeMeo\\[\skipsize]
Master of Science in Mathematics\\[\skipsize]
New York University\\[\skipsize]
Professor Jonathan Goodman, Chair
\end{center}

The rate at which a Markov chain converges to a given probability distribution
has long been an active area of research. Well known bounds on this rate of
convergence involve the subdominant eigenvalue of the chain's underlying
transition probability matrix. However, many transition probability matrices are
so large that we are unable to store even a vector of the matrix in fast
computer memory. Thus, traditional methods for approximating eigenvalues are
rendered useless. 

In this paper we demonstrate that, if the Markov chain is reversible, and we
understand the structure of the chain, we can derive the coefficients of the
traditional Lanczos algorithm without storing a single vector. We do this by
considering the variational properties of observables on the chain's state
space. In the process we present the classical theory which relates the
information contained in the Lanczos coefficients to the eigenvalues of the
Markov chain. 

%%%
\begin{center}
\newcommand\skipsize{6pt}
A Lanczos Procedure for Approximating Eigenvalues of Large Stochastic Matrices\\[\skipsize]
by\\[\skipsize]
William J. DeMeo\\[\skipsize]
Master of Science in Mathematics\\[\skipsize]
New York University\\[\skipsize]
Professor Jonathan Goodman, Chair
\end{center}

The rate at which a Markov chain converges to a given probability distribution
has long been an active area of research. Well known bounds on this rate of
convergence involve the subdominant eigenvalue of the chain's underlying
transition probability matrix. However, many transition probability matrices are
so large that we are unable to store even a vector of the matrix in fast
computer memory. Thus, traditional methods for approximating eigenvalues are
rendered useless. 

In this paper we demonstrate that, if the Markov chain is reversible, and we
understand the structure of the chain, we can derive the coefficients of the
traditional Lanczos algorithm without storing a single vector. We do this by
considering the variational properties of observables on the chain's state
space. In the process we present the classical theory which relates the
information contained in the Lanczos coefficients to the eigenvalues of the
Markov chain. 
%%%----------------------------------------------------------------

\newpage
%%%% Table of Contents %%%%%%%%%%%%
\tableofcontents

\newpage
%%%%% List of Figures %%%%%%%%%%%%%
%% Comment out the following two lines if your thesis does not
%% contain any figures. The list of figures contains only
%% those figures included withing the "figure" environment.
%% \listoffigures\addcontentsline{toc}{section}{List of Figures}
%% \newpage

%%%%% List of Tables %%%%%%%%%%%%%
%% Comment out the following two lines if your thesis does not
%% contain any tables. The list of tables contains only
%% those tables included withing the "table" environment.
%% \listoftables\addcontentsline{toc}{section}{List of Tables}
%% \newpage

%%%%% List of Algorithms %%%%%%%%%%%%%
%% Comment out the following two lines if your thesis does not
%% contain any algorithms. The list of tables contains only
%% those tables included withing the "table" environment.
%% \listofalgorithms\addcontentsline{toc}{section}{List of Algorithms}
%% \newpage

%%%%% Body of thesis starts %%%%%%%%%%%%
\pagenumbering{arabic} % switches page numbering to arabic: 1, 2, 3, etc.
%% Introduction. If your thesis has no introduction, or chapter 1 is
%% meant to be the introduction, then comment out the lines below.
\section*{Introduction}\addcontentsline{toc}{section}{Introduction}
%%%----------------------------------------------------------------
%%%
%%% The rate at which a Markov chain converges to a given probability distribution has long
been an active area of research. This is not surprising considering this problem’s relevance to the ar-
eas of statistics, statistical mechanics, and computer science. Markov Chain Monte Carlo (MCMC)
algorithms provide important examples. These algorithms come in handy when we encounter a
complicated probability distribution from which we want to draw random samples. In statistical
mechanics, we might wish to estimate the phase average of a function on the state space. Goodman
and Sokal [6] examine Monte Carlo methods in this context. Examples from statistics occur in the
Bayesian paradigm when we are forced to simulate an unwieldy posterior distribution (see, e.g.,
Geman and Geman 

To implement the MCMC algorithm, we invent a Markov chain that converges to the
desired distribution (this is often accomplished using the Metropolis algorithm
described in Chapter 5). Realizations of the chain will eventually represent
samples from this distribution. Sometimes ``eventually'' -- meaning all but
finitely many terms of the chain -- is just not enough. We need more practical
results. In particular, we want to know how many terms of the chain should be 
discarded before we are sampling from a distribution that is close (in total variation distance) to
% 03.txt
the distribution of interest. This is the purpose of bounding convergence rates for Markov chains.

Often the Markov chains encountered in this context satisfy a condition known in the
physics literature as detailed balance. Probabilists call chains with this property reversible. This
simply means that the chain has the same probability law whether time moves
forward or 
backward.\footnote{This is not a precise definition. In particular the chain must
  have started from its stationary distribution. Full rigor is postponed until Section 2.1.}
In this paper, we consider the rate at which such chains converge to a \emph{stationary distribution}.\footnote{This and other italicized terms are defined in Section 2.1.}

There are a number of different methods in common use for bounding convergence rates of
Markov chains, and a good review of these methods with many references can be found in  More
recently developed methods, employing logarithmic Sobolev inequalities, are reviewed in  Most
of the bounds in common use involve the sub—dominant eigenvalue of the Markov chain's transition
probability matrix, and thus require good approximations to such eigenvalues. In many applications,
however, the transition probability matrix is so large that it becomes impossible to store even a
single vector of the matrix in conventional computer memory. These so called out-of-core problems
are not amenable to traditional eigenvalue algorithms\footnote{By ``traditional
  eigenvalue algorithms'' we refer to those found, for example, in Golub and Van
  Loan[5]. See also the book by Demmel [1] for a more recent discussion.}
without modification. This paper develops 
such a modification for the Markov chain eigenvalue problem. In particular it develops a method
for approximating the first few eigenvalues of a transition probability matrix when we know the
general structure of the underlying Markov chain. The method does not require storage of large
matrices or vectors. Instead we need only simulate the Markov chain, and conduct a statistical
analysis of the simulation.

Here is a look at what follows. Section 2.1 contains a review of the relevant Markov chain
theory. Readers conversant in the asymptotic theory of Markov chains might wish to at least skim
Section 2.1, if only to become familiar with our notation. Section 2.2 describes functions on the state
% 04.txt
space of the Markov process. This section and Chapter 3 develop the context in which we formulate
the new ideas of the paper. In the last section of Chapter 3, Section 3.3, we present the familiar
Krylov subspace and explain why this represents our best approximation to a subspace containing
extremal eigenvectors of the transition probability matrix.\footnote{or, more
  precisely, a similarity transformation of this matrix.} The first section of
Chapter 4 describes the \emph{Lanczos algorithm} for generating an orthonormal basis
for the Krylov subspace. As it stands, this algorithm is useless for an
out-of-core problem such as ours since, by definition of such problems, 
it requires too much data movement; all the computing time is spent swapping data between slow
and fast memory (e.g. between the hard disk and cache). Therefore, we discuss alternatives to
Lanczos and demonstrate that the \emph{Lanczos coeficients} are readily available through simulations of
the Markov chain, which fact allows us to avoid the standard algorithm altogether. Following this
is a chapter describing the Metropolis algorithm used to produce a reversible stochastic matrix.
It is here that we experiment with the procedure described in Section 4.2 and approximate the
extremal eigenvalues of the matrix, without storing any of its vectors. Finally, Chapter 6 concludes
the paper.
% 05.txt

%%%
The rate at which a Markov chain converges to a given probability distribution
has long been an active area of research. This is not surprising considering
this problem's relevance to the areas of statistics, statistical mechanics, and
computer science. Markov Chain Monte Carlo (MCMC) algorithms provide important
examples. These algorithms come in handy when we encounter a complicated
probability distribution from which we want to draw random samples. In
statistical mechanics, we might wish to estimate the phase average of a function
on the state space. Goodman and Sokal~\cite{GoodmanSokal:1989} examine Monte
Carlo methods in this context. Examples from statistics occur in the Bayesian
paradigm when we are forced to simulate an unwieldy posterior distribution (see,
e.g., Geman and Geman~\cite{Geman:1984}.  

To implement the MCMC algorithm, we invent a Markov chain that converges to the
desired distribution (this is often accomplished using the \emph{Metropolis algorithm}
described in Chapter~\ref{cha:an-appl-conv}). Realizations of the chain will
eventually represent samples from this distribution. Sometimes
``eventually''---meaning all but finitely many terms of the chain---is just not 
enough. We need more practical results. In particular, we want to know how many
terms of the chain should be discarded before we are sampling from a
distribution that is close (in total variation distance) to 
%
%
% ------------- 03.txt -----------------------------------------------------
%
%
the distribution of interest. This is the purpose of bounding convergence rates
for Markov chains. 

Often the Markov chains encountered in this context satisfy a condition known in
the physics literature as \emph{detailed balance}. Probabilists call chains with
this property \emph{reversible}. This simply means that the chain has the same
probability law whether time moves forward or backward.\footnote{This is not a
precise definition. In particular the chain must have started from its
stationary distribution. Full rigor is postponed until
Section~\ref{sec:general-theory}.} 
In this paper, we consider the rate at which such chains converge to
a \emph{stationary distribution}. This and other italicized terms are
defined in Section~\ref{sec:general-theory}.

There are a number of different methods in common use for bounding convergence
rates of Markov chains, and a good review of these methods with many references
can be found in~\cite{Rosenthal:1995}.  More recently developed methods
employing logarithmic Sobolev inequalities are reviewed in~\cite{Diaconis:1996}.  
Most of the bounds in common use involve the subdominant eigenvalue of the
Markov chain's transition probability matrix, and thus require good
approximations to such eigenvalues. In many applications, however, the
transition probability matrix is so large that it becomes impossible to store
even a single vector of the matrix in conventional computer memory. These so
called \emph{out-of-core} problems are not amenable to traditional eigenvalue
algorithms without modification.\footnote{By ``traditional 
  eigenvalue algorithms'' we mean those found, for example, in Golub and Van
  Loan~\cite{Golub:1996}. See also the book by Demmel\cite{Demmel:1997} for more 
  recent treatment.}
In this paper we develop such a modification for the Markov chain eigenvalue
problem. In particular we develop a method for approximating the first few
eigenvalues of a transition probability matrix when we know the general
structure of the underlying Markov chain. The method does not require storage
of large matrices or vectors. Instead we need only simulate the Markov chain,
and conduct a statistical analysis of the simulation.

Here is a brief summary of the paper. Section~\ref{sec:general-theory} contains
a review of the relevant Markov chain theory. Readers who already know the
basics of asymptotic theory of Markov chains might wish to skim 
Section\ref{sec:general-theory} if only to familiarize themselves with our
notation. Section~\ref{sec:funct-state-space} describes functions on the state 
% 04.txt
space of the Markov process. This section and Chapter~\ref{cha:invar-appr-invar}
develop the context in which we formulate the new ideas of the paper. In the
last section of Chapter~\ref{cha:invar-appr-invar},
Section~\ref{sec:krylov-subspace}, we describe the \emph{Krylov subspace} and
explain why this represents our best approximation to a subspace containing 
extremal eigenvectors of the transition probability matrix (more
  precisely, a similarity transformation of this matrix). The first section of
Chapter~\ref{cha:lanczos-procedures} describes the \emph{Lanczos algorithm} for
generating an orthonormal basis for the Krylov subspace. As it stands, this
algorithm is useless for an out-of-core problem since, by
definition of such problems, it requires too much data movement; all the
computing time is spent swapping data between slow and fast memory (disk to ram
to cache and back). We develop alternatives to the Lanczos algorithm and
demonstrate that the \emph{Lanczos coefficients} of the algorithm can be
obtained by simulations of the Markov chain, and this allows us to avoid
the standard algorithm altogether. Following this is a chapter describing the
Metropolis algorithm used to produce a reversible stochastic matrix. It is here
that we experiment with the procedure described in
Section~\ref{sec:lancz-proc-mark} and approximate the extremal eigenvalues of
the matrix, without storing any of its vectors. Finally,
Chapter~\ref{cha:conclusion} concludes the paper. 
% 05.txt

%%%===================================================================
%%%-------------------------------------------------------------------

%% If your thesis has different "Parts", use commands such as the following:
\part{Theory\label{part:one}}%

%%%-------------------------------------------------------------------
%%%
%%% \chapter{Markov Chains}

\section{General Theory}

This review of Markov chain theory can be found in any good probability text. The present
discussion is most similar to that of Durrett~\cite{Durret:1996}, to which we
refer the reader desiring greater detail. 

\subsection{The Basic Setup}

Heuristically, a Markov chain is a stochastic process with a lack of memory property. Here
this means that the future of the process, given its past behavior and its present state, depends only
on its present state. This is the probabilistic analogue of a familiar property of classical particle
systems. Given the position and velocities of all particles at time t, the equations of motion can be
completely solved for the future evolution of the system. Thus, information describing the behavior
of the process prior to time t is superfluous. To be a bit more precise, if technical, we need the
following definitions.

\begin{definition}
%Definition 2.1.1 
Let $(\S, \S)$ be a measurable space. A sequence $X_n$, $n\geq 0$, of random variables
taking values in \S\ is said to be a Markov chain with respect to the filtration 
$\sigma(X_0, \dots, X_{n})$ if for all $B \in \S$,
\begin{equation}
% Eqn 2.1
\label{eq:1}
\bP(X_{n+1} \in B \mid \sigma(X_0, \dots, X_{n}))
=\bP(X_{n+1} \in B \mid \sigma(X_{n})).
\end{equation}
\end{definition}
Equation~(\ref{eq:1}) merely states that if we know the present location or state of $X_{n}$,
then information about earlier locations or states is irrelevant for predicting $X_{n+1}$.

\begin{definition}
%% Definition 2.1.2 
A function $p : S \times \S \rightarrow \R$ is said to be a \emph{transition probability} if:
\begin{enumerate}
\item for each $x \in S$, $A \mapsto p(x, A)$ is a probability measure on $(S, \S)$.
\item for each $A \in \S$, $x \mapsto p(x, A)$ is a measurable function.
\end{enumerate}
\end{definition}
We call $X_n$ a Markov chain with transition probabilities $p_n$ if
\begin{equation}
% Eqn 2.2
\label{eq:2.2}
\bP(X_{n+1} \in B \mid  \sigma(X_n)) = p_n(X_n, B)
\end{equation}
The spaces $(S, \S)$ that we encounter below are standard Borel spaces, so the existence of the
transition probabilities follows from the existence of regular conditional
probabilities on Borel spaces---a standard measure theory result 
(see e.g.\cite{Durret:1996}). %[3] page 230

Suppose we are given an initial probability distribution $\mu$ on $(S, \S)$ and a sequence $p_n$ of
transition probabilities. We can define a consistent set of finite dimensional distributions by
\begin{equation}
% Eqn 2.3
\label{eq:2.3}
\bP(X_j\in B_j ,0 \leq j \leq n) = \int_{B_0} \mu(dx_0) \int_{B_1} p_0(x_0, dx_1) 
\int_{B_2} p_1(x_1, dx_2) \cdots 
\int_{B_n} p_{n-1}(x_{n-1}, dx_n).
\end{equation}
Furthermore, denote our probability space by
\[
(\Omega, \F) = (S^\omega,\S^\omega), \quad \text{where} \omega = \{0,l,...\}.
\]
We call this \emph{sequence space} and it is defined more explicitly by
\[
S^\omega  = \{(\omega_0, \omega_1, \dots) : \omega_i in S\}\quad \text{ and }
\quad
\S^\omega  = \sigma(\omega : \omega_i \in A_i \in \S).
\]
% 07.txt
The Markov chain that we will study on this space is simply $X_n(\omega) = \omega$, the coordinate maps.
Then, by the Kolmogorov extension theorem, there exists a unique probability measure $\bP_\mu$ on
$(\Omega, \F)$ so that the $X_n(\omega)$ have finite dimensional distributions~(\ref{eq:2.3}).

If instead of $\mu$, we begin with the initial distribution $\delta_x$, i.e., point mass at $x$, then we
denote the probability measure by $\bP_x$. With such measures defined for each $x$, we can in turn
define distributions $\bP_\mu$, given any initial distribution $\mu$, by
\[
\bP_\mu(A) = \int \mu(dx)\bP_x(A).
\]

That the foregoing construction---which, recall, was derived merely from an
initial distribution $\mu$ and a sequence $p_n$ of transition
probabilities---satisfies Definition~(\ref{eq:2.2}) of a Markov chain is not
obvious, and a proof can be found in~\cite{Durret:1996}.

To state the converse of the foregoing, if $X_n$ is a Markov chain with
transition probabilities $p_n$ and initial distribution $\mu$, then its finite
dimensional distributions are given by~(\ref{eq:2.3}). Proof of this is also found
in~\cite{Durret:1996}. 

Now that we have put the theory on a firm, if abstract, foundation, we can bring
the discussion down to earth by making the forgoing a little more
concrete. First, we specialize our study of Markov chains by assuming that our
chain is \emph{temporally homogeneous}, which means that the transition
probabilities do not depend on time; i.e., 
$p_n(\omega_n, B) = p(\omega_n, B)$. (This is the stochastic analogue of a
conservative system.) 
Next we assume that our state space $S$ is finite, and suppose for all states
$i,j \in S$ that $p(i, j) \geq 0$, and $\sum_j p(i, j) = 1$ for all $i$. In
this case, equation~(\ref{eq:2.2}) takes a more intuitive form,
\[
\bP(X_{n+1}=j \mid X_n = i) =p(i,j),
\]
and our transition probabilities become
\[
p(i,A) = \sigma_{j\in A} p(i,j).
\]
% 08.txt
If $\P$ is a matrix whose $(i, j)$ element is the transition probability 
$p(i, j)$ then P is a \emph{stochastic matrix}; that is, a matrix with elements
$p_{ij}$ satisfying 
\[
\pij \geq 0, \quad \sum_j \pij =1, \quad (i, j =1, 2, \dots, d).
\]
We also refer to $\P$ as the transition probability matrix.

Without loss of generality, we can further suppose our Markov chain is \emph{irreducible}. This
means that, for any states $i, j$, starting in state $i$ the chain will make a
transition to state $j$ at some future time with positive probability. This
state of affairs is often described by saying that all states \emph{communicate}. We
lose no generality with this assumption because any \emph{reducible} Markov chain can
be factored into irreducible classes of states which can each be studied
separately. 

The final two conditions we place on the Markov chains considered below will cost us
some generality. Nonetheless, there remain many examples of chains meeting these conditions
and making the present study worthwhile. Furthermore, it may be the case that, with a little
more work, we will be able to drop these conditions in future studies. The first condition is that
the chain is \emph{aperiodic}. If we let $I_x = \{n \geq 1: p^n(x,x) > 0\}$, we
call a Markov chain \emph{aperiodic} if, for any state $x$, the greatest common
divisor of $I_x$ is 1. The second assumption is that our chain is 
\emph{reversible}. This characterization is understood in terms of the following definition.

\begin{definition}
%% Definition 2.1.3 
\label{def:2.1.3}
A measure $\mu$ is called \emph{reversible} if it satisfies
\[
\mu(x)p(x,y) = \mu(y)p(y,x), \quad  \text{ for all $x$ and $y$.}
\]
We call a Markov chain \emph{reversible} if its stationary distribution (defined
in Section~\ref{sec:convergence-theorem}) is reversible. 
\end{definition}



% ---------- 09.txt ------------------------------------------------------------

\subsection{A Convergence Theorem}
% Section 2.1.2
\label{sec:convergence-theorem}
In succeeding arguments, we use some results concerning the asymptotic behavior of
Markov chains. These results require a few more definitions.

\begin{definition}
%% Definition 2.1.4
A measure $\pi$ is said to be a \emph{stationary measure} if
\begin{equation}
% Eqn 2.4
  \label{eq:2.4}
\sum_x \pi(x) p(x,y) = \pi(x).
\end{equation}
\end{definition}

Equation~(\ref{eq:2.4}) says $\bP_\pi(X_1 = y) = \pi(y)$, 
and by induction that $\bP_\pi(X_n = y) = \pi(y)$ for all $n \geq 1$. 
If $\pi$ is a probability measure, then we call $\pi$ a stationary distribution. It
represents an equilibrium for the chain in the sense that, if $X_0$ has
distribution $\pi$, then so does $X_n$ for all $n$.

When the Markov chain is irreducible and aperiodic, the distribution of the state at time
$n$ converges pointwise to $\pi$ as $n \rightarrow \infty$, regardless of the
initial state. It is convenient to state this convergence result in terms of the
Markov chain's transition probability matrix $\P$. Before doing so, we note that
irreducibility of a Markov chain is equivalent to irreducibility (in the usual 
matrix theory sense) of its transition probability matrix. Furthermore, it turns
out that a transition probability matrix of an aperiodic Markov chain falls into
that class of matrices often called acyclic, but for simplicity we will call
such stochastic matrices aperiodic. With this terminology, we can state the
convergence theorem in terms of the transition probability matrix \P. 

% Theorem 2.1.1
\begin{theorem}
\label{thm-2.1.1}
Suppose $\P$ is irreducible, aperiodic, and has stationary distribution
$\pi$. Then as $n\rightarrow \infty$, $p^n(i,j) \rightarrow \pi(j)$.
\end{theorem}

The notation $p^n(i,j)$ means the $(i,j)$ element of the $n$th power of $\P$.

A Markov chain whose transition probability matrix satisfies the hypotheses of
Theorem~\ref{thm-2.1.1} is called \emph{ergodic}. If we simulate an ergodic chain for
sufficiently many steps, having 
%
%
%
% ---------------10.txt --------------------------------------------------
%
%
%
begun in any initial state, the final state is a sample point from a
distribution that is close to $\pi$. 

To make this statement more precise requires that we define ``close.''
\begin{definition}
%% Definition 2.1.5 
Let $\pi$ be a probability measure on $S$. Then the \emph{total variation distance} at time
$n$ with initial state $x$ is given by 
\[
\Delta_x(n) = \|\P^n(x, A) - \pi(A)\|_{TV} = \max_{A\in \S} | \P^n(x,A)-\pi(A)|.
\]
\end{definition}
In what follows, we will measure rate of convergence using the function
$\tau_x(\epsilon)$, defined as the first time after which the total variation
distance is always less than $\epsilon$. That is, 
\[
\tau_x(\epsilon) = \min\{m : \Delta_x(n) \leq \epsilon \text{ for all } n\geq
m\}.
\]

To begin our consideration of the connection between convergence rates of Markov
chains and eigenvalues, we first note that an aperiodic stochastic matrix $\P$ has real
eigenvalues 
$1 = \lambda_0 > \lambda_1 \geq \lambda_2 \geq \cdots \geq \lambda_{d-1} \geq -1$, 
where $d = |S|$ is the dimension of the state space. For an ergodic chain, the
rate of convergence to the stationary distribution $\pi$ is bounded by a
function of the \emph{subdominant} eigenvalue. By subdominant eigenvalue we mean that
eigenvalue which is second largest in absolute value, and we denote this
eigenvalue by $\lambda_{\max} = \max{\lambda_1, |\lambda_{d-1}|}$. 
The function bounding the rate of convergence of a Markov chain is given by the
following theorem ($\log$ denotes the natural logarithm):
\begin{theorem}
% Theorem 2.1.2
\label{thm-2.1.2}
The quantity $\tau_x(\epsilon)$ satisfies
\begin{enumerate}
\item 
$\tau_x(\epsilon) \leq (1-\lambda_{\max})^{-1}(\log \pi(x)^{-1} + \log \epsilon^{-1})$;
\item $\max_{x \in S} \tau_x(\epsilon) \geq \frac{1}{2} \lambda_{\max}(1-\lambda_{\max})^{-1} \log(2\epsilon)^{-1}$.
\end{enumerate}
\end{theorem}
As this theorem shows, if we have an upper bound on the subdominant eigenvalue, then we have
an upper bound on the function $\tau_x(\epsilon)$. In what follows, we will
derive an approximation to the 
% 11.txt
subdominant eigenvalue and supply error bounds. Together, an approximation and error bounds
for $\lambda_{\max}$  provide enough information to make Theorem~\ref{thm-2.1.2} useful.

\section{Functions on the State Space}
% Sec 2.2
\label{sec:funct-state-space}
Recall that $X_n(\omega) = \omega_n\in S$ denotes the state in which the Markov
chain exists at time $n$. 
Suppose that 
$\Phi = \{\phi_1, \dots, \phi_p\}$ 
is a collection of $p$ \emph{observables}, or functions defined on the state
space $S$. Furthermore, let these observables be real valued, 
$\phi_i : S \rightarrow \R$. 
It is often useful to assume that none of the observables is a constant function. Suppose now that
the state space $S$ is finite with $d$ possible states. Then, since an
observable is simply a map of the state space, we can think of each $\phi_i$ as a
vector of $d$ real numbers---the $d$ values that it takes on at the different states.

Now assume the Markov chain is irreducible, and let $\pi$ denote its stationary distribution.
If we start the chain from its stationary distribution---i.e., suppose $X_0$ has
distribution $\pi$---then $X_n$ is a stationary process. 
Furthermore, for each $i$, $\phi_i(X_n)$ is a stationary stochastic process with
\emph{mean}
\[
\bE_\pi\phi_i = \sum_{x\in S} \pi(x)\phi_i(x)
\]
and \emph{autocovariance function}
\begin{align}
% Eqn 2.5
\bC_\pi (\phi_i(X_n),  \phi_i (X_{n+s})) 
&= \bE_\pi[(\phi_i(X_n) - \bE_\pi\phi_i)(\phi_i(X_{n+s}) - \bE_\pi\phi_i)]\\
&= \sum_{x, y\in S} \bP_\pi(X_n = x, X_{n+s} = y) (\phi_i(x)- \bE_\pi\phi_i)(\phi_i(y) - \bE_\pi\phi_i).\nonumber
\end{align}
By the definition of conditional probability, we can write~(\ref{eq:2.4}) as follows:
\[
\sum_{x, y\in S} \bP_\pi(X_n = x) \bP_\pi(X_{n+s} = y\mid X_n = x) (\phi_i(x)-
\bE_\pi\phi_i)(\phi_i(y) - \bE_\pi\phi_i).
\]
Equivalently,
\[
\sum_{x, y\in S} \pi(x)p^s_{xy} (\phi_i(x)-\bE_\pi\phi_i)(\phi_i(y) - \bE_\pi\phi_i).
\]
Here $p^s_{xy}$ denotes the element in row $x$ and column $y$ of $\P^s$, the
$s$th power of the transition probability matrix. 
Similarly, we define the \emph{cross-covariance} between the function 
$\phi_i$ at time $n$ and $\phi_j$ at time $n+s$ as
\begin{align}
% Eqn 2.6
\bC_\pi (\phi_i(X_n),  \phi_j (X_{n+s})) 
&= \bE_\pi[(\phi_i(X_n) - \bE_\pi\phi_i)(\phi_j(X_{n+s}) - \bE_\pi\phi_j)]\nonumber\\
&= \sum_{x, y\in S} \pi(x)p^s_{xy} (\phi_i(x)-\bE_\pi\phi_i)(\phi_j(y) - \bE_\pi\phi_j).
\end{align}
Now let $\<\Phi\>$ denote the matrix of mean vectors whose $j$th column is 
$\bE_\pi\phi_j\one$, where $\one = (1,\dots, 1)^t$,
and let $\Pi = \diag(\pi(\omega_1),\dots, \pi(\omega_d))$ be the $d \times d$
diagonal matrix with stationary probabilities $\pi(\omega)$
on the main diagonal and zeros elsewhere. 
Finally, denoting by $\C(s)$ the $p \times p$ covariance matrix
whose $(i,j)$ element is $\bC_\pi(\phi_i(X_n), \phi_j(X_{n+s}))$, we have
\begin{align*}
\C(0) &= \E(\Phi(X_n) - \<\Phi\>)(\Phi(X_n) - \<\Phi\>)^t\\
&= (\Phi - \<\Phi\>)^t\Pi (\Phi-\<\Phi\>), \\
\C(s) &= \E(\Phi(X_n) - \<\Phi\>)(\Phi(X_{n+s}) - \<\Phi\>)^t\\
&= (\Phi - \<\Phi\>)^t \Pi \P^s (\Phi - \<\Phi\>).
\end{align*}

Below, we will also find it useful to have at our disposal a new matrix that is \emph{similar} to the
transition probability matrix. We have in mind the matrix 
$\M = \Pi^{1/2}\P\Pi^{-1/2}$. 
As is easily verified, this allows us to write the covariance matrix as
\begin{align}
\label{eq:2.7}
\C(s) &= 
(\Phi - \<\Phi\>)^t \Pi^{\frac{1}{2}t}\M^s\Pi^{\frac{1}{2}}(\Phi - \<\Phi\>)\nonumber\\
&=\Psi^t \M^s \Psi,  % Eqn 2.7
\end{align}
%% \\Phi\Phi‘M’\\Phi\Phi (2.7)
where we have defined $\Psi = \Pi^{\frac{1}{2}}(\Phi - \<\Phi\>)$. Recall that
our main motivation is that for out-of-core problems traditional eigenvalue
algorithms are inadequate. With this fact and the form~(\ref{eq:2.7}) % (2.7) 
in 
%
%
% ------------------ 13.txt ---------------------------------------------------
%
%
mind, we will consider using the covariance of observables on the state space to implement the
%% (otherwise useless) 
Rayleigh-Ritz procedure, which we describe below. This procedure requires
that $\M$ be symmetric. As the next fact demonstrates, 
%% this required symmetry is the reason for our
%% interest in the reversibility property of the Markov chain.
this need for symmetry is the reason we insist that the Markov chain be reversible.
\begin{fact}
%% Fact 2.2.1 
The matrix $\M$ is symmetric if and only if the Markov chain is reversible (i.e., iff the
process satisfies the \emph{detailed balance} condition).
\end{fact}
\begin{proof}
  \begin{align*}
  \M \text{ is symmetric } 
& \quad \Longleftrightarrow  \quad
 (\Pi^{1/2}\P\Pi^{-1/2})^t = \Pi^{1/2}\P\Pi^{-1/2}\\
& \quad \Longleftrightarrow  \quad
 \Pi^{-\frac{1}{2}t}\P^t\Pi^{\frac{1}{2}t} = \Pi^{1/2}\P\Pi^{-1/2}\\
& \quad \Longleftrightarrow  \quad
 \P^t\Pi^t = \Pi\P.
  \end{align*}
Elementwise, the final equality is $\pi_i\pij = \pi_j\pji$. 
According to Definition~\ref{def:2.1.3}, this states
that $p$ is a reversible measure.
\end{proof}

% 14.txt

 % Chapter 2
%%%
\chapter{Markov Chains}

\section{General Theory}
\label{sec:general-theory}
This review of Markov chain theory can be found in any good probability text. The present
discussion is most similar to that of Durrett~\cite{Durret:1996}, to which we
refer the reader desiring greater detail. 

\subsection{The Basic Setup}

Heuristically, a Markov chain is a stochastic process with a lack of memory property. Here
this means that the future of the process, given its past behavior and its present state, depends only
on its present state. This is the probabilistic analogue of a familiar property of classical particle
systems. Given the position and velocities of all particles at time t, the equations of motion can be
completely solved for the future evolution of the system. Thus, information describing the behavior
of the process prior to time t is superfluous. To be a bit more precise, if technical, we need the
following definitions.

\begin{definition}
%Definition 2.1.1 
Let $(S, \S)$ be a measurable space. A sequence $X_n$, $n\geq 0$, of random variables
taking values in $S$ is said to be a Markov chain with respect to the filtration 
$\sigma(X_0, \dots, X_{n})$ if for all $B \in \S$,
\begin{equation}
% Eqn 2.1
\label{eq:1}
P(X_{n+1} \in B \mid \sigma(X_0, \dots, X_{n}))
=P(X_{n+1} \in B \mid \sigma(X_{n})).
\end{equation}
\end{definition}
Equation~(\ref{eq:1}) merely states that if we know the present location or state of $X_{n}$,
then information about earlier locations or states is irrelevant for predicting $X_{n+1}$.

\begin{definition}
%% Definition 2.1.2 
A function $p : S \times \S \rightarrow \R$ is said to be a \emph{transition probability} if:
\begin{enumerate}
\item for each $x \in S$, $A \mapsto p(x, A)$ is a probability measure on $(S, \S)$.
\item for each $A \in \S$, $x \mapsto p(x, A)$ is a measurable function.
\end{enumerate}
\end{definition}
We call $X_n$ a Markov chain with transition probabilities $p_n$ if
\begin{equation}
% Eqn 2.2
\label{eq:2.2}
P(X_{n+1} \in B \mid  \sigma(X_n)) = p_n(X_n, B).
\end{equation}
The spaces $(S, \S)$ that we encounter below are standard Borel spaces, so the existence of the
transition probabilities follows from the existence of regular conditional
probabilities on Borel spaces---a standard measure theory result 
(see e.g.\cite{Durret:1996}). %[3] page 230

Suppose we are given an initial probability distribution $\mu$ on $(S, \S)$ and a sequence $p_n$ of
transition probabilities. We can define a consistent set of finite dimensional distributions by
\begin{equation}
% Eqn 2.3
\label{eq:2.3}
P(X_j\in B_j ,0 \leq j \leq n) = \int_{B_0} \mu(dx_0) \int_{B_1} p_0(x_0, dx_1) 
%\int_{B_2} p_1(x_1, dx_2) 
\cdots 
\int_{B_n} p_{n-1}(x_{n-1}, dx_n).
\end{equation}
Furthermore, denote our probability space by
\[
(\Omega, \F) = (S^\omega,\S^\omega), \quad \text{ where } \omega = \{0,1,\dots\}.
\]
We call this \emph{sequence space} and it is defined more explicitly by
\[
S^\omega  = \{(\omega_0, \omega_1, \dots) : \omega_i \in S\}\quad \text{ and }
\quad
\S^\omega  = \sigma(\omega : \omega_i \in A_i \in \S).
\]
% 07.txt
The Markov chain that we will study on this space is simply $X_n(\omega) = \omega$, the coordinate maps.
Then, by the Kolmogorov extension theorem, there exists a unique probability measure $P_\mu$ on
$(\Omega, \F)$ so that the $X_n(\omega)$ have finite dimensional distributions~(\ref{eq:2.3}).

If instead of $\mu$ we begin with the initial distribution $\delta_x$, i.e., point mass at $x$, then we
denote the probability measure by $P_x$. With such measures defined for each $x$, we can in turn
define distributions $P_\mu$, given any initial distribution $\mu$, by
\[
P_\mu(A) = \int \mu(dx)P_x(A).
\]

That the foregoing construction---which, recall, was derived merely from an
initial distribution $\mu$ and a sequence $p_n$ of transition
probabilities---satisfies Definition~(\ref{eq:2.2}) of a Markov chain is not
obvious, and a proof can be found in~\cite{Durret:1996}.

To state the converse of the foregoing, if $X_n$ is a Markov chain with
transition probabilities $p_n$ and initial distribution $\mu$, then its finite
dimensional distributions are given by~(\ref{eq:2.3}). Proof of this is also found
in~\cite{Durret:1996}. 

Now that we have put the theory on a firm, if abstract, foundation, we can bring
the discussion down to earth by making the forgoing a little more
concrete. First, we specialize our study of Markov chains by assuming that our
chain is \emph{temporally homogeneous}, which means that the transition
probabilities do not depend on time; i.e., 
$p_n(\omega_n, B) = p(\omega_n, B)$. (This is the stochastic analogue of a
conservative system.) 
Next we assume that our state space $S$ is finite, and suppose for all states
$i,j \in S$ that $p(i, j) \geq 0$, and $\sum_j p(i, j) = 1$ for all $i$. In
this case, equation~(\ref{eq:2.2}) takes a more intuitive form,
\[
P(X_{n+1}=j \mid X_n = i) =p(i,j),
\]
and our transition probabilities become
\[
p(i,A) = \sum_{j\in A} p(i,j).
\]
% 08.txt
If $\P$ is a matrix whose $(i, j)$ element is the transition probability 
$p(i, j)$ then P is a \emph{stochastic matrix}; that is, a matrix with elements
$p_{ij}$ satisfying 
\[
\pij \geq 0, \quad \sum_j \pij =1, \quad (i, j =1, 2, \dots, d).
\]
We also refer to $\P$ as the transition probability matrix.

Without loss of generality, we can further suppose our Markov chain is \emph{irreducible}. This
means that, for any states $i, j$, starting in state $i$ the chain will make a
transition to state $j$ at some future time with positive probability. This
state of affairs is often described by saying that all states \emph{communicate}. We
lose no generality with this assumption because any \emph{reducible} Markov chain can
be factored into irreducible classes of states which can each be studied
separately. 

The final two conditions we place on the Markov chains considered below will cost us
some generality. Nonetheless, there are many examples of chains meeting these conditions
and making the present study worthwhile. Furthermore, it may be the case that, with a little
more work, we will be able to drop these conditions in future studies. The first condition is that
the chain is \emph{aperiodic}. If we let $I_x = \{n \geq 1: p^n(x,x) > 0\}$, we
call a Markov chain \emph{aperiodic} if, for any state $x$, the greatest common
divisor of $I_x$ is 1. The second assumption is that our chain is 
\emph{reversible}. This characterization is understood in terms of the following definition.

\begin{definition}
%% Definition 2.1.3 
\label{def:2.1.3}
A measure $\mu$ is called \emph{reversible} if it satisfies
\[
\mu(x)p(x,y) = \mu(y)p(y,x), \quad  \text{ for all $x$ and $y$.}
\]
We call a Markov chain \emph{reversible} if its stationary distribution (defined
in Section~\ref{sec:convergence-theorem}) is reversible. 
\end{definition}



% ---------- 09.txt ------------------------------------------------------------

\subsection{A Convergence Theorem}
% Section 2.1.2
\label{sec:convergence-theorem}
In succeeding arguments, we use some results concerning the asymptotic behavior of
Markov chains. These results require a few more definitions.

\begin{definition}
%% Definition 2.1.4
\label{def:2.1.4}
A measure $\pi$ is said to be a \emph{stationary measure} if
\begin{equation}
% Eqn 2.4
  \label{eq:2.4}
\sum_x \pi(x) p(x,y) = \pi(x).
\end{equation}
\end{definition}

Equation~(\ref{eq:2.4}) says $P_\pi(X_1 = y) = \pi(y)$, 
and by induction that $P_\pi(X_n = y) = \pi(y)$ for all $n \geq 1$. 
If $\pi$ is a probability measure, then we call $\pi$ a stationary distribution. It
represents an equilibrium for the chain in the sense that, if $X_0$ has
distribution $\pi$, then so does $X_n$ for all $n$.

When the Markov chain is irreducible and aperiodic, the distribution of the state at time
$n$ converges pointwise to $\pi$ as $n \rightarrow \infty$, regardless of the
initial state. It is convenient to state this convergence result in terms of the
Markov chain's transition probability matrix $\P$. Before doing so, we note that
irreducibility of a Markov chain is equivalent to irreducibility (in the usual 
matrix theory sense) of its transition probability matrix. Furthermore, it turns
out that a transition probability matrix of an aperiodic Markov chain falls into
that class of matrices often called acyclic, but for simplicity we will call
such stochastic matrices aperiodic. With this terminology, we can state the
convergence theorem in terms of the transition probability matrix \P. 

% Theorem 2.1.1
\begin{theorem}
\label{thm-2.1.1}
Suppose $\P$ is irreducible, aperiodic, and has stationary distribution
$\pi$. Then as $n\rightarrow \infty$, $p^n(i,j) \rightarrow \pi(j)$.
\end{theorem}

The notation $p^n(i,j)$ means the $(i,j)$ element of the $n$th power of $\P$.

A Markov chain whose transition probability matrix satisfies the hypotheses of
Theorem~\ref{thm-2.1.1} is called \emph{ergodic}. If we simulate an ergodic chain for
sufficiently many steps, having 
%
%
%
% ---------------10.txt --------------------------------------------------
%
%
%
begun in any initial state, the final state is a sample point from a
distribution that is close to $\pi$. 

To make this statement more precise requires that we define ``close.''
\begin{definition}
%% Definition 2.1.5 
Let $\pi$ be a probability measure on $S$. Then the \emph{total variation distance} at time
$n$ with initial state $x$ is given by 
\[
\Delta_x(n) = \|\P^n(x, A) - \pi(A)\|_{TV} = \max_{A\in \S} | \P^n(x,A)-\pi(A)|.
\]
\end{definition}
In what follows, we will measure rate of convergence using the function
$\tau_x(\epsilon)$, defined as the first time after which the total variation
distance is always less than $\epsilon$. That is, 
\[
\tau_x(\epsilon) = \min\{m : \Delta_x(n) \leq \epsilon \text{ for all } n\geq
m\}.
\]

To begin our consideration of the connection between convergence rates of Markov
chains and eigenvalues, we first note that an aperiodic stochastic matrix $\P$ has real
eigenvalues 
$1 = \lambda_0 > \lambda_1 \geq \lambda_2 \geq \cdots \geq \lambda_{d-1} \geq -1$, 
where $d = |S|$ is the dimension of the state space. For an ergodic chain, the
rate of convergence to the stationary distribution $\pi$ is bounded by a
function of the \emph{subdominant} eigenvalue. By subdominant eigenvalue we mean that
eigenvalue which is second largest in absolute value, and we denote this
eigenvalue by $\lambda_{\max} = \max\{\lambda_1, |\lambda_{d-1}|\}$. 
The function bounding the rate of convergence of a Markov chain is given by the
following theorem ($\log$ denotes the natural logarithm):
\begin{theorem}
% Theorem 2.1.2
\label{thm-2.1.2}
The quantity $\tau_x(\epsilon)$ satisfies
\begin{enumerate}
\item 
$\tau_x(\epsilon) \leq (1-\lambda_{\max})^{-1}(\log \pi(x)^{-1} + \log \epsilon^{-1})$;
\item $\max_{x \in S} \tau_x(\epsilon) \geq \frac{1}{2} \lambda_{\max}(1-\lambda_{\max})^{-1} \log(2\epsilon)^{-1}$.
\end{enumerate}
\end{theorem}
As this theorem shows, if we have an upper bound on the subdominant eigenvalue, then we have
an upper bound on the function $\tau_x(\epsilon)$. In what follows, we will
derive an approximation to the 
% 11.txt
subdominant eigenvalue and supply error bounds. Together, an approximation and error bounds
for $\lambda_{\max}$  provide enough information to make Theorem~\ref{thm-2.1.2} useful.

\section{Functions on the State Space}
% Sec 2.2
\label{sec:funct-state-space}
Recall that $X_n(\omega) = \omega_n\in S$ denotes the state in which the Markov
chain exists at time $n$. 
Suppose that 
$\Phi = \{\phi_1, \dots, \phi_p\}$ 
is a collection of $p$ \emph{observables}, or functions defined on the state
space $S$. Furthermore, let these observables be real valued, 
$\phi_i : S \rightarrow \R$. 
It is often useful to assume that none of the observables is a constant function. Suppose now that
the state space $S$ is finite with $d$ possible states. Then, since an
observable is simply a map of the state space, we can think of each $\phi_i$ as a
vector of $d$ real numbers---the $d$ values that it takes on at the different states.

Now assume the Markov chain is irreducible, and let $\pi$ denote its stationary distribution.
If we start the chain from its stationary distribution---i.e., suppose $X_0$ has
distribution $\pi$---then $X_n$ is a stationary process. 
Furthermore, for each $i$, $\phi_i(X_n)$ is a stationary stochastic process with
\emph{mean}
\[
\bE_\pi\phi_i = \sum_{x\in S} \pi(x)\phi_i(x)
\]
and \emph{autocovariance function}
\begin{align}
% Eqn 2.5
\bC_\pi (\phi_i(X_n),  \phi_i (X_{n+s})) 
&= \bE_\pi[(\phi_i(X_n) - \bE_\pi\phi_i)(\phi_i(X_{n+s}) - \bE_\pi\phi_i)]\\
&= \sum_{x, y\in S} P_\pi(X_n = x, X_{n+s} = y) (\phi_i(x)- \bE_\pi\phi_i)(\phi_i(y) - \bE_\pi\phi_i).\nonumber
\end{align}
By the definition of conditional probability, we can write~(\ref{eq:2.4}) as follows:
\[
\sum_{x, y\in S} P_\pi(X_n = x) P_\pi(X_{n+s} = y\mid X_n = x) (\phi_i(x)-
\bE_\pi\phi_i)(\phi_i(y) - \bE_\pi\phi_i).
\]
Equivalently,
\[
\sum_{x, y\in S} \pi(x)p^s_{xy} (\phi_i(x)-\bE_\pi\phi_i)(\phi_i(y) - \bE_\pi\phi_i).
\]
Here $p^s_{xy}$ denotes the element in row $x$ and column $y$ of $\P^s$, the
$s$th power of the transition probability matrix. 
Similarly, we define the \emph{cross-covariance} between the function 
$\phi_i$ at time $n$ and $\phi_j$ at time $n+s$ as
\begin{align}
% Eqn 2.6
\bC_\pi (\phi_i(X_n),  \phi_j (X_{n+s})) 
&= \bE_\pi[(\phi_i(X_n) - \bE_\pi\phi_i)(\phi_j(X_{n+s}) - \bE_\pi\phi_j)]\nonumber\\
&= \sum_{x, y\in S} \pi(x)p^s_{xy} (\phi_i(x)-\bE_\pi\phi_i)(\phi_j(y) - \bE_\pi\phi_j).
\end{align}
Now let $\<\Phi\>$ denote the matrix of mean vectors whose $j$th column is 
$\bE_\pi\phi_j\one$, where $\one = (1,\dots, 1)^t$,
and let $\Pi = \diag(\pi(\omega_1),\dots, \pi(\omega_d))$ be the $d \times d$
diagonal matrix with stationary probabilities $\pi(\omega)$
on the main diagonal and zeros elsewhere. 
Finally, denoting by $\C(s)$ the $p \times p$ covariance matrix
whose $(i,j)$ element is $\bC_\pi(\phi_i(X_n), \phi_j(X_{n+s}))$, we have
\begin{align*}
\C(0) &= \bE(\Phi(X_n) - \<\Phi\>)(\Phi(X_n) - \<\Phi\>)^t\\
&= (\Phi - \<\Phi\>)^t\Pi (\Phi-\<\Phi\>), \\
\C(s) &= \bE(\Phi(X_n) - \<\Phi\>)(\Phi(X_{n+s}) - \<\Phi\>)^t\\
&= (\Phi - \<\Phi\>)^t \Pi \P^s (\Phi - \<\Phi\>).
\end{align*}

Below, we will also find it useful to have at our disposal a new matrix that is \emph{similar} to the
transition probability matrix. We have in mind the matrix 
$\M = \Pi^{1/2}\P\Pi^{-1/2}$. 
As is easily verified, this allows us to write the covariance matrix as
\begin{align}
\label{eq:2.7}
\C(s) &= 
(\Phi - \<\Phi\>)^t \Pi^{\frac{1}{2}t}\M^s\Pi^{\frac{1}{2}}(\Phi - \<\Phi\>)\nonumber\\
&=\Psi^t \M^s \Psi,  % Eqn 2.7
\end{align}
%% \\Phi\Phi‘M’\\Phi\Phi (2.7)
where we have defined $\Psi = \Pi^{\frac{1}{2}}(\Phi - \<\Phi\>)$. Recall that
our main motivation is that for out-of-core problems traditional eigenvalue
algorithms are inadequate. With this fact and the form~(\ref{eq:2.7}) % (2.7) 
in 
%
%
% ------------------ 13.txt ---------------------------------------------------
%
%
mind, we will consider using the covariance of observables on the state space to implement the
%% (otherwise useless) 
Rayleigh-Ritz procedure, which we describe below. This procedure requires
that $\M$ be symmetric. As the next fact demonstrates, 
%% this required symmetry is the reason for our
%% interest in the reversibility property of the Markov chain.
this need for symmetry is the reason we insist that the Markov chain be reversible.
\begin{fact}
%% Fact 2.2.1 
The matrix $\M$ is symmetric if and only if the Markov chain is reversible (i.e., iff the
process satisfies the \emph{detailed balance} condition).
\end{fact}
\begin{proof}
  \begin{align*}
  \M \text{ is symmetric } 
& \quad \Longleftrightarrow  \quad
 (\Pi^{1/2}\P\Pi^{-1/2})^t = \Pi^{1/2}\P\Pi^{-1/2}\\
& \quad \Longleftrightarrow  \quad
 \Pi^{-\frac{1}{2}t}\P^t\Pi^{\frac{1}{2}t} = \Pi^{1/2}\P\Pi^{-1/2}\\
& \quad \Longleftrightarrow  \quad
 \P^t\Pi^t = \Pi\P.
  \end{align*}
Elementwise, the final equality is $\pi_i\pij = \pi_j\pji$. 
According to Definition~\ref{def:2.1.3}, this states
that $\pi$ is a reversible measure.
\end{proof}

% 14.txt

%%%-------------------------------------------------------------------
%%%
%%% \chapter{Invariant and Approximate Invariant Subspaces}
\label{cha:invar-appr-invar}
\section{Invariant Subspaces}

\begin{definition}
%% Definition 3.1.1
A subspace $S \subseteq \R^n$ with the property that
\[
x \in S \quad \Longrightarrow \quad \M x \in S
\]
is said to be \emph{invariant} for $\M$.
\end{definition}

Recall that, having chosen observables $\<\Phi\> = (\phi_1, \dots, \phi_p)$, we
constructed the covariance matrix 
$\C(s) = \Psi^t \M \Psi$. 
If the column space of $\Psi$, which we denote by $\ran(\Psi)$, is an invariant
subspace for $\M$, the definition implies $\M\psi_j \in \ran(\Psi)$.
That is, for each $\psi_i$ there exists a vector
$t$ of coeflicients such that $\M \psi_j = \sum_{i=1}^p t_i \psi_{ij}$. 
This is true for all $j$ and, putting each vector of
coefficients into a matrix $\T$, we see that 
$\M \Psi = \Psi \T$. Conversely, $\M\Psi = \Psi \T$ implies that $\M\psi_j$ is a
linear combination of columns of $\Psi$, so $\ran(\Psi)$ is invariant. We have
thus proved the following 
% 15.txt
\begin{fact}
%% Fact 3.1.1 
\label{fact:3.1.1}
The subspace $\ran(\Psi)$ is invariant for $\M$ if and only if there exists 
$\T \in \R^{p\times p}$ such that $\M\Psi = \Psi \T$.
\end{fact}

Consequently,
\begin{fact}
%% Fact 3.1.2
\label{fact:3.1.2}
$\lambda(\T) \subseteq \lambda(\M)$.
\end{fact}
\begin{proof}
%Proof of 3.1.2:
~
\vskip-2cm
  \begin{align*}
\lambda \in \lambda(\T)
& \quad \Longleftrightarrow  \quad
(\exists v \in \R^p) \T v = \lambda v\\
& \quad \Longleftrightarrow  \quad
\Psi \T v = \lambda \Psi v\\
& \quad \Longleftrightarrow  \quad
\M \Psi v = \lambda \Psi v\\
& \quad \Longleftrightarrow  \quad
\lambda \in \lambda(M).
  \end{align*}
The second equivalence follows from Fact~\ref{fact:3.1.1}.  %3.1.1.
\end{proof}
%Facts~\ref{fact:3.1.1} and~\ref{fact:3.1.2} are theoretically useful. To see why
To see why Facts~\ref{fact:3.1.1} and~\ref{fact:3.1.2} are theoretically useful,
%consider the equation of Fact~\ref{fact:3.1.1}:
%% \begin{align*}
%% & \Psi \T = \M \Psi \\
%% \Longleftrightarrow \quad & \Psi^t\Psi \T = \Psi^t \M \Psi\\
%% \Longleftrightarrow \quad & \T = (\Psi^t\Psi)^t \Psi^t \M\Psi.
%% \end{align*}
%% \begin{equation*}
%% \Psi \T = \M \Psi \quad\Longleftrightarrow \quad \Psi^t\Psi \T = \Psi^t \M \Psi
%% \Longleftrightarrow \quad  \T = (\Psi^t\Psi)^t \Psi^t \M\Psi.
%% \end{equation*}
note that the equation of Fact~\ref{fact:3.1.1}, $\Psi \T = \M \Psi$, is
equivalent to 
$\Psi^t\Psi \T = \Psi^t \M \Psi$, which is equivalent to 
$\T = (\Psi^t\Psi)^t \Psi^t \M\Psi$.
In this form, we recognize that $\T = \C^{-1}(0)\C(1)$. That is, using only the
covariance of observables on the state space, we can generate a matrix $\T$ with
the property $\lambda(\T) \subseteq \lambda(\M)$. Recalling that 
$\M = \Pi^{1/2}\P\Pi^{-1/2}$, 
we see that $\M$ is similar to our original stochastic matrix $\P$, 
and thus $\lambda(M) = \lambda(P)$.

\section{Approximate Invariant Subspaces}
% Sec 3.2
\label{sec:appr-invar-subsp}
%% We qualified the foregoing by stating that the facts are only theoretically
%% useful. This
We noted above that Facts~\ref{fact:3.1.1} and~\ref{fact:3.1.2} are
theoretically useful. Speaking practically now,
%% This is because 
when choosing observables on the state space we may not be sure that
they will satisfy the 
%
%
% -------------16.txt---------------------------------------------------
%
%
primary assumption underlying the two facts. %Facts 3.1.1 and 3.1.2. 
Recall the assumption: $\ran(\Psi)$ is an invariant
subspace for $\M$ where $\Psi = \Pi^{1/2}(\Phi - \<\Phi\>)$. It may well be the
case that there exists $\psi_j\in \ran(\Psi)$ such that 
$\M\psi_j \notin \ran(\Psi)$, thereby violating the assumption. 
Even if we are lacking an invariant subspace, however, for some applications it
is reasonable to expect that observables can be chosen to provide at least
an \emph{approximate invariant subspace}, which is defined as follows:
\begin{definition}
%% Definition 3.2.1
If the columns of $\Psi\in \R^{d\times p}$ are independent and the norm of the
\emph{residual matrix} $\E = \M\Psi - \Psi \T$ is small for some  
$\T \in \R^{p\times p}$, then
$\ran(\Psi)$ defines an \emph{approximate invariant subspace}. 
\end{definition}
To see how an approximate invariant subspace can be useful for approximating
eigenvalues of $\M$, we recall a theorem from Golub and Van Loan~\cite{Golub:1996}.
\begin{theorem}
% Theorem 3.2.1
\label{thm:3.2.1}  
Suppose $\M \in R^{d\times d}$ and $\T \in \R^{p\times p}$ are symmetric and let
$\E = \M\Q - \Q\T$, where $\Q \in \R^{d\times p}$ is orthonormal 
(i.e., $\Q^t\Q = \I$). Then there exist $\mu_1,\dots, \mu_p \in \lambda(\T)$ and
$\lambda_1, \dots, \lambda_p \in \lambda(\M)$ such that
\[
|\mu_k - \lambda_k| \leq \sqrt{2}\|\E\|_2, \quad \text{ for $k = 1, \dots, p$.}
\]
\end{theorem}
If the subspace $\ran(\Q)$ is an approximate invariant subspace, then the definition implies that
there is a choice $\T$ rendering the error $\|\E\|_2$ small, and thus the
eigenvalues of $\T$ provide a good approximation to those of $\M$.

When considering the foregoing ideas, it is apparent that their application
presents new---but hopefully less prohibitive---problems. As these problems
are the focus of the rest of the paper, now is a good time to examine
them. 

First, the preceding theorem assumes a matrix $\Q$ whose columns form an
orthonormal basis for the approximate invariant subspace. For our problem, 
derivation of such a $\Q$ is tricky, and we must prepare for this. 

Next, having an approximate invariant subspace at our disposal merely tells us
that there exists some matrix $\T$ which makes the error 
%
%
% ------------- 17.txt ----------------------------------------------------
%
%
$\|\E\|_2$ small. We must discover the form of such a $\T$. Moreover, it is
natural to seek that $\T$ which minimizes $\|\E\|_2$ for a given approximate
invariant subspace. 

Finally, in order to apply these ideas to realistic eigenvalue problems, we must
find a practical way to generate %% the most appropriate
a good approximate invariant subspace. 

We now address each of these issues in turn.
%the order raised.

Recall the matrix $\Psi = \Pi^{1/2}(\Phi — \<\Phi\>)$. The columns of this
matrix, though independent (by choice of independent observables), are not
necessarily orthonormal. However, consider the polar decomposition 
$\Psi = \Q \Z$, where $\Q^t\Q = \I$ and $\Z^2 = \Psi^t\Psi$ is a symmetric
positive semidefinite matrix.\footnote{Recall that the polar decompostion is
  derived from the singular value decomposition, $\Psi = \U\Sigma \V^t$, by
  letting $\Q = \U \V^t$ and $\Z = \V\Sigma \V^t$.}
%% ‘Recall that the polar decomposition is derived from the SVD, \Psi = UZJV‘, by letting Q = UV‘ and Z = VEV‘.
Note that $\Z = (\Psi^t\Psi)^{1/2}$ is nonsingular, so $\Q$ has the form
\[
\Q = \Psi \Z^{-1} = \Psi(\Psi^t\Psi)^{-1/2},
\]
and it is clear that $\ran(\Q) = \ran(\Psi)$. 
Perhaps $\ran(\Q)$ is a useful approximation to the invariant
subspace for $\M$. If $\ran(Q)$ is not itself invariant, we have the error 
matrix $\E =\M\Q - \Q\T$.  Below we show that the
$\T$ which minimizes $\|\E\|_2$ is $\T = \Q^t\M\Q$. 
This yields the following
\begin{theorem}
%Theorem 3.2.2
\label{thm:3.2.2}  
If $\Psi = \Q\Z$ is the polar decomposition of $\Psi$, then the matrix
\[
\T = \Q^t\M\Q = (\Psi^t\Psi)^{-1/2}\Psi^t \M^t \Psi (\Psi^t\Psi)^{-1/2}
\]
minimizes $\|\E\|_2 = \|\M\Q - \Q\T\|_2$.
\end{theorem}
\begin{proof}
We prove the result by establishing the following
\\[6pt]
\underline{Claim:} Given $\M \in \R^{d\times d}$ suppose $Q \in  \R^{d\times d}$
satisfies $\Q^t\Q =\I$. 
Then,
\[
\min_{\T\in \R^{p\times p}} \|\M\Q - \Q\T\|_2 = \|(\I- \Q\Q^t)\M\Q\|_2,
\]
and $\T = \Q^t\M\Q$ is the minimizer.
%
%
% -------------- 18.txt -------------------------------
%
%
The claim is verified by an easy application of the Pythagorean theorem:
For any $\T \in \R^{p\times p}$ we have
\begin{align}
\label{eq:3.1}
\|\M\Q - \Q\T\|^2_2 &= \|\M\Q - \Q\Q^t\M\Q + \Q\Q^t\M\Q - \Q\T\|^2_2 \nonumber\\
&=\|(\I - \Q\Q^t)\M\Q + \Q(\Q^t\M\Q - \T)\|^2_2 \nonumber \\
&=\|(\I - \Q\Q^t)\M\Q\|^2_2 + \|\Q(\Q^t\M\Q - \T)\|^2_2 \\
&=\|(\I - \Q\Q^t)\M\Q\|^2_2 \nonumber
\end{align}
Equality~(\ref{eq:3.1}) holds since $\I - \Q\Q^t$ projects $\M\Q$ onto the
subspace orthogonal to 
$\ran(Q)$. Thus, the two terms in the expression on the right are orthogonal,
and the Pythagorean theorem yields equality. The concluding inequality
establishes that the minimizing $\T$ is that which annihilates the 
second term in~(\ref{eq:3.1}), that is, $\T = \Q^t\M\Q$.

The second equality in Theorem~\ref{thm:3.2.2} is a consequence of the polar decomposition, in
which $\Q = \Psi \Z^{-1} = \Psi(\Psi^t\Psi)^{-1/2}$. Thus,
$\T = (\Psi^t\Psi)^{-1/2}\Psi^t \M^t \Psi (\Psi^t\Psi)^{-1/2}$
is the minimizer, as claimed.
\end{proof}

\subsection{The Krylov Subspace}
% Sec 3.3
\label{sec:krylov-subspace}
Suppose the columns of a matrix $\Q \in \R^{d\times p}$ give an orthonormal
basis for an approximate invariant subspace. Then, as we have seen, 
\begin{enumerate}
\item 
$\T = \Q^t\M\Q$ minimizes $\|\E\|_2 = \|\M\Q - \Q\T\|_2$ and
\item there exist $\mu_1, \dots, \mu_p\in \lambda(\T)$ and 
$\lambda_1,\dots, \lambda_p \in \lambda(\M)$ such that
\[
\|\mu_k-\lambda_k\| \leq \sqrt{2}\|\E\|_2, \quad \text{ for $k=1,\dots, p$.}
\]
\end{enumerate}
%
%
% ---------------- 19.txt --------------------------------------------
%
%
Given an approximate invariant subspace $\ran(\Q)$ of dimension $p$, these facts
tell us what matrix we should use to approximate $p$ elements of the spectrum of
$\M$. Now all we lack is a description of $\ran(\Q)$. 
That is, we have not specified which approximate invariant subspace would best
suit our objective of approximating the subdominant eigenvalue
$\lambda_{\max}(\M)$. For this purpose the following definition is useful:
\begin{definition}
%% Definition 3.3.1
The \emph{Raleigh quotient} of a symmetric matrix $\M$ and a nonzero vector $x$ is
\[
\rho(x, \M) = \frac{x^t\M x}{x^tx}.
\]
\end{definition}
We will denote Raleigh quotient by $\rho(x)$ when the context makes clear what
matrix is involved. 

To find the approximate invariant subspace most appropriate for our problem, we
choose each dimension successively, providing justification at each step. 
We start with one observable $\phi$ on the state space, and let 
$\psi_1 = \Pi^{1/2}(\phi -\E_\pi\phi)$. That is, $\psi_1$ is a centered (mean
zero) observable whose $i$th coordinate is weighted by $\sqrt{\pi(i)}$. 
Notice that the definition 
%% (which comes directly from the definition of ^tII
%% following equation (2.7)) is such that 1/)1 is a constant vector if 
(which comes from the definition of $\Psi$ following Equation~(\ref{eq:2.7})) 
is such that $\psi_1$ is a constant vector if and only if $\phi$ is oonstant on the
state space, in which case $\psi_1$ is the constant zero function. 
%% This fact
%% provides one reason observables that are constant on the state space are not
%% interesting. 
(Observables that are constant on the state space are not interesting.) 
Second, recall that the row sums of the matrix $\P$ are all one, therefore the
eigenvector corresponding to the eigenvalue $\lambda_0(\P) = 1$ is the constant
vector.  By definition of $\M = \Pi^{1/2}\P\Pi^{-1/2}$, we see that the constant
vector is also the eigenvector corresponding to $\lambda_0$.  This will play an
important role in what follows, as it allows us to focus %primarily 
on the subdominant eigenvalue 
$\lambda_{\max}(\M) = \max\{\lambda_1(\M), |\lambda_{d-1}(\M)|\}$
rather than on $\lambda_0(M)$ (which we already know is 1).

Now, notice that
\begin{equation}
  \label{eq:3.2}
|\rho(\psi_1,\M)| \leq \max |\rho(x,\M)| = \lambda_{\max}(\M)
\end{equation}
% 20.txt
where the max is taken over all nonconstant vectors. Since our interest 
centers on 
$\lambda_{\max}(\M)$, 
we would like a $\psi_1$ that makes the left hand side of~(\ref{eq:3.2}) large. 
This would be achieved if $\psi_1$ were to lie in the space spanned by, say, the
first two eigenvectors of the Markov chain (sometimes referred 
to as the \emph{slowest modes} of the process). However, this subspace spans
only two dimensions of the entire $d$-dimensional space, and it is more likely
that $\psi_1$ only comes close, at best, to lying in the subspace of
interest. Now, given $\psi_1$, a judicious choice for the second dimension $\psi_2$, and
hence $\Psi_2 = [\psi_1, \psi_2]$, would be that which makes 
$\max_{a\neq 0} |\rho(\Psi_2 a)|$ as large as possible. To establish that 
this is indeed the right objective, note the following:
\begin{align}
\label{eq:3.3}
\max_{a\neq 0}|\rho(\Psi_2a,\M)| &= \max_{a\neq 0} \left|\frac{a^t \Psi^t_2 \M \Psi_2 a}{a^t \Psi^t_2 \Psi_2 a}\right|\nonumber\\
&= \max_{x\in \ran(\Psi_2)}|\rho(x,\M)|\nonumber\\
&= \max|\rho(x,\M)|\\
&= \lambdamax\nonumber
\end{align}
Again, the max on the right side of~(\ref{eq:3.3}) is over nonconstant vectors. 
In other words, we wish to chose
$\Psi_2 = [\psi_1, \psi_2]$ so that there is a vector $a \in \R^2$ making 
$|\rho(\Psi_2 a,\M)|$ close to $\lambdamax$.

Now, $\rho(\psi_1)$ changes most rapidly in the direction of the gradient
$\nabla \rho(\psi_1)$.
\begin{equation}
  \label{eq:3.4}
\nabla \rho(\psi_1) = \left(\frac{\partial \rho(\psi_1)}{\partial \psi_1(1)}, \dots,
\frac{\partial \rho(\psi_1)}{\partial \psi_1(d)}\right)
 = \frac{2}{\psi_1^t\psi_1} (\M\psi_1 - \rho(\psi_1)\psi_1).
\end{equation}
So, to maximize the left hand side of~(\ref{eq:3.3}), $\Psi_2$ should be chosen
so that the subspace $\ran(\Psi_2)$ contains the gradient vector.
That is, we must choose $\psi_2$ so that
\begin{equation}
\label{eq:3.5}
\nabla \rho(\psi_1) \in \ran\{\psi_1, \psi_2\} = \ran(\Psi_2).
\end{equation}
Clearly, Equation~(\ref{eq:3.4}) implies 
$\nabla \rho(\psi_1) \in \ran\{\psi_1, \M\psi_1\}$.
Thus, if $\ran\{\psi_1, \psi_2\} = \ran\{\psi_1, \M\psi_1\}$, 
then~(\ref{eq:3.5}) is satisfied.
%
%
% ------------ 21.txt ----------------------------------------------
%
%
In general, having chosen $\Psi_k = [\psi_1, \dots,  \psi_k]$ so that
$\ran\{\psi_1, \psi_2, \dots, \psi_k\} = \ran\{\psi_1, \M\psi_1, \dots,
\M\psi_k\}$,
we must chose $\psi_{k+1}$ so that for any nonzero vector $a\in \R^k$,
\begin{equation}
\label{eq:3.6}
\nabla \rho(\Psi_ka) \in \ran\{\psi_1, \psi_2, \dots, \psi_k\}.
\end{equation}
Now,
\[
\nabla \rho(\Psi_ka)   
 = \frac{2}{a^t\Psi_k^t\Psi_ka} (\M\Psi_k - \rho(\Psi_ka)\Psi_ka).
\]
and therefore,
\[
\nabla \rho(\Psi_ka)  \in 
\ran(\M\Psi_k) \cup \ran(\Psi_k) = \ran\{\psi_1, \M\psi_1, \dots, \M^k\psi_1\}.
\]
Thus, the requirement~(\ref{eq:3.6}) is satisfied when 
\[
\ran\{\psi_1, \psi_2, \dots, \psi_k\} = \ran\{\psi_1, \M\psi_1, \dots,
\M^k\psi_1\}.
\]

In conclusion, the $p$-dimensional approximate invariant subspace that is most suitable
for our objective is
\[
\sK(\M, \psi_1, p) = \ran\{\psi_1, \M\psi_1, \dots, \M^{p-1}\psi_1\}.
\]
This is known as the \emph{Krylov subspace}. Therefore, to answer the problem posed at the outset
of this section, if we take the columns of $\Q$ to be an orthonormal basis for 
$\sK(\M,\psi_1, p)$, then the eigenvalues of $\T = \Q^t\M\Q$ should provide good
estimates of $p$ extremal eigenvalues of $\M$. 
%
%
% --------------- 22.txt ---------------------------------------
%
%
 % Chapter 3
%%%
\chapter{Invariant and Approximate Invariant Subspaces}
\label{cha:invar-appr-invar}
\section{Invariant Subspaces}

\begin{definition}
%% Definition 3.1.1
A subspace $V \subseteq \R^n$ with the property that
\[
v\in V \quad \Longrightarrow \quad \M v \in V
\]
is said to be \emph{invariant} for $\M$.
\end{definition}

Recall that, having chosen observables $\<\Phi\> = (\phi_1, \dots, \phi_p)$, we
constructed the covariance matrix 
$\C(s) = \Psi^t \M \Psi$. 
If the column space of $\Psi$, which we denote by $\ran(\Psi)$, is an invariant
subspace for $\M$, the definition implies $\M\psi_j \in \ran(\Psi)$.
That is, for each $\psi_i$ there exists a vector
$t$ of coefficients such that $\M \psi_j = \sum_{i=1}^p t_i \psi_{ij}$. 
This is true for all $j$ and, putting each vector of
coefficients into a matrix $\T$, we see that 
$\M \Psi = \Psi \T$. Conversely, $\M\Psi = \Psi \T$ implies that $\M\psi_j$ is a
linear combination of columns of $\Psi$, so $\ran(\Psi)$ is invariant. We have
thus proved the following 
% 15.txt
\begin{fact}
%% Fact 3.1.1 
\label{fact:3.1.1}
The subspace $\ran(\Psi)$ is invariant for $\M$ if and only if there exists 
$\T \in \R^{p\times p}$ such that $\M\Psi = \Psi \T$.
\end{fact}

Consequently,
\begin{fact}
%% Fact 3.1.2
\label{fact:3.1.2}
$\lambda(\T) \subseteq \lambda(\M)$.
\end{fact}
\begin{proof}
%Proof of 3.1.2:
~
\vskip-2cm
  \begin{align*}
\lambda \in \lambda(\T)
& \quad \Longleftrightarrow  \quad
(\exists v \in \R^p) \T v = \lambda v\\
& \quad \Longleftrightarrow  \quad
\Psi \T v = \lambda \Psi v\\
& \quad \Longleftrightarrow  \quad
\M \Psi v = \lambda \Psi v\\
& \quad \Longrightarrow  \quad
\lambda \in \lambda(M).
  \end{align*}
The second equivalence follows from Fact~\ref{fact:3.1.1}.  %3.1.1.
\end{proof}
%Facts~\ref{fact:3.1.1} and~\ref{fact:3.1.2} are theoretically useful. To see why
To see why Facts~\ref{fact:3.1.1} and~\ref{fact:3.1.2} are theoretically useful,
%consider the equation of Fact~\ref{fact:3.1.1}:
%% \begin{align*}
%% & \Psi \T = \M \Psi \\
%% \Longleftrightarrow \quad & \Psi^t\Psi \T = \Psi^t \M \Psi\\
%% \Longleftrightarrow \quad & \T = (\Psi^t\Psi)^t \Psi^t \M\Psi.
%% \end{align*}
%% \begin{equation*}
%% \Psi \T = \M \Psi \quad\Longleftrightarrow \quad \Psi^t\Psi \T = \Psi^t \M \Psi
%% \Longleftrightarrow \quad  \T = (\Psi^t\Psi)^t \Psi^t \M\Psi.
%% \end{equation*}
note that the equation of Fact~\ref{fact:3.1.1}, $\Psi \T = \M \Psi$, is
equivalent to 
$\Psi^t\Psi \T = \Psi^t \M \Psi$, which is equivalent to 
$\T = (\Psi^t\Psi)^t \Psi^t \M\Psi$.
In this form, we recognize that $\T = \C^{-1}(0)\C(1)$. That is, using only the
covariance of observables on the state space, we can generate a matrix $\T$ with
the property $\lambda(\T) \subseteq \lambda(\M)$. Recalling that 
$\M = \Pi^{1/2}\P\Pi^{-1/2}$, 
we see that $\M$ is similar to our original stochastic matrix $\P$, 
and thus $\lambda(M) = \lambda(P)$.

\section{Approximate Invariant Subspaces}
% Sec 3.2
\label{sec:appr-invar-subsp}
%% We qualified the foregoing by stating that the facts are only theoretically
%% useful. This
We noted above that Facts~\ref{fact:3.1.1} and~\ref{fact:3.1.2} are
theoretically useful. Speaking practically now,
%% This is because 
when choosing observables on the state space we may not be sure that
they will satisfy the 
%
%
% -------------16.txt---------------------------------------------------
%
%
primary assumption underlying the two facts. %Facts 3.1.1 and 3.1.2. 
Recall the assumption: $\ran(\Psi)$ is an invariant
subspace for $\M$ where $\Psi = \Pi^{1/2}(\Phi - \<\Phi\>)$. It may well be the
case that there exists $\psi_j\in \ran(\Psi)$ such that 
$\M\psi_j \notin \ran(\Psi)$, thereby violating the assumption. 
Even if we are lacking an invariant subspace, however, for some applications it
is reasonable to expect that observables can be chosen to provide at least
an \emph{approximate invariant subspace}, which is defined as follows:
\begin{definition}
%% Definition 3.2.1
If the columns of $\Psi\in \R^{d\times p}$ are independent and the norm of the
\emph{residual matrix} $\E = \M\Psi - \Psi \T$ is small for some  
$\T \in \R^{p\times p}$, then
$\ran(\Psi)$ defines an \emph{approximate invariant subspace} for $\M$. 
\end{definition}
To see how an approximate invariant subspace can be useful for approximating
eigenvalues of $\M$, we recall a theorem from Golub and Van Loan~\cite{Golub:1996}.
\begin{theorem}
% Theorem 3.2.1
\label{thm:3.2.1}  
Suppose $\M \in R^{d\times d}$ and $\T \in \R^{p\times p}$ are symmetric and let
$\E = \M\Q - \Q\T$, where $\Q \in \R^{d\times p}$ is orthonormal 
(i.e., $\Q^t\Q = \I$). Then there exist $\mu_1,\dots, \mu_p \in \lambda(\T)$ and
$\lambda_1, \dots, \lambda_p \in \lambda(\M)$ such that
\[
|\mu_k - \lambda_k| \leq \sqrt{2}\|\E\|_2, \quad \text{ for } \quad k = 1, \dots, p.
\]
\end{theorem}
If the subspace $\ran(\Q)$ is an approximate invariant subspace, then the definition implies that
there is a choice $\T$ rendering the error $\|\E\|_2$ small, and thus the
eigenvalues of $\T$ provide a good approximation to those of $\M$.

When considering the foregoing ideas, it is apparent that their application
presents new---but hopefully less prohibitive---problems. As these problems
are the focus of the rest of the paper, now is a good time to examine
them. 

First, the preceding theorem assumes a matrix $\Q$ whose columns form an
orthonormal basis for the approximate invariant subspace. For our problem, 
derivation of such a $\Q$ is tricky, and we must prepare for this. 

Next, having an approximate invariant subspace at our disposal merely tells us
that there exists some matrix $\T$ which makes the error 
%
%
% ------------- 17.txt ----------------------------------------------------
%
%
$\|\E\|_2$ small. We must discover the form of such a $\T$. Moreover, it is
natural to seek that $\T$ which minimizes $\|\E\|_2$ for a given approximate
invariant subspace. 

Finally, in order to apply these ideas to realistic eigenvalue problems, we must
find a practical way to generate %% the most appropriate
a good approximate invariant subspace. 

We now address each of these issues in turn.
%the order raised.

Recall the matrix $\Psi = \Pi^{1/2}(\Phi - \<\Phi\>)$. The columns of this
matrix, though independent (by choice of independent observables), are not
necessarily orthonormal. However, consider the polar decomposition 
$\Psi = \Q \Z$, where $\Q^t\Q = \I$ and $\Z^2 = \Psi^t\Psi$ is a symmetric
positive semidefinite matrix.\footnote{Recall that the polar decomposition is
  derived from the singular value decomposition, $\Psi = \U\Sigma \V^t$, by
  letting $\Q = \U \V^t$ and $\Z = \V\Sigma \V^t$.}
%% ‘Recall that the polar decomposition is derived from the SVD, \Psi = UZJV‘, by letting Q = UV‘ and Z = VEV‘.
Note that $\Z = (\Psi^t\Psi)^{1/2}$ is nonsingular, so $\Q$ has the form
\[
\Q = \Psi \Z^{-1} = \Psi(\Psi^t\Psi)^{-1/2},
\]
and it is clear that $\ran(\Q) = \ran(\Psi)$. 
Perhaps $\ran(\Q)$ is a useful approximation to the invariant
subspace for $\M$. If $\ran(Q)$ is not itself invariant, we have the error 
matrix $\E =\M\Q - \Q\T$.  Below we show that the
$\T$ which minimizes $\|\E\|_2$ is $\T = \Q^t\M\Q$. 
This yields the following
\begin{theorem}
%Theorem 3.2.2
\label{thm:3.2.2}  
If $\Psi = \Q\Z$ is the polar decomposition of $\Psi$, then the matrix
\[
\T = \Q^t\M\Q = (\Psi^t\Psi)^{-1/2}\Psi^t \M^t \Psi (\Psi^t\Psi)^{-1/2}
\]
minimizes $\|\E\|_2 = \|\M\Q - \Q\T\|_2$.
\end{theorem}
\begin{proof}
We prove the result by establishing the following
\\[6pt]
\underline{Claim:} Given $\M \in \R^{d\times d}$ suppose $Q \in  \R^{d\times d}$
satisfies $\Q^t\Q =\I$. 
Then,
\[
\min_{\T\in \R^{p\times p}} \|\M\Q - \Q\T\|_2 = \|(\I- \Q\Q^t)\M\Q\|_2,
\]
and $\T = \Q^t\M\Q$ is the minimizer.
%
%
% -------------- 18.txt -------------------------------
%
%
The claim is verified by an easy application of the Pythagorean theorem:
For any $\T \in \R^{p\times p}$ we have
\begin{align}
\label{eq:3.1}
\|\M\Q - \Q\T\|^2_2 &= \|\M\Q - \Q\Q^t\M\Q + \Q\Q^t\M\Q - \Q\T\|^2_2 \nonumber\\
&=\|(\I - \Q\Q^t)\M\Q + \Q(\Q^t\M\Q - \T)\|^2_2 \nonumber \\
&=\|(\I - \Q\Q^t)\M\Q\|^2_2 + \|\Q(\Q^t\M\Q - \T)\|^2_2 \\
&=\|(\I - \Q\Q^t)\M\Q\|^2_2 \nonumber
\end{align}
Equality~(\ref{eq:3.1}) holds since $\I - \Q\Q^t$ projects $\M\Q$ onto the
subspace orthogonal to 
$\ran(Q)$. Thus, the two terms in the expression on the right are orthogonal,
and the Pythagorean theorem yields equality. The concluding inequality
establishes that the minimizing $\T$ is that which annihilates the 
second term in~(\ref{eq:3.1}), that is, $\T = \Q^t\M\Q$.

The second equality in Theorem~\ref{thm:3.2.2} is a consequence of the polar decomposition, in
which $\Q = \Psi \Z^{-1} = \Psi(\Psi^t\Psi)^{-1/2}$. Thus,
$\T = (\Psi^t\Psi)^{-1/2}\Psi^t \M^t \Psi (\Psi^t\Psi)^{-1/2}$
is the minimizer, as claimed.
\end{proof}

\subsection{The Krylov Subspace}
% Sec 3.3
\label{sec:krylov-subspace}
Suppose the columns of a matrix $\Q \in \R^{d\times p}$ give an orthonormal
basis for an approximate invariant subspace. Then, as we have seen, 
\begin{enumerate}
\item 
$\T = \Q^t\M\Q$ minimizes $\|\E\|_2 = \|\M\Q - \Q\T\|_2$ and
\item there exist $\mu_1, \dots, \mu_p\in \lambda(\T)$ and 
$\lambda_1,\dots, \lambda_p \in \lambda(\M)$ such that
\[
\|\mu_k-\lambda_k\| \leq \sqrt{2}\|\E\|_2, \quad \text{ for $k=1,\dots, p$.}
\]
\end{enumerate}
%
%
% ---------------- 19.txt --------------------------------------------
%
%
Given an approximate invariant subspace $\ran(\Q)$ of dimension $p$, these facts
tell us what matrix we should use to approximate $p$ elements of the spectrum of
$\M$. Now all we lack is a description of $\ran(\Q)$. 
That is, we have not specified which approximate invariant subspace would best
suit our objective of approximating the subdominant eigenvalue
$\lambda_{\max}(\M)$. For this purpose the following definition is useful:
\begin{definition}
%% Definition 3.3.1
The \emph{Raleigh quotient} of a symmetric matrix $\M$ and a nonzero vector $x$ is
\[
\rho(x, \M) = \frac{x^t\M x}{x^tx}.
\]
\end{definition}
We will denote Raleigh quotient by $\rho(x)$ when the context makes clear what
matrix is involved. 

To find the approximate invariant subspace most appropriate for our problem, we
choose each dimension successively, providing justification at each step. 
We start with one observable $\phi$ on the state space, and let 
$\psi_1 = \Pi^{1/2}(\phi -\E_\pi\phi)$. That is, $\psi_1$ is a centered (mean
zero) observable whose $i$th coordinate is weighted by $\sqrt{\pi(i)}$. 
Notice that the definition 
%% (which comes directly from the definition of ^tII
%% following equation (2.7)) is such that 1/)1 is a constant vector if 
(which comes from the definition of $\Psi$ following Equation~(\ref{eq:2.7})) 
is such that $\psi_1$ is a constant vector if and only if $\phi$ is constant on the
state space, in which case $\psi_1$ is the constant zero function. 
%% This fact
%% provides one reason observables that are constant on the state space are not
%% interesting. 
(Observables that are constant on the state space are not interesting.) 
Second, recall that the row sums of the matrix $\P$ are all one, therefore the
eigenvector corresponding to the eigenvalue $\lambda_0(\P) = 1$ is the constant
vector.  By definition of $\M = \Pi^{1/2}\P\Pi^{-1/2}$, we see that the constant
vector is also the eigenvector corresponding to $\lambda_0$.  This will play an
important role in what follows, as it allows us to focus %primarily 
on the subdominant eigenvalue 
$\lambda_{\max}(\M) = \max\{\lambda_1(\M), |\lambda_{d-1}(\M)|\}$
rather than on $\lambda_0(M)$ (which we already know is 1).

Now, notice that
\begin{equation}
  \label{eq:3.2}
|\rho(\psi_1,\M)| \leq \max |\rho(x,\M)| = \lambda_{\max}(\M)
\end{equation}
% 20.txt
where the max is taken over all nonconstant vectors. Since our interest 
centers on 
$\lambda_{\max}(\M)$, 
we would like a $\psi_1$ that makes the left hand side of~(\ref{eq:3.2}) large. 
This would be achieved if $\psi_1$ were to lie in the space spanned by, say, the
first two eigenvectors of the Markov chain (sometimes referred 
to as the \emph{slowest modes} of the process). However, this subspace spans
only two dimensions of the entire $d$-dimensional space, and it is more likely
that $\psi_1$ only comes close, at best, to lying in the subspace of
interest. Now, given $\psi_1$, a judicious choice for the second dimension $\psi_2$, and
hence $\Psi_2 = [\psi_1, \psi_2]$, would be that which makes 
$\max_{a\neq 0} |\rho(\Psi_2 a)|$ as large as possible. To establish that 
this is indeed the right objective, note the following:
\begin{align}
\label{eq:3.3}
\max_{a\neq 0}|\rho(\Psi_2a,\M)| &= \max_{a\neq 0} \left|\frac{a^t \Psi^t_2 \M \Psi_2 a}{a^t \Psi^t_2 \Psi_2 a}\right|\nonumber\\
&= \max_{x\in \ran(\Psi_2)}|\rho(x,\M)|\nonumber\\
&= \max|\rho(x,\M)|\\
&= \lambdamax\nonumber
\end{align}
Again, the max on the right side of~(\ref{eq:3.3}) is over nonconstant vectors. 
In other words, we wish to chose
$\Psi_2 = [\psi_1, \psi_2]$ so that there is a vector $a \in \R^2$ making 
$|\rho(\Psi_2 a,\M)|$ close to $\lambdamax$.

Now, $\rho(\psi_1)$ changes most rapidly in the direction of the gradient
$\nabla \rho(\psi_1)$.
\begin{equation}
  \label{eq:3.4}
\nabla \rho(\psi_1) = \left(\frac{\partial \rho(\psi_1)}{\partial \psi_1(1)}, \dots,
\frac{\partial \rho(\psi_1)}{\partial \psi_1(d)}\right)
 = \frac{2}{\psi_1^t\psi_1} (\M\psi_1 - \rho(\psi_1)\psi_1).
\end{equation}
So, to maximize the left hand side of~(\ref{eq:3.3}), $\Psi_2$ should be chosen
so that the subspace $\ran(\Psi_2)$ contains the gradient vector.
That is, we must choose $\psi_2$ so that
\begin{equation}
\label{eq:3.5}
\nabla \rho(\psi_1) \in \ran\{\psi_1, \psi_2\} = \ran(\Psi_2).
\end{equation}
Clearly, Equation~(\ref{eq:3.4}) implies 
$\nabla \rho(\psi_1) \in \ran\{\psi_1, \M\psi_1\}$.
Thus, if $\ran\{\psi_1, \psi_2\} = \ran\{\psi_1, \M\psi_1\}$, 
then~(\ref{eq:3.5}) is satisfied.
%
%
% ------------ 21.txt ----------------------------------------------
%
%
In general, having chosen $\Psi_k = [\psi_1, \dots,  \psi_k]$ so that
$\ran\{\psi_1, \psi_2, \dots, \psi_k\} = \ran\{\psi_1, \M\psi_1, \dots,
\M\psi_k\}$,
we must chose $\psi_{k+1}$ so that for any nonzero vector $a\in \R^k$,
\begin{equation}
\label{eq:3.6}
\nabla \rho(\Psi_ka) \in \ran\{\psi_1, \psi_2, \dots, \psi_k\}.
\end{equation}
Now,
\[
\nabla \rho(\Psi_ka)   
 = \frac{2}{a^t\Psi_k^t\Psi_ka} (\M\Psi_k - \rho(\Psi_ka)\Psi_ka).
\]
and therefore,
\[
\nabla \rho(\Psi_ka)  \in 
\ran(\M\Psi_k) \cup \ran(\Psi_k) = \ran\{\psi_1, \M\psi_1, \dots, \M^k\psi_1\}.
\]
Thus, the requirement~(\ref{eq:3.6}) is satisfied when 
\[
\ran\{\psi_1, \psi_2, \dots, \psi_k\} = \ran\{\psi_1, \M\psi_1, \dots,
\M^k\psi_1\}.
\]

In conclusion, the $p$-dimensional approximate invariant subspace that is most suitable
for our objective is
\[
\sK(\M, \psi_1, p) = \ran\{\psi_1, \M\psi_1, \dots, \M^{p-1}\psi_1\}.
\]
This is known as the \emph{Krylov subspace}. Therefore, to answer the problem posed at the outset
of this section, if we take the columns of $\Q$ to be an orthonormal basis for 
$\sK(\M,\psi_1, p)$, then the eigenvalues of $\T = \Q^t\M\Q$ should provide good
estimates of $p$ extremal eigenvalues of $\M$. 
%
%
% --------------- 22.txt ---------------------------------------
%
%


%%%-------------------------------------------------------------------
%%%
%%% \chapter{Lanczos Procedures}

The following will review the usual procedure for generating an orthonormal basis for
$\sK(\M, \psi_1, p)$, 
and conclude by showing that, for our problem, we can find the matrix 
$\Q^t\M\Q$
without actually carrying out this procedure. Note that this conclusion is essential since the procedure
requires matrix vector multiplication---an operation we have assumed impossible for large enough $d$.

\section{The General Lanczos Algorithm}
% Sec 4.1
\label{sec:gener-lancz-algor}
Let $\psip{k} =  \M^k \psi$ and consider the $d\times d$ matrix
\[
\Psi = [\psi, \M\psi, \dots, \M^{d-1}\psi] = [\psip{0}, \psip{1}, \dots, \psip{d-1}]
\]
Notice that $\M\Psi = [\psip{1}, \psip{2}, \dots, \psip{d-1}, \M^d \psi]$,
and assuming for the moment that $\Psi$ is nonsingular,
we can compute the vector $h = -\Psi^{-1}\M^d \psi$. Thus,
\begin{equation}
\label{eq:10000}
\M\Psi = \Psi [e_2,\dots,e_d,-h],  
\end{equation}
where $e_j$ is the column vector with 1 in
the $j$th position and zeros elsewhere. We define $H = [e_2,\dots,e_d,-h]$,
so~(\ref{eq:10000}) becomes $\M\Psi = \Psi H$.
%
%
% ----------------- 23.txt ----------------------------------------------
%
%
Equivalently,
\begin{equation}
  \label{eq:4.1}
H = 
\begin{pmatrix}
0 & 0      & \cdots & 0 & -h_1 \\
1 & 0      & \cdots & 0 & -h_2\\
  & \ddots &        &   & \vdots \\
  &        & \ddots & 0 & -h_{d-1}\\
  &        &        & 1 & -h_{d}
\end{pmatrix} = \Psi^{-1} \M \Psi.
\end{equation}
$\H$ is a \emph{companion matrix} which means that its characteristic polynomial
is $p(x) = x^d + \sum_{i=1}^d h_i x^{i-1}$.
Since $\H$ is similar to $\M$, finding the eigenvalues of $\M$ is equivalent to
finding the roots of $p(x)$. However, this is of little practical use since
finding $h$, constructing $p(x)$, and finding its roots is 
probably a harder problem than the one we started with. Instead, the value of decomposition (4.1)
derives from its \emph{upper Hessenberg form}. We exploit this property below.

Let $\Psi = \Q\mR$ be the QR decomposition of $\Psi$. Since $\Psi$ is assumed nonsingular,
\[
\Psi_d^{-1}\M \Psi_d = (\mR^{-1}\Q^t)\M(\Q\mR) = \H
\]
Therefore, $\Q^t\M\Q = \mR\H\mR^{-1}$.  Let $\T \mR\H\mR^{-1}$.  Then, since $\mR$
and $\mR^{-1}$ are both upper triangular and $\H$ is upper Hessenberg $\T = \mR\H\mR^{-1}$ is also upper
Hessenberg. Furthermore, since $\M$ is symmetric, it is clear that 
$\T^t =\Q^t\M\Q = T$. Thus, $\T$ is both upper Hessenberg and symmetric. Therefore,
$\T$ is tridiagonal and we can write it as follows:
\begin{equation}
  \label{eq:4.2}
\T = 
\begin{pmatrix}
\alpha_1 & \beta_1 &      0 & \cdots & 0 \\
\beta_1  & \alpha_2 & \ddots &        & \vdots\\
         & \ddots & \ddots & \ddots   &\\
\vdots   &       &  \ddots & \ddots & \beta_{d-1}\\
0  &   \cdots    &   & \beta_{d-1} &\alpha_d
\end{pmatrix}.
\end{equation}
%
%
% ------------ 24.txt ----------------------------------------------
%
%
Equating columns $j$ on both sides of $\M\Q = \Q\T$ yields
\begin{equation}
\label{eq:4.3}
\M q_j = \beta_{j-1}q_{j-1}+\alpha_j q_j + \beta_j q_{j+1}
\end{equation}
Since the columns of $\Q$ are orthonormal, multiplying both sides 
of~(\ref{eq:4.3})
by $q_j$ and $q_{j+1}$ yields
$\alpha_j = q_j^t \M q_j$ and $\beta_j = q_{j+1}^t \M q_j$.
The foregoing justifies what is called the \emph{Lanczos algorithm}, which is performed as follows:
         \begin{algorithm}
           \caption{Lanczos algorithm for partial reduction to symmetric
             tridiaganal form}
           \begin{algorithmic}
             \STATE $q_1 = \psi/\|\psi\|_2$, $\beta_0 = 0$, $q_0= 0$
             \FOR{$j = 1, \dots, p$}
             \STATE $z \leftarrow \M q_j$
             \STATE $\alpha_j \leftarrow q_j^tz$
             \STATE $z \leftarrow z - \alpha_j q_j - \beta_{j-1} q_{j-1}$
             \STATE $\beta_j \leftarrow \|z\|_2$
             \IF{$\beta_j=0$}
             \STATE return
             \ENDIF
             \STATE $q_{j+1} = z/\beta_j$
             \ENDFOR
           \end{algorithmic}
         \end{algorithm}
The $q_j$ computed by the Lanczos algorithm are often called the \emph{Lanczos vectors}. If the
loop in the algorithm is terminated because $\beta_{k}= 0$, this indicates that
an exact invariant subspace has been computed, and is given by 
$\ran\{q_1, \dots, q_k\}$. Otherwise, we usually halt the algorithm after 
$p$ steps, in which case the algorithm converges to approximations of at most
  $p$ eigenvalues.\footnote{A Detailed discussion of such convergence (and
    misconvergence) is given in~\cite{Demmel:1997}.}

Note that Equation~(\ref{eq:4.3}) results from the equation $\M\Q = \Q\T$. The latter equation holds
since, starting with $d$ vectors in our subspace, $\Q$ is a $d \times d$ matrix
whose columns form an orthonormal basis for all of $\R^d$, which is clearly an
invariant subspace. The Lanczos algorithm above, however, 
%
%
% ------------ 25.txt ---------------------------------------------------------
%
%
proceeds for only $p$ steps, producing a $d \times p$ matrix $\Q$, whose columns
form an orthonormal basis for the \emph{approximate} invariant subspace 
$\sK(\M, \psi, p)$, and a $p \times p$ matrix $\T, = \Q^t\M\Q$. In that case,
$\|\M\Q - \Q\T_p\|_2 = \|\E\|_2$ is nonzero and gives the error bound described
in Theorem~\ref{thm:3.2.1}. Now, writing the full $d \times d$ matrix $\T$
of~(\ref{eq:4.2}) as
\begin{equation}
  \label{eq:4.4}
\T = 
\left( \begin{array}{c|c}
\T_p & \T_{pu}^t \\
 \hline
\T_{pu} & \T_{u}
\end{array}
\right)
=
\left( \begin{array}{cccc|cccc}
\alpha_1 & \beta_1 &      &                 & & && \\
\beta_1 & \ddots  & \ddots&                 & & && \\
        & \ddots  & \ddots  & \beta_{p-1}    &          & && \\
        &         & \beta_{p-1}  & \alpha_{p} & \beta_{p}& && \\
\hline
        &         &             & \beta_{p}  & \alpha_{p+1} & \beta_{p+1} && \\
        &         &             &           & \beta_{p+1}  & \ddots & \ddots &\\
        &         &             &           &             & \ddots & \ddots &\beta_{d-1} \\
        &         &             &           &             &        & \beta_{d-1} &\alpha_{d}
\end{array}
\right)
\end{equation}
allows us to describe the error bound $\|\E\|_2$ in terms of the submatrix
$\T_{pu}$, and hence in terms of $\beta_p$.
\begin{theorem}
%Theorem 4.1.1
\label{thm:4.1.1}
If $\T_p$ and $\T_{pu}$ are the matrices appearing in~(\ref{eq:4.4}), and if the $p$
columns of $\Q$ are computed by the 
Lanczos algorithm, then there exist $\mu_1, \dots, \mu_p \in \lambda(\T_p)$ and 
$\lambda_1, \dots, \lambda_p \in \lambda(\M)$
such that
\[
|\mu_k - \lambda_k| \leq \|\T_{pu}\|_2 = \beta_p, \quad \text{ for $k=1,\dots,p$.}
\]
\end{theorem}
For a proof, see~\cite[Page 365]{Demmel:1997}.
%
%
% --------------- 26.txt -------------------------
%
%
\section{A Lanczos Procedure for Markov Chains}
Notice that the Lanczos algorithm requires the computation of $\M q_j$. Throughout this
paper we have assumed that $\M \in \R^{d\times d}$, and that $d$ is so large as
to make the matrix vector multiplication $\M q_j$ impossible. Now, the Lanczos
algorithm constructs an orthonormal basis for the Krylov subspace. For our
purposes, this construction is not essential. What is essential is that we find
a way to generate the Lanczos coeficients $\alpha_i$, $\beta_j$, $(j = 1,\dots, p)$, and
whence the matrix $\T$, without performing the operation $\M q_j$ required by the
Lanczos algorithm. We now address this problem.

Begin with the centered and weighted observable 
$\psi = \Pi^{1/2} (\phi - \bE_\pi\phi)$,
and let $q_1 = \psi/\|\psi\|_2$.
To simplify notation, let $\phi(n) = \phi(X_n)$. 
The form of the first Lanczos coeflicient $\alpha_1$ is straight
forward:
\[
\alpha_1 = q_1^t \M q_1  = \frac{\psi^t\M\psi}{\psi^t\psi}.
\]
and now the special properties of the Markov chain setting play an important
role. Indeed, we have
\begin{align}
\label{eq:4.5}
\psi^t\M\psi &= (\phi - \bE_\pi\phi)^t\Pi^{\frac{1}{2}t}\M\Pi^{\frac{1}{2}}(\phi- \bE_\pi \phi)\nonumber\\
&= \bC_\pi (\phi(n), \phi(n + 1)).
\end{align}
and
\begin{align}
\label{eq:4.6}
\psi^t\psi &= (\phi - \bE_\pi\phi)^t\Pi(\phi- \bE_\pi \phi)\nonumber\\
&= \Var_\pi(\phi(n)).
\end{align}
Recall that the definition of $\M$ implies 
$\Pi^{\frac{1}{2}t}\M\Pi^{\frac{1}{2}}= \Pi \P$, and this justifies 
Equation~(\ref{eq:4.5}). Putting
%
%
% -------------- 27.txt ------------------------------
%
%
these equations together, we have a nice form for $\alpha_1$:
\[
\alpha_1= \frac{\bC_\pi (\phi(n), \phi(n + 1))}{\Var_\pi(\phi(n))}.
\]
The next Lanczos coefficient, $\beta_1$, is only slightly more complicated.
\begin{align}
\label{eq:4.7}
\beta_1 &= \|\M q_1 - \alpha_1 q_1\|_2 \nonumber\\
&= \bigl[(\M q_1 - \alpha_1 q_1)^t(\M q_1 - \alpha_1 q_1)\bigr]^{1/2}\nonumber\\
&= \bigl[q_1^t\M^2 q_1 - \alpha^2_1\bigr]^{1/2}.
\end{align}
Again recalling the definition $\M =\Pi^{1/2}\P\Pi^{1/2}$, a simple manipulation verifies
$\Pi^{\frac{1}{2}t}\M^s\Pi^{\frac{1}{2}} = \Pi \P^s$,
whereby it follows that
\begin{align*}
q_1^t\M^2 q_1  &= \frac{(\phi - \bE_\pi\phi)^t\Pi \P^2(\phi- \bE_\pi \phi)}
{(\phi - \bE_\pi\phi)^t\Pi (\phi- \bE_\pi \phi)} \\
&= \frac{\bC_\pi (\phi(n), \phi(n + 2))}{\Var_\pi(\phi(n))}.
\end{align*}
Inserting the last expression into Equation~(\ref{eq:4.7}) gives
\[
\beta_1 = \left[\frac{\bC_\pi (\phi(n), \phi(n + 2))}{\Var_\pi(\phi(n))}
-\bigl( \frac{\bC_\pi (\phi(n), \phi(n + 1))}{\Var_\pi(\phi(n))}\bigr)^2
\right]^{1/2}.
\]
Notice that we can estimate these expressions of $\alpha_1$ and $\beta_1$ by
running simulations of the Markov chain. 
To do so, we run the Markov chain starting from a random initial state, compute the value of
the observable at each step of the chain and, from the data, compute estimates of the covariances
and variance of the observable.

Before generalizing this for the $j$th Lanczos coefficient, we consider one more coeflicient
in detail. Recall that the second basis element $q_2$ in the Lanczos algorithm is given by 
$q_2 = (\M q_1 — \alpha_1 q_1)/\beta_1$. 
From this we can derive an expression for $\alpha_2$ which, while not as neat as those
%
%
% ----------------- 28.txt ----------------------------------------------------------
%
%
for $\alpha_1$ and $\beta_1$, is nonetheless tractable.
\begin{align}
\label{eq:4.8}
\alpha_2 &= q_2^t \M q_2 \nonumber\\
&= \frac{1}{\beta_1^2}(\M q_1 - \alpha_1 q_1)^t\M (\M q_1 - \alpha_1 q_1)\nonumber\\
&= \frac{1}{\beta_1^2}(q_1^t\M^3 q_1 - 2\alpha_1 q_1^t\M^2 q_1 + \alpha_1^2q_1^t\M q_1 )\nonumber\\
&= \frac{1}{\beta_1^2}(q_1^t\M^3 q_1 - 2\alpha_1 \beta_1^2 + \alpha_1^3 )
&= \bigl[q_1^t\M^2 q_1 - \alpha^2_1\bigr]^{1/2}.
\end{align}
Equation~(\ref{eq:4.8}) follows from Equation~(\ref{eq:4.7}), which can be
written as $q_1^t\M^2 q_1 = \alpha_1^2 + \beta_1^2$.  
The preceding paragraphs define all but one of the quantities in the last
exression of~(\ref{eq:4.8}). The one remaining
is $q_1^t\M^3 q_1$, which can also be expressed in terms of covariances of $\phi$,
and the analysis is almost identical to that for $q_1^t\M^2q_1$:
\begin{align*}
q_1^t\M^3 q_1  &= \frac{(\phi - \bE_\pi\phi)^t\Pi \P^3(\phi- \bE_\pi \phi)}
{(\phi - \bE_\pi\phi)^t\Pi (\phi- \bE_\pi \phi)} \\
&= \frac{\bC_\pi (\phi(n), \phi(n + 3))}{\Var_\pi(\phi(n))}.
\end{align*}
To generalize the foregoing for higher order Lanczos coeflicients requires a recurrence
relation between $(\alpha_{j+1}, \beta_j)$ and $(\alpha_{j}, \beta_{j-1})$. 
As we will see, such a relation involves higher order
Lanczos vectors $q_j$---vectors we do not wish to compute. Recall that the only ``vector'' we can
handle is our initial observable. Therefore, we must also establish a recurrence
among the $q_j$'s, thus making them available inductively from our observable vector.

To begin with the $\beta_j$'s, recall that 
$\beta_j = \|\M q_j - \alpha_jq_j - \beta_{j-1}q_{j-1}\|_2$. 
Whence,
\begin{align}
\label{eq:4.9}
\beta_j^2 &= (\M q_j - \alpha_jq_j - \beta_{j-1}q_{j-1})^t(\M q_j - \alpha_jq_j - \beta_{j-1}q_{j-1})\nonumber\\
&= q_j^t\M^2 q_j - \alpha_j^2 - \beta^2_{j-1}.
\end{align}
Some simple algebra and the definitions $\alpha_j = q_j^t\M q_j$ and
$\beta_{j-1} = q_j^t \M q_{j-1}$ prove equality~(\ref{eq:4.9}).  The
%
%
% -------------- 29.txt ------------------------------------------------------
%
%
recurrence for $\alpha_{j+1}$ is equally straight forward, though the final form
is not as neat as the expression above. Indeed,
\begin{align}
\label{eq:4.10}
\alpha_{j+1} &= q_{j+1}^t \M q_{j+1} \nonumber\\
&= \frac{1}{\beta_j^2}(\M q_j - \alpha_j q_j - \beta_{j-1} q_{j-1})^t\M 
  (\M q_j - \alpha_j q_j - \beta_{j-1} q_{j-1})\nonumber\\
&= \frac{1}{\beta_j^2}(q_j^t\M^3 q_j 
    - 2\alpha_j q_j^t\M^2 q_j 
    - 2\beta^2_{j-1} q_{j-1}^t\M^2 q_j 
    + \alpha_1^3
    + 2\alpha_j \beta^2_{j-1} + \beta^2_{j-1}).
\end{align}
From~(\ref{eq:4.9}), $\alpha_j^2 +\beta_j^2 + \beta^2_{j-1}$,
and upon plugging this into~(\ref{eq:4.10}), we arrive at 
\begin{equation}
\label{eq:4.11}
\alpha_{j+1}= \frac{1}{\beta_j^2}(q_j^t\M^3 q_j 
    - \alpha_j^3
    - 2\alpha_j\beta^2_{j}
    - 2\beta_{j-1}q_{j-1}^t\M^2q_j + \beta^2_{j-1}).
\end{equation}

The expressions above involve the higher order Lanczos vectors $q_j$. In the present context,
these vectors will not be readily available. However, another recursion---this time relating forms
involving $q_{j+1}$ to forms involving $q_j$ and $q_{j-1} $---will make these
expressions accessible by induction on our initial observable. The general forms
of interest are $q_{j+1}^t\M^k q_{j+1}$ and $q^t_{j+1}\M^k q_j$. Again, we 
begin with the simpler of the two:
\begin{align*}
  q_{j+1}^tM^kq_j &= 
 \frac{1}{\beta_j}(\M q_j - \alpha_j q_j - \beta_{j-1} q_{j-1})^t\M^k q_j\\
 \frac{1}{\beta_j}(q_j^t\M^{k+1} q_j - \alpha_j q_j^t\M^{k} q_j -\beta_{j-1} q_j^t\M^k).
\end{align*}
The second is,
\begin{align}
\label{eq:4.12}
  q_{j+1}^tM^kq_{j+1} 
&=  \frac{1}{\beta_j^2}(\M q_j - \alpha_j q_j - \beta_{j-1} q_{j-1})^t\M^k 
  (\M q_j - \alpha_j q_j - \beta_{j-1} q_{j-1})\nonumber\\
&=  \frac{1}{\beta_j^2}(q_j^t\M^{k+2} q_j + \alpha^2_jq_j^t\M^{k} q_j 
  + \beta^2_{j-1} q_{j-1}^t\M^k q_{j-1} 
    - 2\alpha_jq_j^t\M^{k+1} q_j
    + 2\alpha_j \beta_{j-1}q_j^t\M^k q_{j-1}
    - 2\beta_{j-1}q_j^t \M^{k+1}q_{j-1}).
\end{align}
The point of all this is that each pair of coeflicients $(\alpha_{j+1}, \beta_j)$
can be expressed as a function of the covariance between 
$\phi(n)$ and $\phi(n+ m)$, for $m = 0, \dots, 2j+1$. Since the expressions
are based on four recurrence relations, they become quite complicated as $j$ gets large. However,
defining these recursions in a computer program is trivial, and we use this fact
in Section~\label{sec:computations} % Sec 5.4
when applying the foregoing to a particular problem.
%
%
% ----------------- 30.txt ------------------------------------------------------
%
%

 % Chapter 4
%%%
\chapter{Lanczos Procedures}
\label{cha:lanczos-procedures}
The following will review the usual procedure for generating an orthonormal basis for
$\sK(\M, \psi_1, p)$, 
and conclude by showing that, for our problem, we can find the matrix 
$\Q^t\M\Q$
without actually carrying out this procedure. Note that this conclusion is essential since the procedure
requires matrix vector multiplication---an operation we have assumed impossible for large enough $d$.

\section{The General Lanczos Algorithm}
% Sec 4.1
\label{sec:gener-lancz-algor}
Let $\psip{k} =  \M^k \psi$ and consider the $d\times d$ matrix
\[
\Psi = [\psi, \M\psi, \dots, \M^{d-1}\psi] = [\psip{0}, \psip{1}, \dots, \psip{d-1}]
\]
Notice that $\M\Psi = [\psip{1}, \psip{2}, \dots, \psip{d-1}, \M^d \psi]$,
and assuming for the moment that $\Psi$ is nonsingular,
we can compute the vector $h = -\Psi^{-1}\M^d \psi$. Thus,
\begin{equation}
\label{eq:10000}
\M\Psi = \Psi [e_2,\dots,e_d,-h],  
\end{equation}
where $e_j$ is the column vector with 1 in
the $j$th position and zeros elsewhere. We define $H = [e_2,\dots,e_d,-h]$,
so~(\ref{eq:10000}) becomes $\M\Psi = \Psi H$.
%
%
% ----------------- 23.txt ----------------------------------------------
%
%
Equivalently,
\begin{equation}
  \label{eq:4.1}
H = 
\begin{pmatrix}
0 & 0      & \cdots & 0 & -h_1 \\
1 & 0      & \cdots & 0 & -h_2\\
  & \ddots &        &   & \vdots \\
  &        & \ddots & 0 & -h_{d-1}\\
  &        &        & 1 & -h_{d}
\end{pmatrix} = \Psi^{-1} \M \Psi.
\end{equation}
$\H$ is a \emph{companion matrix} which means that its characteristic polynomial
is $p(x) = x^d + \sum_{i=1}^d h_i x^{i-1}$.
Since $\H$ is similar to $\M$, finding the eigenvalues of $\M$ is equivalent to
finding the roots of $p(x)$. However, this is of little practical use since
finding $h$, constructing $p(x)$, and finding its roots is 
probably a harder problem than the one we started with. Instead, the value of decomposition (4.1)
derives from its \emph{upper Hessenberg form}. We exploit this property below.

Let $\Psi = \Q\mR$ be the QR decomposition of $\Psi$. Since $\Psi$ is assumed nonsingular,
\[
\Psi_d^{-1}\M \Psi_d = (\mR^{-1}\Q^t)\M(\Q\mR) = \H
\]
Therefore, $\Q^t\M\Q = \mR\H\mR^{-1}$.  Let $\T \mR\H\mR^{-1}$.  Then, since $\mR$
and $\mR^{-1}$ are both upper triangular and $\H$ is upper Hessenberg $\T = \mR\H\mR^{-1}$ is also upper
Hessenberg. Furthermore, since $\M$ is symmetric, it is clear that 
$\T^t =\Q^t\M\Q = T$. Thus, $\T$ is both upper Hessenberg and symmetric. Therefore,
$\T$ is tridiagonal and we can write it as follows:
\begin{equation}
  \label{eq:4.2}
\T = 
\begin{pmatrix}
\alpha_1 & \beta_1 &      0 & \cdots & 0 \\
\beta_1  & \alpha_2 & \ddots &        & \vdots\\
         & \ddots & \ddots & \ddots   &\\
\vdots   &       &  \ddots & \ddots & \beta_{d-1}\\
0  &   \cdots    &   & \beta_{d-1} &\alpha_d
\end{pmatrix}.
\end{equation}
%
%
% ------------ 24.txt ----------------------------------------------
%
%
Equating columns $j$ on both sides of $\M\Q = \Q\T$ yields
\begin{equation}
\label{eq:4.3}
\M q_j = \beta_{j-1}q_{j-1}+\alpha_j q_j + \beta_j q_{j+1}
\end{equation}
Since the columns of $\Q$ are orthonormal, multiplying both sides 
of~(\ref{eq:4.3})
by $q_j$ and $q_{j+1}$ yields
$\alpha_j = q_j^t \M q_j$ and $\beta_j = q_{j+1}^t \M q_j$.
The foregoing justifies what is called the \emph{Lanczos algorithm}, which is performed as follows:
         \begin{algorithm}
           \caption{The Lanczos algorithm for partial reduction to symmetric
             tridiagonal form.}
           \begin{algorithmic}
             \STATE $q_1 = \psi/\|\psi\|_2$, $\beta_0 = 0$, $q_0= 0$
             \FOR{$j = 1, \dots, p$}
             \STATE $z \leftarrow \M q_j$
             \STATE $\alpha_j \leftarrow q_j^tz$
             \STATE $z \leftarrow z - \alpha_j q_j - \beta_{j-1} q_{j-1}$
             \STATE $\beta_j \leftarrow \|z\|_2$
             \IF{$\beta_j=0$}
             \STATE return
             \ENDIF
             \STATE $q_{j+1} = z/\beta_j$
             \ENDFOR
           \end{algorithmic}
         \end{algorithm}
The $q_j$ computed by the Lanczos algorithm are often called the \emph{Lanczos vectors}. If the
loop in the algorithm is terminated because $\beta_{k}= 0$, this indicates that
an exact invariant subspace has been computed, and is given by 
$\ran\{q_1, \dots, q_k\}$. Otherwise, we usually halt the algorithm after 
$p$ steps, in which case the algorithm converges to approximations of at most
  $p$ eigenvalues.\footnote{A Detailed discussion of such convergence (and
    misconvergence) is given in~\cite{Demmel:1997}.}

Note that Equation~(\ref{eq:4.3}) results from the equation $\M\Q = \Q\T$. The latter equation holds
since, starting with $d$ vectors in our subspace, $\Q$ is a $d \times d$ matrix
whose columns form an orthonormal basis for all of $\R^d$, which is clearly an
invariant subspace. The Lanczos algorithm above, however, 
%
%
% ------------ 25.txt ---------------------------------------------------------
%
%
proceeds for only $p$ steps, producing a $d \times p$ matrix $\Q$, whose columns
form an orthonormal basis for the \emph{approximate} invariant subspace 
$\sK(\M, \psi, p)$, and a $p \times p$ matrix $\T, = \Q^t\M\Q$. In that case,
$\|\M\Q - \Q\T_p\|_2 = \|\E\|_2$ is nonzero and gives the error bound described
in Theorem~\ref{thm:3.2.1}. Now, writing the full $d \times d$ matrix $\T$
of~(\ref{eq:4.2}) as
\begin{equation}
  \label{eq:4.4}
\T = 
\left( \begin{array}{c|c}
\T_p & \T_{pu}^t \\
 \hline
\T_{pu} & \T_{u}
\end{array}
\right)
=
\left( \begin{array}{cccc|cccc}
\alpha_1 & \beta_1 &      &                 & & && \\
\beta_1 & \ddots  & \ddots&                 & & && \\
        & \ddots  & \ddots  & \beta_{p-1}    &          & && \\
        &         & \beta_{p-1}  & \alpha_{p} & \beta_{p}& && \\
\hline
        &         &             & \beta_{p}  & \alpha_{p+1} & \beta_{p+1} && \\
        &         &             &           & \beta_{p+1}  & \ddots & \ddots &\\
        &         &             &           &             & \ddots & \ddots &\beta_{d-1} \\
        &         &             &           &             &        & \beta_{d-1} &\alpha_{d}
\end{array}
\right)
\end{equation}
allows us to describe the error bound $\|\E\|_2$ in terms of the submatrix
$\T_{pu}$, and hence in terms of $\beta_p$.
\begin{theorem}
%Theorem 4.1.1
\label{thm:4.1.1}
If $\T_p$ and $\T_{pu}$ are the matrices appearing in~(\ref{eq:4.4}), and if the $p$
columns of $\Q$ are computed by the 
Lanczos algorithm, then there exist $\mu_1, \dots, \mu_p \in \lambda(\T_p)$ and 
$\lambda_1, \dots, \lambda_p \in \lambda(\M)$
such that
\[
|\mu_k - \lambda_k| \leq \|\T_{pu}\|_2 = \beta_p, \quad \text{ for $k=1,\dots,p$.}
\]
\end{theorem}
For a proof, see~\cite[Page 365]{Demmel:1997}.
%
%
% --------------- 26.txt -------------------------
%
%
\section{A Lanczos Procedure for Markov Chains}
% Sec 4.2
\label{sec:lancz-proc-mark}
Notice that the Lanczos algorithm requires the computation of $\M q_j$. Throughout this
paper we have assumed that $\M \in \R^{d\times d}$, and that $d$ is so large as
to make the matrix vector multiplication $\M q_j$ impossible. Now, the Lanczos
algorithm constructs an orthonormal basis for the Krylov subspace. For our
purposes, this construction is not essential. What is essential is that we find
a way to generate the Lanczos coefficients $\alpha_i$, $\beta_j$, $(j = 1,\dots, p)$, and
whence the matrix $\T$, without performing the operation $\M q_j$ required by the
Lanczos algorithm. We now address this problem.

Begin with the centered and weighted observable 
$\psi = \Pi^{1/2} (\phi - \bE_\pi\phi)$,
and let $q_1 = \psi/\|\psi\|_2$.
To simplify notation, let $\phi(n) = \phi(X_n)$. 
The form of the first Lanczos coefficient $\alpha_1$ is straight
forward:
\[
\alpha_1 = q_1^t \M q_1  = \frac{\psi^t\M\psi}{\psi^t\psi}.
\]
and now the special properties of the Markov chain setting play an important
role. Indeed, we have
\begin{align}
\label{eq:4.5}
\psi^t\M\psi &= (\phi - \bE_\pi\phi)^t\Pi^{\frac{1}{2}t}\M\Pi^{\frac{1}{2}}(\phi- \bE_\pi \phi)\nonumber\\
&= \bC_\pi (\phi(n), \phi(n + 1)).
\end{align}
and
\begin{align}
\label{eq:4.6}
\psi^t\psi &= (\phi - \bE_\pi\phi)^t\Pi(\phi- \bE_\pi \phi)\nonumber\\
&= \Var_\pi(\phi(n)).
\end{align}
Recall that the definition of $\M$ implies 
$\Pi^{\frac{1}{2}t}\M\Pi^{\frac{1}{2}}= \Pi \P$, and this justifies 
Equation~(\ref{eq:4.5}). Putting
%
%
% -------------- 27.txt ------------------------------
%
%
these equations together, we have a nice form for $\alpha_1$:
\[
\alpha_1= \frac{\bC_\pi (\phi(n), \phi(n + 1))}{\Var_\pi(\phi(n))}.
\]
The next Lanczos coefficient, $\beta_1$, is only slightly more complicated.
\begin{align}
\label{eq:4.7}
\beta_1 &= \|\M q_1 - \alpha_1 q_1\|_2 \nonumber\\
&= \bigl[(\M q_1 - \alpha_1 q_1)^t(\M q_1 - \alpha_1 q_1)\bigr]^{1/2}\nonumber\\
&= \bigl[q_1^t\M^2 q_1 - \alpha^2_1\bigr]^{1/2}.
\end{align}
Again recalling the definition $\M =\Pi^{1/2}\P\Pi^{1/2}$, a simple manipulation verifies
$\Pi^{\frac{1}{2}t}\M^s\Pi^{\frac{1}{2}} = \Pi \P^s$,
whereby it follows that
\begin{align*}
q_1^t\M^2 q_1  &= \frac{(\phi - \bE_\pi\phi)^t\Pi \P^2(\phi- \bE_\pi \phi)}
{(\phi - \bE_\pi\phi)^t\Pi (\phi- \bE_\pi \phi)} \\[8pt]
&= \frac{\bC_\pi (\phi(n), \phi(n + 2))}{\Var_\pi(\phi(n))}.
\end{align*}
Inserting the last expression into Equation~(\ref{eq:4.7}) gives
\[
\beta_1 = \left[\frac{\bC_\pi (\phi(n), \phi(n + 2))}{\Var_\pi(\phi(n))}
-\left( \frac{\bC_\pi (\phi(n), \phi(n + 1))}{\Var_\pi(\phi(n))}\right)^2
\right]^{1/2}.
\]
Notice that we can estimate these expressions of $\alpha_1$ and $\beta_1$ by
running simulations of the Markov chain. 
To do so, we run the Markov chain starting from a random initial state, compute the value of
the observable at each step of the chain and, from the data, compute estimates of the covariances
and variance of the observable.

Before generalizing this for the $j$th Lanczos coefficient, we consider one more coefficient
in detail. Recall that the second basis element $q_2$ in the Lanczos algorithm is given by 
$q_2 = (\M q_1 - \alpha_1 q_1)/\beta_1$. 
From this we can derive an expression for $\alpha_2$ which, while not as neat as those
%
%
% ----------------- 28.txt ----------------------------------------------------------
%
%
for $\alpha_1$ and $\beta_1$, is nonetheless tractable.
\begin{align}
\label{eq:4.8}
\alpha_2 &= q_2^t \M q_2 \nonumber\\
&= \frac{1}{\beta_1^2}(\M q_1 - \alpha_1 q_1)^t\M (\M q_1 - \alpha_1 q_1)\nonumber\\
&= \frac{1}{\beta_1^2}(q_1^t\M^3 q_1 - 2\alpha_1 q_1^t\M^2 q_1 + \alpha_1^2q_1^t\M q_1 )\nonumber\\
&= \frac{1}{\beta_1^2}(q_1^t\M^3 q_1 - 2\alpha_1 \beta_1^2 + \alpha_1^3 )
%&= \bigl[q_1^t\M^2 q_1 - \alpha^2_1\bigr]^{1/2}.
\end{align}
Equation~(\ref{eq:4.8}) follows from Equation~(\ref{eq:4.7}), which can be
written as $q_1^t\M^2 q_1 = \alpha_1^2 + \beta_1^2$.  
The preceding paragraphs define all but one of the quantities in the last
expression of~(\ref{eq:4.8}). The one remaining
is $q_1^t\M^3 q_1$, which can also be expressed in terms of covariances of $\phi$,
and the analysis is almost identical to that for $q_1^t\M^2q_1$:
\begin{align*}
q_1^t\M^3 q_1  &= \frac{(\phi - \bE_\pi\phi)^t\Pi \P^3(\phi- \bE_\pi \phi)}
{(\phi - \bE_\pi\phi)^t\Pi (\phi- \bE_\pi \phi)} \\[8pt]
&= \frac{\bC_\pi (\phi(n), \phi(n + 3))}{\Var_\pi(\phi(n))}.
\end{align*}
To generalize the foregoing for higher order Lanczos coefficients requires a recurrence
relation between $(\alpha_{j+1}, \beta_j)$ and $(\alpha_{j}, \beta_{j-1})$. 
As we will see, such a relation involves higher order
Lanczos vectors $q_j$---vectors we do not wish to compute. Recall that the only ``vector'' we can
handle is our initial observable. Therefore, we must also establish a recurrence
among the $q_j$'s, thus making them available inductively from our observable vector.

To begin with the $\beta_j$'s, recall that 
$\beta_j = \|\M q_j - \alpha_jq_j - \beta_{j-1}q_{j-1}\|_2$. 
Whence,
\begin{align}
\label{eq:4.9}
\beta_j^2 &= (\M q_j - \alpha_jq_j - \beta_{j-1}q_{j-1})^t(\M q_j - \alpha_jq_j - \beta_{j-1}q_{j-1})\nonumber\\
&= q_j^t\M^2 q_j - \alpha_j^2 - \beta^2_{j-1}.
\end{align}
Some simple algebra and the definitions $\alpha_j = q_j^t\M q_j$ and
$\beta_{j-1} = q_j^t \M q_{j-1}$ prove equality~(\ref{eq:4.9}).  The
%
%
% -------------- 29.txt ------------------------------------------------------
%
%
recurrence for $\alpha_{j+1}$ is equally straight forward, though the final form
is not as neat as the expression above. Indeed,
\begin{align}
\label{eq:4.10}
\alpha_{j+1} &= q_{j+1}^t \M q_{j+1} \nonumber\\
&= \frac{1}{\beta_j^2}(\M q_j - \alpha_j q_j - \beta_{j-1} q_{j-1})^t\M 
  (\M q_j - \alpha_j q_j - \beta_{j-1} q_{j-1})\nonumber\\
&= \frac{1}{\beta_j^2}(q_j^t\M^3 q_j 
    - 2\alpha_j q_j^t\M^2 q_j 
    - 2\beta^2_{j-1} q_{j-1}^t\M^2 q_j 
    + \alpha_1^3
    + 2\alpha_j \beta^2_{j-1} + \beta^2_{j-1}).
\end{align}
From~(\ref{eq:4.9}), $\alpha_j^2 +\beta_j^2 + \beta^2_{j-1}$,
and upon plugging this into~(\ref{eq:4.10}), we arrive at 
\begin{equation}
\label{eq:4.11}
\alpha_{j+1}= \frac{1}{\beta_j^2}(q_j^t\M^3 q_j 
    - \alpha_j^3
    - 2\alpha_j\beta^2_{j}
    - 2\beta_{j-1}q_{j-1}^t\M^2q_j + \beta^2_{j-1}).
\end{equation}

The expressions above involve the higher order Lanczos vectors $q_j$. In the present context,
these vectors will not be readily available. However, another recursion---this time relating forms
involving $q_{j+1}$ to forms involving $q_j$ and $q_{j-1} $---will make these
expressions accessible by induction on our initial observable. The general forms
of interest are $q_{j+1}^t\M^k q_{j+1}$ and $q^t_{j+1}\M^k q_j$. Again, we 
begin with the simpler of the two:
\begin{align*}
  q_{j+1}^tM^kq_j &= 
 \frac{1}{\beta_j}(\M q_j - \alpha_j q_j - \beta_{j-1} q_{j-1})^t\M^k q_j\\
 \frac{1}{\beta_j}(q_j^t\M^{k+1} q_j - \alpha_j q_j^t\M^{k} q_j -\beta_{j-1} q_j^t\M^k).
\end{align*}
The second is,
\begin{align}
\label{eq:4.12}
  q_{j+1}^tM^kq_{j+1} 
&=  \frac{1}{\beta_j^2}(\M q_j - \alpha_j q_j - \beta_{j-1} q_{j-1})^t\M^k 
  (\M q_j - \alpha_j q_j - \beta_{j-1} q_{j-1})\nonumber\\
&=  \frac{1}{\beta_j^2}(q_j^t\M^{k+2} q_j + \alpha^2_jq_j^t\M^{k} q_j 
  + \beta^2_{j-1} q_{j-1}^t\M^k q_{j-1} - 2\alpha_jq_j^t\M^{k+1} q_j \nonumber\\
& \qquad    
    + 2\alpha_j \beta_{j-1}q_j^t\M^k q_{j-1}
    - 2\beta_{j-1}q_j^t \M^{k+1}q_{j-1}).
\end{align}
The point of all this is that each pair of coefficients $(\alpha_{j+1}, \beta_j)$
can be expressed as a function of the covariance between 
$\phi(n)$ and $\phi(n+ m)$, for $m = 0, \dots, 2j+1$. Since the expressions
are based on four recurrence relations, they become quite complicated as $j$ gets large. However,
defining these recursions in a computer program is trivial, as we demonstrate
in Section~\ref{sec:computations} % Sec 5.4
by applying the foregoing to a particular problem.
%
%
% ----------------- 30.txt ------------------------------------------------------
%
%





%%%===================================================================
%%%-------------------------------------------------------------------
\part{Application\label{part:two}}%
%%%-------------------------------------------------------------------
%%%
%%% 
\chapter{Convergence to Gibbs measure on the Ising lattice}
%Chapter 5
\label{cha:an-appl-conv}

\section{Physical Motivation}
% Sec 5.1
\label{sec:physical-motivation}
Before considering the details of the Metropolis algorithm, it helps to
understand the setting in which it was developed. To do so we consider a
typical example, the \emph{Ising model}. This is a model of a system consisting
of $n$ ``sites,'' each site taking values in 
%% $\{-1, +1\}$. 
$\{0, 1\}$. 
It might 
help to imagine these sites as equally spaced points on a line or a circle,
though we are not restricted to such cases. 
%% Then our state space $S$ consists of
%% all possible $n$-dimensional permutations of $-1$'s and $+1$'s. 
The state space $S$ consists of
all possible $n$-dimensional permutations of $0$'s and $1$'s. 
%% Thus, $S = \{-1, +1\}^n$ and $|S| = 2^n$. We can think of each
Thus, $S = \{0, 1\}^n$ and $|S| = 2^n$. We can think of each
site as a node existing in one of two possible positions, say ``on'' or
``off.'' Each state $i \in S$ is %represented by 
a unique permutation $\sigma_i$ of of $0$'s and $1$'s. 
Let $\sigma_n(k)$ denote the position %(on or off) 
of the $k$th node when the system is in state $i$. 
An observable $\phi = \phi(\sigma_i)$ on this space is a function of
the permutations $\sigma_i$, $i\in \{1,\dots, 2^n\}$.
%
%
% ---------------- 31.txt ------------------------------------------------------
%
%

The \emph{energy} of the system, when in state $i$, is defined by the Hamiltonian
\[
H(\sigma_i) = - \sum_{\<j k\>} \sigma_i(j)\sigma_i(k)
\]
where the sum runs over all neart neighbor pairs. In statistical mechanics one is often concerned
with the \emph{Gibbs measure} of state $i$, which is defined by
\begin{equation}
\label{eq:5.1}
\pi(\sigma_i)= Z^{-1} \exp\{-\beta H(\sigma_i)\}
\end{equation}
where $Z^{-1}$ is a normalizing constant, sometimes called the \emph{partition function}. 
For the Ising model, we put  
\begin{equation*}
Z = \sum_{i=1}^{2^n}\exp\{-\beta H(\sigma_i)\}
\end{equation*}
so that $\pi$ is a probability measure. 
Now, letting $k$ denote the so called \emph{Boltzman constant}, if a
classical mechanical system is in thermal equilibrium with its surroundings with
absolute temperature $T$, and is in state $i$ with energy $H(\sigma_i)$, then the
probability density in phase-space of the point representing state $i$ is given by
(\ref{eq:5.1}) with $\beta = (kT)^{-1}$. A fundamental result of ergodic theory 
implies that we can also interpret $\pi(\sigma_i)$ as the proportion of time the
system spends in state $i$. If the system is observed at a random time, the
expected value $\bE_\pi\phi$ of any observable $\phi$ is thus 
\begin{equation}
\label{eq:5.2}
\bE_\pi\phi = 
\sum_{i=1}^{2^n}\phi(\sigma_i) \pi(\sigma_i) = 
%% \frac{\sum_{i=1}^{2^n}\phi(\sigma_i) \exp\{-\beta H(\sigma_i)\}}
%% {\sum_{i=1}^{2^n} \exp\{-\beta H(\sigma_i)\}}
Z^{-1}\sum_{i=1}^{2^n}\phi(\sigma_i) \exp\{-\beta H(\sigma_i)\}.
\end{equation}

Perhaps the number of sites $n$ in our Ising model is so large that it is
impractical or impossible to evaluate~(\ref{eq:5.2}).
We might instead consider importance sampling and generate states with
the probability density $\pi$ given in~(\ref{eq:5.1}).
Then $\phi$ is itself an unbiased estimator of~(\ref{eq:5.2}).

If evaluation of~(\ref{eq:5.2}) is difficult, there is no reason to believe that
evaluation of~(\ref{eq:5.1}) is any easier. However, Metropolis and his
collaborators~\cite{Metropolis:1953} contrived a method for producing an ergodic 
%
%
% ------------------ 32.txt --------------------------------------------------
%
%
Markov chain with transition probability matrix K whose elements $k(x,y)$ satisfy
\[
\sum_x \pi(x) k(x,y) = \pi(x)
\]
By 
%Definition 2.1.4 
Definition~\ref{def:2.1.4}
this means that the Markov chain has stationary distribution $\pi$. Since the
chain is ergodic, we know that it will eventually converge to the desired
distribution. So, our main 
problem---the rate at which such a Markov chain will converge---is of primary
concern in applications of the Metropolis algorithm. Moreover, as we will see
below, the Metropolis algorithm ensures that the Markov chain has the desired
stationary distribution by requiring that the chain satisfy the stronger
condition of reversibility, or ``detailed balance.'' Therefore, Markov chains
produced by the Metropolis algorithm satisfy the assumptions of this paper,
and our theory and methods can be used to bound convergence rates of such chains. 

\section{Description of the Metropolis Algorithm}
% Sec 5.2 
\label{sec:descr-metr-algor}
The following explains how the Metropolis Algorithm is carried out. The exposition is
similar to that given in Hammersley and Handscomb's book 
\emph{Monte Carlo Methods} \cite{Hammersley:1964}. We refer the reader desiring
more details to~\cite[Section 9.3]{Hammersley:1964}.

To begin, choose an arbitrary symmetric Markov chain; that is, a Markov chain whose
transition probability matrix P is symmetric. The matrix P is called the 
\emph{proposition matrix}, and its elements
%, $p(i,j)$,
satisfy, for all $i$ and $j$ in $S$,
\begin{equation}
\label{eq:5.4}
p(i,j) \geq 0, \quad p(i,j) = p(j,i), \quad \sum_m p(i,m)=1.
\end{equation}
Using P, we will derive a new transition probability matrix K with elements
$k(i,j)$ satisfying
\begin{equation}
\label{eq:5.5}
\sum_i \pi(i)k(i,j) = \pi(j)
\end{equation}
For $i\neq j$ define
\begin{equation}
\label{eq:5.6}
k(i,j) = 
\begin{cases}
p(i,j)\pi(j)/\pi(i), & \text{ if $\pi(j)/\pi(i) < 1$,}\\
p(i,j), & \text{ if $\pi(j)/\pi(i) \geq 1$.}
\end{cases}
\end{equation}
%
%
% ----------------- 33.txt ----------------------------------------------
%
%
For $i=j$ define
\begin{equation*}
k(i,j) = 
p(i,i) + \sum_{j\in J_i}\bigl(1 - \frac{\pi(j)}{\pi(i)}\bigr)p(i,j), 
\end{equation*}
where $J_i = \{j : j\neq i, \pi(j)/\pi(i) < 1\}$.
Notice that $\pi(i) > 0$ implies $k(i,j) \geq 0$, and
\begin{align}
\label{eq:5.7}
\sum_j k(i,j) 
&= p(i,i) + 
\sum_{j\in J_i}\left\{\bigl(1 - \frac{\pi(j)}{\pi(i)}\bigr)p(i,j)
+ \frac{p(i,j)\pi(j)}{\pi(i)}\right\} 
+ \sum_{j\in J_i^c}p(i,j)
\nonumber\\
&= p(i,i) + 
\sum_{j\in J_i}p(i,j)
+ \sum_{j\in J_i^c}p(i,j).
\end{align}
Equation~(\ref{eq:5.7}) shows that for each $i$ we have $\sum_j k(i,j) = 1$
(the row sums of K are 1), which confirms that K is a stochastic matrix.

Next we check that K satisfies the following \emph{detailed balance} condition:
\begin{equation}
\label{eq:detailed-balance}
\pi(i)k(i,j) = \pi(j)k(j,i).
\end{equation}
Obviously this holds when $i=j$, so assume $i\neq j$.

First suppose $\pi(i) = \pi(j)$.
In this case, the definition~(\ref{eq:5.6}) and symmetry of P imply that
\[
k(i,j) = p(i,j) = p(j,i) = k(j,i).
\]
Therefore, when $\pi(i) = \pi(j)$ the detailed balance
condition~(\ref{eq:detailed-balance}) clearly holds.

Now suppose that $\pi(i) > \pi(j)$. Then the second case in definition (\ref{eq:5.6}) implies
$k(j,i) =p(j,i)$, while the first case gives
\[
k(i,j) = p(i,j)\pi(j)/\pi(i) =  p(j,i)\pi(j)/\pi(i) = k(j,i)\pi(j)/\pi(i),
\]
and we see that 
$\pi(i)k(i,j) = \pi(j)k(j,i)$ holds.
Finally, a symmetric argument shows that if $\pi(j) > \pi(i)$,
then detailed balance is again satisfied.

To complete the justification of the Metropolis algorithm, we note
detailed balance~(\ref{eq:detailed-balance})
implies that $\pi$ is indeed the stationary distribution for a Markov 
chain with transition probability matrix K; that is, (\ref{eq:5.5}) is
satisfied. Indeed, by the detailed balance condition and the fact that
the row sums of K are 1, we have
\begin{equation*}
\sum_i \pi(i)k(i,j)
=\sum_i \pi(j)k(j,i)
=\pi(j)\sum_i k(j,i)
 = \pi(j).
\end{equation*}


We now summarize the tasks performed when implementing the Metropolis algorithm.
First, pick as a proposition matrix any symmetric stochastic matrix and denote
it by P. Begin 
%
%
% ----------- 34.txt --------------------------------------------------------
%
the simulation of the Markov chain in state $i$, and then take one step of the
proposition Markov chain to arrive in state $j$. That is, choose state $j$
according to the probability mass function $p(i, \cdot)$. 
Next, compute $\pi(j)/\pi(i)$. 
If $\pi(j)/\pi(i) \geq 1$, accept $j$ as the new state. If $\pi(j)/\pi(i) < 1$,
then with probability $\pi(j)/\pi(i)$ accept $j$ as the new state; otherwise
(with probability $1- \pi(j)/\pi(i)$), take $i$ as the new state. Performing
each of these tasks produces one step of the Markov chain represented by the 
transition probability matrix K.


\section{The Ising Model}
% Sec. 5.3 
\label{sec:ising-model}
Returning to the Ising model considered in
Section~\ref{sec:physical-motivation}, we consider how the Metropolis 
algorithm is applied to such a model. Again, start the simulated chain in state
$X_0 = i$ and let $X_1 = j$ be a state chosen according to 
$p(i, \cdot)$, where $p(i,j) = p(j,i)$. Next, compute 
\[
\pi(\sigma_j)/\pi(\sigma_i) = \exp\{-\beta[H(\sigma_j) - H(\sigma_i)]\}
\]
Suppose the energy $H(\sigma_j)$ of the new state satisfies 
$H(\sigma_j) \leq H(\sigma_i)$.
Then $\pi(\sigma_j)/\pi(\sigma_i)$ is at least 1 and we move to state $j$.
On the other hand, if
$H(\sigma_j) > H(\sigma_i)$,
then $\pi(\sigma_j)/\pi(\sigma_i)< 1$ and we take
\begin{equation}
\label{eq:5.8}
X_1 = 
\begin{cases}
j & \text{ with probability $e^{-\beta\Delta H}$},\\
i & \text{ with probability $1-e^{-\beta\Delta H}$}.
\end{cases}
\end{equation}
where $\Delta H = H(\sigma_j) - H(\sigma_i)$ is the change in energy from state $j$ to state $i$.

There are many ways to choose a proposition matrix P. The simplest derives from the
so called \emph{Glauber dynamics}, which proceeds as follows: choose one of the
$n$ sites at random with probability $1/n$---let $s$ denote the chosen
site---and change the position of this site; i.e., subtract its value from 1.
%multiply its value by $-1$. 
The resulting state $j$ is the 
%permutation of 
%-1’s and +1’s
% 0's and 1's 
binary string
given by
\begin{equation*}
\sigma_j(k) = 
\begin{cases}
1-\sigma_i(k) & \text{ if $k =s$},\\
\sigma_i(k) & \text{ if $k =s$},
\end{cases}\quad (k = 1, \dots, d).
\end{equation*}
%
%
% -------------- 35.txt ---------------------------------------------------------
%
%
The procedure we employ below involves a slight variation: choose site $s$ at
random uniformly from the $n$ possible sites. However, instead of changing the
value of the chosen site with absolute certainty, as in Glauber dynamics, we
``flip the switch'' with probability $1/2$. 
Therefore, when $k \neq s$
we have $\sigma_j(k) = \sigma_i(k)$, and when $k = s$,
\begin{equation*}
\sigma_j(s) = 
\begin{cases}
1-\sigma_i(s) & \text{ with probability 1/2},\\
\sigma_i(s) & \text{ with probability 1/2}.
\end{cases}
\end{equation*}
The advantage of this procedure is that the proposition matrix, before altering
it with the Metropolis algorithm, is a bit more interesting than that produced
by traditional Glauber dynamics. For, it has 1/2 along the main diagonal and is
thus aperiodic and converges to a stationary distribution. Glauber dynamics has
period 2, so its proposition matrix does not satisfy the hypotheses of our
convergence theorem (Theorem~\ref{thm-2.1.2}). Keep in mind, however, that any
proposition matrix modified by the Metropolis algorithm satisfies our hypotheses.
\section{Computations}
% Sec 5.4
\label{sec:computations}
\subsection{Computer Programs}
Writing a computer program to simulate realizations of the Markov chain described
above (i.e. with transition matrix K) is straight forward, and we do so using
Matlab and the program {\tt ising.m} which appears in the Appendix.

When performing simulations in the present context, there are a few important
aspects to consider. First, our program begins with a ``hot start,'' which means
that the initial state is picked by assigning each site to the value 0 or 1, each
with probability 1/2.\footnote{We cannot simply draw our initial state
from the stationary distribution since the point of this work
is to deal with cases for which samples from the stationary distribution are not
immediately attainable. Furthermore, we are trying to determine how long it will
take for observations from the simulated chain to represent samples from the
stationary distribution.}
%% figure out how long it will take for observations from the simulated chain to represent samples from
%% the stationary distribution. 
%% Perhaps we should instead have picked our initial state
%% from the stationary distribution. On the contrary, this would defeat the
%% whole purpose of our analysis.  
%% Recall that the method has been developed to deal with
%% cases for which 
%
%
% ----------------- 36.txt --------------------------------------------
%
%
%% sampl from the stationary distribution are not quickly attainable. Furthermore, we are trying to
%% figure out how long it will take for observations from the simulated chain to represent samples from
%% the stationary distribution. 

%% sampl from the stationary distribution are not quickly attainable. Furthermore, we are trying to
%% figure out how long it will take for observations from the simulated chain to represent samples from
%% the stationary distribution. 
%% Second, t

The method we have developed depends solely on covariances
of our observable, which we estimate from the data produced by the
simulations. However, such estimates require that the data come from the
stationary distribution. Therefore, we must discard a number of observations
before using the data to estimate covariances. The number of observations 
to discard depends on how long must we wait 
before we can expect the data to represent samples from the stationary distribution.
%% until we have data from the stationary distribution,
%% and we are back where we started — almost.
When employing our procedure, we can discard a
conservative (very large) number of observations and perform our analysis, deriving a bound on
the convergence time. Then, in all future studies, we will know how much data should be discarded
before samples can be assumed to have come from the stationary distribution.

In considering what observable to use on the state space described above, perhaps the
most obvious is simply the number of nodes in the ``on'' position (i.e., the
number of 1's). Recalling that $\sigma_j(k)\in \{0,1\}$ denotes the position of
the $k$th node when the system is in state $j$, this observable is simply the
sum $\phi(j) = \sum_k\sigma_j(k)$. 
%% where X is the indicator function. 
The program {\tt ising.m} simulates the Markov chain, computes the
values of this observable (and its square) and writes these data to a file
called {\tt phi.dat}. 
The program then computes the first few Lanczos coefficients using covariances
produced by Matlab, so that we can check them against the results we get from the
{\tt lanczos.c} program, which we now describe.

The lanczos.c program (listed, along with its dependencies, in the Appendix) implements
the recursive relations described in Section~\ref{sec:lancz-proc-mark} % Sec 4.2
to compute the Lanczos coefficients by the new
method. It accepts input from the user specifying any observable, any number of
iterations, and any 
%
%
% ---------------  37.txt  -----------------------------------------------
%
%
number of desired Lauczos coefiicients. That is, the observable could have come from a simulation
of any reversible Markov chain and not just our special Metropolis chain. The output of this
program is a matrix T containing the maximum number of Lanczos coefficients (up
to the number requested by the user) before an approximate invariant subspace
was reached. The eigenvalues of the nonzero part of this matrix, in accordance
with the theory developed above, should provide a close approximation to the
eigenvalues of the Markov chain's stochastic matrix.

\subsection{Results}
% Sec. 5.4.2 
\begin{table}
\centering
\label{tab:1}
\caption{Estimates of $\lambda_{\max}(\K) = 0.998746$ using $\phi_1$.}
\vskip2mm
\begin{tabular}{r|c|c|c|c}
simulation \# & $\alpha_1$ & $\beta_1$ & $\lambda_{\max}(\T_2)$& $\beta_2$\\
\hline
1& 0.995107& 0.006960& 0.995220& 0.600243\\
2& 0.995099& 0.005891& 0.994863& 0.477393\\
3& 0.994862& 0.009671& 0.994925& N/A\\
4& 0.995321& 0.009319& 0.995654& N/A\\
5& 0.994787& 0.004872& 0.405446& 0.994742
\end{tabular}
\end{table}
Table~\ref{tab:1} shows the results of 5 simulations each of 100,000 iterations (discarding the first
5,000 observations) for the one dimensional Ising lattice with 10 nodes and $\beta= 1$. Displayed is
the estimate of the first Lanczos coefficient $\alpha_1$ (which is the
eigenvalue estimate after one step; i.e., $\lambda_{\max}(\T_1)$), along with its
error bound $\beta_1$. Appearing in the third column is the largest eigenvalue,
$\lambda_0(T_2)$, of the $2 \times 2$ matrix $\T_2$, and $\beta_2$ appears in the
fourth column when it is was computable. When it was not computable, it was
%% probably 
because an approximate invariant subspace was reached 
at the first step, so $\beta_2$ involved expressions which were close to machine
epsilon and thus could not be computed accurately. Table~\ref{tab:2}  shows the same
information when using a second observable, 
$\phi_2 = \phi_1^2$ (i.e., the number of 1's squared).

\begin{table}
\caption{Estimates of $\lambda_{\max}(K) = 0.998746$ using $\phi_2$}
\label{tab:2}
\begin{tabular}{r|c|c|c|c}
simulation \# & $\alpha_1$ & $\beta$ & $\lambda_{\max}(\T_2)$& $\beta_2$\\
\hline
1& 0.994262& 0.006142& 0.994222& N/A\\
2& 0.994333& 0.002974& 0.994332& N/A\\
3& 0.994307& 0.008892& 0.994374& 0.726320\\
4& 0.994436& 0.012884& 0.994685& 0.361404\\
5& 0.993789& 0.006909& 0.993820& 0.788710
\end{tabular}
\end{table}
The true value of the subdominant eigenvalue for this problem is $\lambda_{\max}(\K) = 0.998746$,
and the next largest is $\lambda_2(\K) = 0.993303$. These were computed directly
from the transition matrix 
%
%
% ---------- 38.txt ----------------------------------------------
%
%
using the Matlab program {\tt tpm.m} listed in the Appendix. They should be very accurate, and it
takes the Matlab routine {\tt eig()} just over 20 minutes of cpu time on a Sun
Ultra Sparc to find all the eigenvalues of K.

As montioned in Section~\ref{sec:gener-lancz-algor} (see
Theorem~\ref{thm:4.1.1}), the $\beta$ coefficients provide conservative error 
bounds on the eigenvalue estimates from the previous Lanczos
step. Unfortunately, for this example $\beta$ appears to be too conservative as
it does not even allow us to bound our estimates away from 1. Furthermore, it  
seems that the estimates generated by the method above fall systematically below
the true value of the subdominant eigenvalue. In every simulation, an
approximate invariant subspace (to machine tolerance) was reached after
computing at most 2 pairs of Lanczos coefficients. For $\phi_1$ we see that 
it was reached after computing only one pair on simulations 3 and 4, similarly
for $\phi_2$ on simulations 1 and 2. These observations indicate that the chosen
observable is close to the slowest mode (or 
eigenfunction) of the process. That is, we should observe fast convergence to an approximate
invariant space. However, the space found might contain only the second slowest mode of the
process and, in that case, our eigenvalue estimates would be closer to the third largest eigenvalue,
instead of the second largest.\footnote{Recall from above that we won't
converge to the eigenspace corresponding to the largest eigenvalue 
(which is always 1), unless we choose a nearly constant observable function.}

The matrix T was produced by the program {\tt lanczos.c} and its eigenvalues were computed
by Matlab, both operations taking a few seconds. Granted, comparing a few seconds to the 20
minutes it takes Matlab to compute all the eigenvales is not really fair, 
since calling Matlab's {\tt eig()} routine is not always the
%
%
% --------- 39.txt -----------------------------------------------------
%
%
optimal traditional method for computing only a few eigenvalues of K. However, recall that
the primary motivation for the new method are those examples where the matrix K is so large
that we can't store even a single column vector in fast memory. Once we start
swapping data to and from slow memory, traditional algorithms can become
intractable, and such examples are easy to come by. For instance, if the Ising
model described has 1000 nodes, it produces a $2^{1000}\times 2^{1000}$
transition probability matrix.
%
%
% 40.txt
%
%



 % Chapter 5
%%%
\chapter{Convergence to Gibbs measure on the Ising lattice}
%Chapter 5
\label{cha:an-appl-conv}

\section{Physical Motivation}
% Sec 5.1
\label{sec:physical-motivation}
Before considering the details of the Metropolis algorithm, it helps to
understand the setting in which it was developed. To do so we consider a
typical example, the \emph{Ising model}. This is a model of a system consisting
of $n$ ``sites,'' each site taking values in 
%% $\{-1, +1\}$. 
$\{0, 1\}$. 
It might 
help to imagine these sites as equally spaced points on a line or a circle,
though we are not restricted to such cases. 
%% Then our state space $S$ consists of
%% all possible $n$-dimensional permutations of $-1$'s and $+1$'s. 
The state space $S$ consists of
all possible $n$-dimensional permutations of $0$'s and $1$'s. 
%% Thus, $S = \{-1, +1\}^n$ and $|S| = 2^n$. We can think of each
Thus, $S = \{0, 1\}^n$ and $|S| = 2^n$. We can think of each
site as a node existing in one of two possible positions, say ``on'' or
``off.'' Each state $i \in S$ is %represented by 
a unique permutation $\sigma_i$ of of $0$'s and $1$'s. 
Let $\sigma_n(k)$ denote the position %(on or off) 
of the $k$th node when the system is in state $i$. 
An observable $\phi = \phi(\sigma_i)$ on this space is a function of
the permutations $\sigma_i$, $i\in \{1,\dots, 2^n\}$.
%
%
% ---------------- 31.txt ------------------------------------------------------
%
%

The \emph{energy} of the system, when in state $i$, is defined by the Hamiltonian
\[
H(\sigma_i) = - \sum_{\<j k\>} \sigma_i(j)\sigma_i(k)
\]
where the sum runs over all nearest neighbor pairs. In statistical mechanics one is often concerned
with the \emph{Gibbs measure} of state $i$, which is defined by
\begin{equation}
\label{eq:5.1}
\pi(\sigma_i)= Z^{-1} \exp\{-\beta H(\sigma_i)\}
\end{equation}
where $Z^{-1}$ is a normalizing constant, sometimes called the \emph{partition function}. 
For the Ising model, we put  
\begin{equation*}
Z = \sum_{i=1}^{2^n}\exp\{-\beta H(\sigma_i)\}
\end{equation*}
so that $\pi$ is a probability measure. 
Now, letting $k$ denote the so called \emph{Boltzmann constant}, if a
classical mechanical system is in thermal equilibrium with its surroundings with
absolute temperature $T$, and is in state $i$ with energy $H(\sigma_i)$, then the
probability density in phase-space of the point representing state $i$ is given by
(\ref{eq:5.1}) with $\beta = (kT)^{-1}$. A fundamental result of ergodic theory 
implies that we can also interpret $\pi(\sigma_i)$ as the proportion of time the
system spends in state $i$. If the system is observed at a random time, the
expected value $\bE_\pi\phi$ of any observable $\phi$ is thus 
\begin{equation}
\label{eq:5.2}
\bE_\pi\phi = 
\sum_{i=1}^{2^n}\phi(\sigma_i) \pi(\sigma_i) = 
%% \frac{\sum_{i=1}^{2^n}\phi(\sigma_i) \exp\{-\beta H(\sigma_i)\}}
%% {\sum_{i=1}^{2^n} \exp\{-\beta H(\sigma_i)\}}
Z^{-1}\sum_{i=1}^{2^n}\phi(\sigma_i) \exp\{-\beta H(\sigma_i)\}.
\end{equation}

Perhaps the number of sites $n$ in our Ising model is so large that it is
impractical or impossible to evaluate~(\ref{eq:5.2}).
We might instead consider importance sampling and generate states with
the probability density $\pi$ given in~(\ref{eq:5.1}).
Then $\phi$ is itself an unbiased estimator of~(\ref{eq:5.2}).

If evaluation of~(\ref{eq:5.2}) is difficult, there is no reason to believe that
evaluation of~(\ref{eq:5.1}) is any easier. However, Metropolis and his
collaborators~\cite{Metropolis:1953} contrived a method for producing an ergodic 
%
%
% ------------------ 32.txt --------------------------------------------------
%
%
Markov chain with transition probability matrix K whose elements $k(x,y)$ satisfy
\[
\sum_x \pi(x) k(x,y) = \pi(x)
\]
By 
%Definition 2.1.4 
Definition~\ref{def:2.1.4}
this means that the Markov chain has stationary distribution $\pi$. Since the
chain is ergodic, we know that it will eventually converge to the desired
distribution. So, our main 
problem---the rate at which such a Markov chain will converge---is of primary
concern in applications of the Metropolis algorithm. Moreover, as we will see
below, the Metropolis algorithm ensures that the Markov chain has the desired
stationary distribution by requiring that the chain satisfy the stronger
condition of reversibility, or ``detailed balance.'' Therefore, Markov chains
produced by the Metropolis algorithm satisfy the assumptions of this paper,
and our theory and methods can be used to bound convergence rates of such chains. 

\section{Description of the Metropolis Algorithm}
% Sec 5.2 
\label{sec:descr-metr-algor}
The following explains how the Metropolis Algorithm is carried out. The exposition is
similar to that given in Hammersley and Handscomb's book 
\emph{Monte Carlo Methods} \cite{Hammersley:1964}. We refer the reader desiring
more details to~\cite[Section 9.3]{Hammersley:1964}.

To begin, choose an arbitrary symmetric Markov chain; that is, a Markov chain whose
transition probability matrix P is symmetric. The matrix P is called the 
\emph{proposition matrix}, and its elements
%, $p(i,j)$,
satisfy, for all $i$ and $j$ in $S$,
\begin{equation}
\label{eq:5.4}
p(i,j) \geq 0, \quad p(i,j) = p(j,i), \quad \sum_m p(i,m)=1.
\end{equation}
Using P, we will derive a new transition probability matrix K with elements
$k(i,j)$ satisfying
\begin{equation}
\label{eq:5.5}
\sum_i \pi(i)k(i,j) = \pi(j)
\end{equation}
For $i\neq j$ define
\begin{equation}
\label{eq:5.6}
k(i,j) = 
\begin{cases}
p(i,j)\pi(j)/\pi(i), & \text{ if $\pi(j)/\pi(i) < 1$,}\\
p(i,j), & \text{ if $\pi(j)/\pi(i) \geq 1$.}
\end{cases}
\end{equation}
%
%
% ----------------- 33.txt ----------------------------------------------
%
%
For $i=j$ define
\begin{equation*}
k(i,j) = 
p(i,i) + \sum_{j\in J_i}\bigl(1 - \frac{\pi(j)}{\pi(i)}\bigr)p(i,j), 
\end{equation*}
where $J_i = \{j : j\neq i, \pi(j)/\pi(i) < 1\}$.
Notice that $\pi(i) > 0$ implies $k(i,j) \geq 0$, and
\begin{align}
\label{eq:5.7}
\sum_j k(i,j) 
&= p(i,i) + 
\sum_{j\in J_i}\left\{\bigl(1 - \frac{\pi(j)}{\pi(i)}\bigr)p(i,j)
+ \frac{p(i,j)\pi(j)}{\pi(i)}\right\} 
+ \sum_{j\in J_i^c}p(i,j)
\nonumber\\
&= p(i,i) + 
\sum_{j\in J_i}p(i,j)
+ \sum_{j\in J_i^c}p(i,j).
\end{align}
Equation~(\ref{eq:5.7}) shows that for each $i$ we have $\sum_j k(i,j) = 1$
(the row sums of K are 1), which confirms that K is a stochastic matrix.

Next we check that K satisfies the following \emph{detailed balance} condition:
\begin{equation}
\label{eq:detailed-balance}
\pi(i)k(i,j) = \pi(j)k(j,i).
\end{equation}
Obviously this holds when $i=j$, so assume $i\neq j$.

First suppose $\pi(i) = \pi(j)$.
In this case, the definition~(\ref{eq:5.6}) and symmetry of P imply that
\[
k(i,j) = p(i,j) = p(j,i) = k(j,i).
\]
Therefore, when $\pi(i) = \pi(j)$ the detailed balance
condition~(\ref{eq:detailed-balance}) clearly holds.

Now suppose that $\pi(i) > \pi(j)$. Then the second case in definition (\ref{eq:5.6}) implies
$k(j,i) =p(j,i)$, while the first case gives
\[
k(i,j) = p(i,j)\pi(j)/\pi(i) =  p(j,i)\pi(j)/\pi(i) = k(j,i)\pi(j)/\pi(i),
\]
and we see that 
$\pi(i)k(i,j) = \pi(j)k(j,i)$ holds.
Finally, a symmetric argument shows that if $\pi(j) > \pi(i)$,
then detailed balance is again satisfied.

To complete the justification of the Metropolis algorithm, we note
detailed balance~(\ref{eq:detailed-balance})
implies that $\pi$ is indeed the stationary distribution for a Markov 
chain with transition probability matrix K; that is, (\ref{eq:5.5}) is
satisfied. Indeed, by the detailed balance condition and the fact that
the row sums of K are 1, we have
\begin{equation*}
\sum_i \pi(i)k(i,j)
=\sum_i \pi(j)k(j,i)
=\pi(j)\sum_i k(j,i)
 = \pi(j).
\end{equation*}


We now summarize the tasks performed when implementing the Metropolis algorithm.
First, pick as a proposition matrix any symmetric stochastic matrix and denote
it by P. Begin 
%
%
% ----------- 34.txt --------------------------------------------------------
%
the simulation of the Markov chain in state $i$, and then take one step of the
proposition Markov chain to arrive in state $j$. That is, choose state $j$
according to the probability mass function $p(i, \cdot)$. 
Next, compute $\pi(j)/\pi(i)$. 
If $\pi(j)/\pi(i) \geq 1$, accept $j$ as the new state. If $\pi(j)/\pi(i) < 1$,
then with probability $\pi(j)/\pi(i)$ accept $j$ as the new state; otherwise
(with probability $1- \pi(j)/\pi(i)$), take $i$ as the new state. Performing
each of these tasks produces one step of the Markov chain represented by the 
transition probability matrix K.


\section{The Ising Model}
% Sec. 5.3 
\label{sec:ising-model}
Returning to the Ising model considered in
Section~\ref{sec:physical-motivation}, we consider how the Metropolis 
algorithm is applied to such a model. Again, start the simulated chain in state
$X_0 = i$ and let $X_1 = j$ be a state chosen according to 
$p(i, \cdot)$, where $p(i,j) = p(j,i)$. Next, compute 
\[
\pi(\sigma_j)/\pi(\sigma_i) = \exp\{-\beta[H(\sigma_j) - H(\sigma_i)]\}
\]
Suppose the energy $H(\sigma_j)$ of the new state satisfies 
$H(\sigma_j) \leq H(\sigma_i)$.
Then $\pi(\sigma_j)/\pi(\sigma_i)$ is at least 1 and we move to state $j$.
On the other hand, if
$H(\sigma_j) > H(\sigma_i)$,
then $\pi(\sigma_j)/\pi(\sigma_i)< 1$ and we take
\begin{equation}
\label{eq:5.8}
X_1 = 
\begin{cases}
j & \text{ with probability $e^{-\beta\Delta H}$},\\
i & \text{ with probability $1-e^{-\beta\Delta H}$}.
\end{cases}
\end{equation}
where $\Delta H = H(\sigma_j) - H(\sigma_i)$ is the change in energy from state $j$ to state $i$.

There are many ways to choose a proposition matrix P. The simplest derives from the
so called \emph{Glauber dynamics}, which proceeds as follows: choose one of the
$n$ sites at random with probability $1/n$---let $s$ denote the chosen
site---and change the position of this site; i.e., subtract its value from 1.
%multiply its value by $-1$. 
The resulting state $j$ is the 
%permutation of 
%-1’s and +1’s
% 0's and 1's 
binary string
given by
\begin{equation*}
\sigma_j(k) = 
\begin{cases}
1-\sigma_i(k) & \text{ if $k =s$},\\
\sigma_i(k) & \text{ if $k =s$},
\end{cases}\quad (k = 1, \dots, d).
\end{equation*}
%
%
% -------------- 35.txt ---------------------------------------------------------
%
%
The procedure we employ below involves a slight variation: choose site $s$ at
random uniformly from the $n$ possible sites. However, instead of changing the
value of the chosen site with absolute certainty, as in Glauber dynamics, we
``flip the switch'' with probability $1/2$. 
Therefore, when $k \neq s$
we have $\sigma_j(k) = \sigma_i(k)$, and when $k = s$,
\begin{equation*}
\sigma_j(s) = 
\begin{cases}
1-\sigma_i(s) & \text{ with probability 1/2},\\
\sigma_i(s) & \text{ with probability 1/2}.
\end{cases}
\end{equation*}
The advantage of this procedure is that the proposition matrix, before altering
it with the Metropolis algorithm, is a bit more interesting than that produced
by traditional Glauber dynamics. For, it has 1/2 along the main diagonal and is
thus aperiodic and converges to a stationary distribution. Glauber dynamics has
period 2, so its proposition matrix does not satisfy the hypotheses of our
convergence theorem (Theorem~\ref{thm-2.1.2}). Keep in mind, however, that any
proposition matrix modified by the Metropolis algorithm satisfies our hypotheses.
\section{Computations}
% Sec 5.4
\label{sec:computations}
\subsection{Computer Programs}
Writing a computer program to simulate realizations of the Markov chain described
above (i.e. with transition matrix K) is straight forward, and we do so using
Matlab and the program {\tt ising.m} which appears in the Appendix.

When performing simulations in the present context, there are a few important
aspects to consider. First, our program begins with a ``hot start,'' which means
that the initial state is picked by assigning each site to the value 0 or 1, each
with probability 1/2.\footnote{We cannot simply draw our initial state
from the stationary distribution since the point of this work
is to deal with cases for which samples from the stationary distribution are not
immediately attainable. Furthermore, we are trying to determine how long it will
take for observations from the simulated chain to represent samples from the
stationary distribution.}
%% figure out how long it will take for observations from the simulated chain to represent samples from
%% the stationary distribution. 
%% Perhaps we should instead have picked our initial state
%% from the stationary distribution. On the contrary, this would defeat the
%% whole purpose of our analysis.  
%% Recall that the method has been developed to deal with
%% cases for which 
%
%
% ----------------- 36.txt --------------------------------------------
%
%
%% sampl from the stationary distribution are not quickly attainable. Furthermore, we are trying to
%% figure out how long it will take for observations from the simulated chain to represent samples from
%% the stationary distribution. 

%% sampl from the stationary distribution are not quickly attainable. Furthermore, we are trying to
%% figure out how long it will take for observations from the simulated chain to represent samples from
%% the stationary distribution. 
%% Second, t

The method we have developed depends solely on covariances
of our observable, which we estimate from the data produced by the
simulations. However, such estimates require that the data come from the
stationary distribution. Therefore, we must discard a number of observations
before using the data to estimate covariances. The number of observations 
to discard depends on how long must we wait 
before we can expect the data to represent samples from the stationary distribution.
%% until we have data from the stationary distribution,
%% and we are back where we started - almost.
When employing our procedure, we can discard a
conservative (very large) number of observations and perform our analysis, deriving a bound on
the convergence time. Then, in all future studies, we will know how much data should be discarded
before samples can be assumed to have come from the stationary distribution.

In considering what observable to use on the state space described above, perhaps the
most obvious is simply the number of nodes in the ``on'' position (i.e., the
number of 1's). Recalling that $\sigma_j(k)\in \{0,1\}$ denotes the position of
the $k$th node when the system is in state $j$, this observable is simply the
sum $\phi(j) = \sum_k\sigma_j(k)$. 
%% where X is the indicator function. 
The program {\tt ising.m} simulates the Markov chain, computes the
values of this observable (and its square) and writes these data to a file
called {\tt phi.dat}. 
The program then computes the first few Lanczos coefficients using covariances
produced by Matlab, so that we can check them against the results we get from the
{\tt lanczos.c} program, which we now describe.

The lanczos.c program (listed, along with its dependencies, in the Appendix) implements
the recursive relations described in Section~\ref{sec:lancz-proc-mark} % Sec 4.2
to compute the Lanczos coefficients by the new
method. It accepts input from the user specifying any observable, any number of
iterations, and any 
%
%
% ---------------  37.txt  -----------------------------------------------
%
%
number of desired Lanczos coefficients. That is, the observable could have come from a simulation
of any reversible Markov chain and not just our special Metropolis chain. The output of this
program is a matrix T containing the maximum number of Lanczos coefficients (up
to the number requested by the user) before an approximate invariant subspace
was reached. The eigenvalues of the nonzero part of this matrix, in accordance
with the theory developed above, should provide a close approximation to the
eigenvalues of the Markov chain's stochastic matrix.

\subsection{Results}
% Sec. 5.4.2 
\begin{table}
\centering
\label{tab:1}
\caption{Estimates of $\lambda_{\max}(\K) = 0.998746$ using $\phi_1$.}
\vskip4mm
\begin{tabular}{r|c|c|c|c}
simulation \# & $\alpha_1$ & $\beta_1$ & $\lambda_{\max}(\T_2)$& $\beta_2$\\
\hline
1& 0.995107& 0.006960& 0.995220& 0.600243\\
2& 0.995099& 0.005891& 0.994863& 0.477393\\
3& 0.994862& 0.009671& 0.994925& N/A\\
4& 0.995321& 0.009319& 0.995654& N/A\\
5& 0.994787& 0.004872& 0.405446& 0.994742
\end{tabular}
\end{table}
Table~\ref{tab:1} shows the results of 5 simulations each of 100,000 iterations (discarding the first
5,000 observations) for the one dimensional Ising lattice with 10 nodes and $\beta= 1$. Displayed is
the estimate of the first Lanczos coefficient $\alpha_1$ (which is the
eigenvalue estimate after one step; i.e., $\lambda_{\max}(\T_1)$), along with its
error bound $\beta_1$. Appearing in the third column is the largest eigenvalue,
$\lambda_0(T_2)$, of the $2 \times 2$ matrix $\T_2$, and $\beta_2$ appears in the
fourth column when it is was computable. When it was not computable, it was
%% probably 
because an approximate invariant subspace was reached 
at the first step, so $\beta_2$ involved expressions which were close to machine
epsilon and thus could not be computed accurately. Table~\ref{tab:2}  shows the same
information when using a second observable, 
$\phi_2 = \phi_1^2$ (i.e., the number of 1's squared).

\begin{table}
\centering
\label{tab:2}
\caption{Estimates of $\lambda_{\max}(K) = 0.998746$ using $\phi_2$}
\vskip4mm
\begin{tabular}{r|c|c|c|c}
simulation \# & $\alpha_1$ & $\beta$ & $\lambda_{\max}(\T_2)$& $\beta_2$\\
\hline
1& 0.994262& 0.006142& 0.994222& N/A\\
2& 0.994333& 0.002974& 0.994332& N/A\\
3& 0.994307& 0.008892& 0.994374& 0.726320\\
4& 0.994436& 0.012884& 0.994685& 0.361404\\
5& 0.993789& 0.006909& 0.993820& 0.788710
\end{tabular}
\end{table}
The true value of the subdominant eigenvalue for this problem is $\lambda_{\max}(\K) = 0.998746$,
and the next largest is $\lambda_2(\K) = 0.993303$. These were computed directly
from the transition matrix 
%
%
% ---------- 38.txt ----------------------------------------------
%
%
using the Matlab program {\tt tpm.m} listed in the Appendix. They should be very accurate, and it
takes the Matlab routine {\tt eig()} just over 20 minutes of cpu time on a Sun
Ultra Sparc to find all the eigenvalues of K.

As mentioned in Section~\ref{sec:gener-lancz-algor} (see
Theorem~\ref{thm:4.1.1}), the $\beta$ coefficients provide conservative error 
bounds on the eigenvalue estimates from the previous Lanczos
step. Unfortunately, for this example $\beta$ appears to be too conservative as
it does not even allow us to bound our estimates away from 1. Furthermore, it  
seems that the estimates generated by the method above fall systematically below
the true value of the subdominant eigenvalue. In every simulation, an
approximate invariant subspace (to machine tolerance) was reached after
computing at most 2 pairs of Lanczos coefficients. For $\phi_1$ we see that 
it was reached after computing only one pair on simulations 3 and 4, similarly
for $\phi_2$ on simulations 1 and 2. These observations indicate that the chosen
observable is close to the slowest mode (or 
eigenfunction) of the process. That is, we should observe fast convergence to an approximate
invariant space. However, the space found might contain only the second slowest mode of the
process and, in that case, our eigenvalue estimates would be closer to the third largest eigenvalue,
instead of the second largest.\footnote{Recall from above that we won't
converge to the eigenspace corresponding to the largest eigenvalue 
(which is always 1), unless we choose a nearly constant observable function.}

The matrix T was produced by the program {\tt lanczos.c} and its eigenvalues were computed
by Matlab, both operations taking a few seconds. Granted, comparing a few seconds to the 20
minutes it takes Matlab to compute all the eigenvalues is not really fair, 
since calling Matlab's {\tt eig()} routine is not always the
%
%
% --------- 39.txt -----------------------------------------------------
%
%
optimal traditional method for computing only a few eigenvalues of K. However, recall that
the primary motivation for the new method are those examples where the matrix K is so large
that we can't store even a single column vector in fast memory. Once we start
swapping data to and from slow memory, traditional algorithms can become
intractable, and such examples are easy to come by. For instance, if the Ising
model described has 1000 nodes, it produces a $2^{1000}\times 2^{1000}$
transition probability matrix.
%
%
% 40.txt
%
%


%%%-------------------------------------------------------------------
%%%
%%% Chapter 6

Conclusion

We conclude by noting a few aspects of the special cases to which the foregoing applies.
We are interested in the subdominant eigenvalue of a large transition probability matrix. The

foregoing can be used to approximate these eigenvalues when the following conditions are satisfied:
o The Markov chain is reversible (detailed balance condition).

0 We can easily simulate realizations of the chain as well as functions (observables) of these

realizations.

When these conditions hold, we can side step the traditional Lanczos algorithm, and get at the
values of the Lanczos coeflicients via the variational properties of observables on the state space.
There are many ideas in this paper that should be expanded upon in future studies. Per-
haps most obvious is the desire to find analogous techniques for non-reversible (i.e. non-symmetric)
Markov chains. This is a considerable problem since the Lanczos methods described above are no
longer applicable. Perhaps the Amoldi algorithm ([1], algorithm 6.9), or the non-symmetric Lanczos

algorithm can also be usefully specialized for stochastic matrices.

% 41.txt
41

When applying the Lanczos algorithm, we usually choose a single starting vector (in the
present context, an observable), perform the algorithm for a number of steps, and gather results.
Then we do the same with a new starting vector and compare the results to those obtained with
the first vector. Proceeding in this way for a number of starting vectors, we can provide statistical
evidence that we have, indeed, located the eigenvalues of interest. However, when there are two
observables 451, :15; of particular interest, perhaps we would benefit by considering both observables
simultaneously. Future studies might develop an algorithm which incorporates both observables,

again via ¢.- = H1/2(¢.' - E,.¢.-). In this case, a Lanczos algorithm would be based on the space

Ic*(M,¢1,¢»2.p) 2 ran{«/»mz»2,M«/»1,M«/:2, . . .,MP—‘¢1.M"—‘«/22}

The Lanczos coefiicients would then involve not only the autocovariances C, (¢,<(n),¢,-(n + m)),

but also the cross-covariances C, (¢.~(n), ¢j(n + m)) (defined in (2.6)).

%%%
\chapter{Conclusion}
% Chapter 6
\label{cha:conclusion}
We conclude with some remarks about the special situations in which our
method is useful.
%few aspects of the cases to which the foregoing applies.
We are interested in the subdominant eigenvalue of a large transition probability matrix. Our
method can be used to approximate these eigenvalues when the following conditions are satisfied:
\begin{itemize}
\item 
The Markov chain is reversible (satisfies detailed balance).
\item
We can simulate realizations of the chain as well as functions of these
realizations (observables on the state space).
\end{itemize}
When these conditions hold, we can avoid the traditional Lanczos algorithm, and
estimate the values of the Lanczos coefficients via the variational properties of
observables on the state space. 
There are a number of ideas in this paper that could be expanded upon in future
studies.  Perhaps most obvious would be to try to find analogous techniques for
nonreversible (i.e. nonsymmetric) Markov chains. This is a considerable problem
since the Lanczos methods described above are no longer applicable. Perhaps the
\emph{Arnoldi algorithm} (see \cite[Algorithm 6.9]{Demmel:1997}), or the
\emph{nonsymmetric Lanczos algorithm} could also be modified to exploit
special properties of stochastic matrices. %%  (as we have done with the Lanczos
%% algorithm in the symmetric case). 
%
%
% 41.txt
%
%

When applying the Lanczos algorithm, we usually choose a single starting vector (in the
present context, an observable), perform the algorithm for a number of steps, and gather results.
Then we do the same with a new starting vector and compare the results to those obtained with
the first vector. Proceeding in this way for a number of starting vectors, we can provide statistical
evidence that we have, indeed, located the eigenvalues of interest. However, when there are two
observables, $\phi_1$ and $\phi_2$, of particular interest, perhaps we would
benefit from considering both observables simultaneously. Future studies might
develop an algorithm which incorporates both observables, 
again via $\psi_i = \Pi^{1/2}(\phi_i - \bE_{\pi}\phi_i)$. %% In this case, a
The Lanczos algorithm would then be based on the space 
\[
\sK^*(M,\psi_1, \psi_2, p) =  \ran\{\psi_1, \psi_2, \M\psi_1, 
\M\psi_2, \dots, \M^{p-1}\psi_1, \M^{p-1}\psi_2\}
\]
The Lanczos coefficients would then involve not only the autocovariances,
%$\bC_\pi(\phi_i(n), \phi_i(n+m))$
but also the cross-covariances, 
$\bC_\pi(\phi_i(n), \phi_j(n+m))$, defined in (2.6) above.

\newpage

%%%%% Appendices start %%%%%%%%%%%%%%%%
%% Comment out the following line if your thesis has no appendix
\singlespacing
\appendix
%%%-------------------------------------------------------------------
%%%
%%% \begin{center}
  {\Large Appendix: computer programs}
\end{center}

{\small 
\begin{verbatim}
% Matlab code ising.m
%
% Written by William J. DeMeo on 12/15/97
% last modified 2013.10.19
% Inputs:
%         d = number of nodes of the ising lattice (e.g. d=100)
%         n = number of iterations (e.g. n=1000)
%         beta = Annealing schedule (e.g. beta = 2,
%            when beta -> 0 will always go to nee state
%            when beta -> infty will never go to state of higher energy)
%
X = zeros(d,1);
Energy = zeros(n,1);
Ave=0;

% Initialize using hot start
for i=1:d,
  if rand < .5
    X(i) = -1;
  else
    X(i) = 1;
  end
end

% Initialize at state of high energy (not used)
if 0
  X(1:2:d-1)=1;
  X(2:2:d)=—1;
end

% Initialize energy
H=0;
for i=1:(d-1)
  H = H - X(i)*X(i+1);
end

% Initialize observable
phi = zeros(n+1,1);

for i=1:d
  if X(i) == 1
    phi(1)= phi(1)+1;
  end
end

for j=1:n
  site=0;
  while site==0
    site = round(rand*d);
  end
  % Compute new energy (with new X(site) = -X(site)):
  if site == 1
    Hnew = H + 2*X(1)*X(2);
  elseif site == d
    Hnew = H + 2*X(d)*X(d-1);
  else
    Hnew = H + 2*X(site)*(X(site-1) + X(site+1));
  end

  % Change to new state in two ways:
  % with probability 1/2, flip the switch
  if rand < .5
    if (Hnew <= H) | (rand < exp(-beta*(Hnew-H)))
      X(site) = -X(site);
      H = Hnew;
      % Compute new value of observable
      if X(site) == 1
        phi(j+1) = phi(j) + 1;
      else
        phi(j+1) = phi(j) -1;
      end
    else phi(j+1)=phi(j);
    end
  else phi(j+1)=phi(j);
  end
  Energy(j) = H;
end
plot(Energy)

% A second observable
phi2 = zeros(n+1,1);
for i=1:n+1
  phi2(i) = phi(i)*phi(i);
end

% Write first observable to phi.dat:
fid = fopen('phi.dat','w');
fprintf(fid,'%f\n',phi);
fclose(fid);

% Write second observable to phi2.dat:
fid = fopen('phi2.dat','w');
fprintf(fid,'%f\n',phi2);
fc1ose(fid);

% Compute a few Lanczos coefficients by new method
START=1001;
V1=zeros(2);
C1=zeros(2);
C2=zeros(2);
C3=zeros(2);

V1 = cov(phi(START:n+1),phi(START:n+1));
V1 = V1(1,1);
C1 = cov(phi(START:n),phi(START+1:n+1));
C1 = C1(1,2);
C2 = cov(phi(START:n-1),phi(START+2:n+1));
C2 = C2(1,2);
C3 = cov(phi(START:n-2),phi(START+3:n+1));
C3 = C3(1,2);

alpha1 = C1/V1;
beta1 = sqrt(C2/V1 -(C1/V1)^2);
alpha2 = (1/(beta1^2))*(C3/V1 - 2*alpha1*(C2/V1) + alpha1^3);

T1 = [alphal beta1; beta1 alpha2]
V1 = cov(phi2(START:n+1),phi2(START:n+1));
V1 = V1(1,1);
C1 = cov(phi2(START:n),phi2(START+1:n+1));
C1 = C1(1,2);
C2 = cov(phi2(START:n-1),phi2(START+2:n+1));
C2 = C2(1,2);
C3 = cov(phi2(START:n-2),phi2(START+3:n+1));
C3 = C3(1,2);

alpha1 = C1/V1;
beta1 = sqrt(C2/V1 - (C1/V1)^2);
alpha2 = (1/(beta1^2))#(C3/V1 - 2*alpha1*(C2/V1) + alpha1^3);

T2 = [alpha1 beta1; beta1 alpha2]

fid = fopen('Tmats.dat','v');
fprintf(fid,'%f\n',T1);
fprintf(fid,'%f\n',T2);
fclose(fid);

%%% end ising.m

\end{verbatim}}

\vskip2cm
%
%
% -------------- 45.txt ------------------------------------------
%
%
{\small 
\begin{verbatim}
% Matlab code tpm.m for generating transition probability matrix
% and computing its eigenvalues directly
%
% Written by William J. DeMeo on 1/10/98
%
% Inputs:
%        d = number of nodes of the ising lattice (e.g. d=10)
%     beta = Annealing schedule (e.g. beta - 2,
%            when beta -> 0 will always go to new state
%            when beta -> infty will never go to state of higher energy)

disp('bui1ding proposition matrix...')

states = 2^d;
A = 0;
for i=0:(d-1)
  E=eye(2^i);
  A = [A E; E A];
end

B = .5*eye(states);
A = B + .5*(1/d)*A;
% our modified Glauber dynamics requires the B and the .5*(1/d)

disp('...done')

% display pattern of nonzero entries
% spy(A)

% check that all row sums are 1
check=O;
for i=1:states
  check = check + (not(sum(A(i,1:states))<.99));
end
% if not all rows sum to 1, print check = (# of rows with sum=1)
if not(check==states)
  check
  error('Row sums are not all 1')
end

% construct matrix of states
E = zeros(d,states);
flip=-1;
for i=1:d
  for j=1:2^(i-1):states
    flip = -1*flip;
    for k=0:2^(i-1)
      E(i,j+k) = flip;
      end
  end
end
E=E';

% display first 64 states
% E(1:64,:)

% compute energy of each state
H = zeros(states,1);
for i=1:states
  for j=1:d-1
    H(i) = H(i) - E(i,j)*E(i,j+1);
  end
end

disp('building Hetropolis transition matrix...')

for i=1:states
  for j=i+1:states
    if not(A(i,j)==0)
      if(H(j)>H(i))
        alt = A(i,j)*(1-exp(-beta*(H(j)-H(i))));
        A(i,i) = A(i,i)+a1t;
        A(i,j) = A(i,j)-alt;
      elseif(H(j)<H(i))
        alt = A(j,i)*(1-exp(-beta*(H(i)-H(j))));
        A(j,j) = A(j,j)+a1t:
        A(j,i) = A(j,i)-alt;
      end
    end
  end
end

disp('...done')

check=0;
for i=1:states
  check = check + (not(sum(A(i,1:states))<.99));
end

% if not all rows sum to 1, print check = (# of rows with aum=1)

if not(check==states)
  check
  error('Row sums are not all 1')
end

disp('computing eigenvalues of tpm...')
cput = cputime;
evals = eig(A);
ecput = tputime - cput;
disp('the CPU time (in secs) for computing eigenvalues of tpm: ')
ecput

%%% end tpm.m
\end{verbatim}}

\vskip2cm

% 50.txt


{\small 
\begin{verbatim}




/*************************************************************
 * lanczos.c main program for computing Lanczos coefficients *
 *                                                           *
 * Created by William J. DeMeo on 1/7/98                     *
 * Last modified 2013.10.19                                  *
 ************************************************************/

#include <stdlib.h>
#include <stdio.h>
#include <math.h>
#include "prototypes.h"

/* Machine constants on dino (Sun Ultra 1 at Courant) */
#define MACHEPS 1.15828708535533604981e-16
#define SQRTEPS 1.07623746699106147048e-08
#define MAX_NAME 100

void read_name(char *);

/* Prototypes from the file functions.c */
double alpha(int j);
double betasq(int j);
double form1(int j, int k);
double form2(int j, int k);
double moment(double *data, long n, 
              double *ave, double *var, double *cov, long k);

/* external variables to be used by functions */
double *phi,*cov, *var, *ave;

main()
{

  char *filename;
  FILE *ofp;
  double temp=O, *T;
  long i, j, nrow, nlanc, START;
  int flag=O;

  filename = cmalloc(MAX_NAME);

  printf("\nName of file containing observed values: ");
  read_name(filename);
  printf("\nTota1 number of observations in file (iterations): ");
  scanf("%u",&nrow);
  printf("\nNumber of leading observations to discard: ");
  scanf("%u",&START);
  printf("\nNumber of Lanczos coefficients desired: ");
  scanf("%u",&nlanc);
  while(2*nlanc > nrow-START)
    {
      printf("\nNot enough data for that many coefficients.\n");
      printf("\nEnter a smaller number of Lanczos coefficients: ");
      scanf("%u",&nlanc);
    }

  phi = dmalloc(nrow);
  cov=dmalloc(2*nlanc);
  var=dmalloc(1);
  ave=dmalloc(1);
  T = dmalloc(nlanc*nlanc); 
  for(i=0;i<nlanc;i++) 
    T[i]=(double) 0;

  /* observable phi is stored contiguously column-wise by MATLAB */
  matlabread(phi, nrow, 1, filename);

  /* send the observable, offset by START */
  moment(phi+START,nrow-START,ave, var, cov,2*nlanc-1);

  /* First column of T */
  T[0] = alpha(1);
  if((temp=betasq(1))>MACHEPS*10)
    {
      T[1]=sqrt(temp);

      /* General column of T */
      j=2:
      for(i=1; i<nlanc-1, flag==O;)
	{
	  T[i*nlanc+i-1]=sqrt(betasq(i));
	  T[i*nlanc+i] = alpha(i+1);
	  if((temp = betasq(i+1))>MACHEPS*10)
	    {
	      T[i*nlanc+i+1] = sqrt(temp);
	      i++;
	    }
	  else
	    flag=1;
	}

      /* Last column */
      if(flag!=1 && ((temp= betasq(nlanc-1))>MACHEPS*10))
	{
	  T[nlanc*nlanc-2] = sqrt(temp);
	  T[nlanc*nlanc-1] = alpha(nlanc);
	  printf("\nbeta(1) = %1f\nbeta(%d) = %lf (last beta)",
		 T[1],(nlanc-1),T[nlanc*nlanc-2]);
	}
      else
	{
	  printf("\nApproximate invariant space reached at step %d.",i);
	  printf("\nbeta(1) = %lf\nbeta(%d) = %lf (last accurate beta)".
		 T[1],i,T[i*nlanc+i-1]);
	  printf("\nbeta(%d)^2: %lf (first spurious resu1t)",i+1,temp);
	}
      printf("\nThe matrix T is: \n");
      matprint(T,nlanc,nlanc);
    }
  else
    {
      printf("\nmain(): Approximate invariant space reached at first step.");
      printf("\nalpha(1) - %lf\nbeta(1)^2: %lf (first spurious result)", 
             T[O], temp);
    }
  ofp = fopen("Tmat.m","v");
  check(ofp);
  matlabwrite(T,nlanc,nlanc,ofp);
  fclose(ofp);
}

void read_name(char *name)
{
  int c, i = 0;
  while ((c = getchar()) != EOF && c != ' ' && c != '\n')
    name[i++] = c;
  name[i] = '\0';
}

/** end lanczos.c **/


\end{verbatim}}

\vskip2cm


{\small 
\begin{verbatim}


/***************************************************
 * functions.c -- functions required by lanczos.c  *
 *                                                 *
 * Created by William J. DeMeo on 1/7/98           *
 * Last modified 2013/10/19                        *
 ***************************************************/

#include <math.h>
#define START 1000
#define ITER 10000

/* Machine constants on dino (Sun Ultra 1 at Courant) */
#define MACHEPS 1.15828708535533604981e-16
#define SQRTEPS 1.07623746699106147048e-O8

double alpha(int j);
double betasq(int j);
double form1(int j, int k);
double form2(int j, int k);
double moment(double *data, long n, 
              double *ave, double *var, double *cov, int k);

/* external variables to be used by functions */
double *phi,*cov, *var, *ave;

double alpha(int j)
{
  /* alpha is never called with j < 1 */
  if(j==1)
    return form1(1,1);
  else if(j>1)
    return
      ((double)1/betasq(j-1)) * (form1(j-1,3) - pow(alpha(j-1),3)
       - 2 * (alpha(j-1)*betasq(j-1) + sqrt(betasq(j-2))*form2(j-1,2))
       + betasq(j-2));
}


double betasq(int j)
{
  if(j==0)
    return (double)O;
  else if(j>0)
    return (form1(j,2) - pow(alpha(j),2) - betasq(j-1));
}


double form1(int j, int k)
{
  double form14, alpha1, alpha1sq, form12, form13;
  if(j==0)
    return (double)0;
  else if(j==1)
    {
      /* printf("\nvar = %lf, cov(%d) = %lf \n",*var,k,cov[k]); */
      return (cov[k])/(*var); /* the only real value */
    }
  else if(j>1)
    {
      return 
	((double)1/betasq(j-1))
	* ( form1(j-1,k+2) + pow(alpha(j-1),2) * form1(j-1,k)
	    + 2*(alpha(j-1) * sqrt(betasq(j-2)) * form2(j-1,k)
		 - alpha(j-1)*form1(j-1,k+1)
		 - sqrt(betasq(j—2))*form2(j-1,k+1) )
	    + betasq(j-2)*form1(j-2,k));
    }
}


double form2(int j, int k)
{
/* form2 is never called with j < 1 */
  if(j==1)
    return (double)0;
  else if(j>1)
    return
      (pov(betasq(j-1),-.5)) 3
      (form1(j-1,k+1) - alpha(j-1)*form1(j-1,k)
       - sqrt(betasq(j-2)) * form2(j—1,k));
}

/* moment() function for computing var and cov(k)
   arguments:
   data = a nxi array of doubles
   n = length of data[]
   ave =(on exit)= the average of data[]
   var =(on exit)= the variance of data[]
   cov =(on exit)= the covariance of data[i] and data[i+j] for i-1,...,k
   k = the max lag for cov above
*/
double moment(double *data, long n, 
              double *ave, double *var, double *cov, int k)
{
  /* Centered about data[0] algorithm: */
  long i,j;
  double ave1, ave2;
  
  *ave=O;*var=0; ave1=ave2=O;
  for(j=0;j<=k;j++)
    cov[j]=0;

  for(i=1;i<n;i++)
    {
      *ave += (data[i] - data[0]);
      *var += (data[i] - data[0])*(data[i] - data[0]);
    }
  *var /= (double)(n-1);
  *var -= (((*ave)/(double)n) * ((*ave)/(double)(n-1)));
  /* *ave = ((*ave)/(double)n) + data[0]; (the true average; not needed)*/

  for(j=0;j<=k;j++)
    {
      for(i=0;i<(n-j);i++)
	{
	  ave1 += (data[i] - data[0]);
	  ave2 += (data[i+j] - data[0]);
	  cov[j] += (data[i] - data[0])*(data[i+j] - data[0]);

	}

      ave1/=(double)(n-j); ave2/=(double)(n-j);
      cov[j] = ((cov[j] - (double)(n-j)*ave1*ave2)/(double)(n-j-1));
    }

}

/*** end funccions.c ***/

\end{verbatim}
}


%%%
\begin{center}
  {\Large Appendix: computer programs}
\end{center}

{\small 
\begin{verbatim}
% Matlab code ising.m
%
% Written by William J. DeMeo on 12/15/97
% last modified 2013.10.19
% Inputs:
%         d = number of nodes of the ising lattice (e.g. d=100)
%         n = number of iterations (e.g. n=1000)
%         beta = Annealing schedule (e.g. beta = 2,
%            when beta -> 0 will always go to nee state
%            when beta -> infty will never go to state of higher energy)
%
X = zeros(d,1);
Energy = zeros(n,1);
Ave=0;

% Initialize using hot start
for i=1:d,
  if rand < .5
    X(i) = -1;
  else
    X(i) = 1;
  end
end

% Initialize at state of high energy (not used)
if 0
  X(1:2:d-1)=1;
  X(2:2:d)=-1;
end

% Initialize energy
H=0;
for i=1:(d-1)
  H = H - X(i)*X(i+1);
end

% Initialize observable
phi = zeros(n+1,1);

for i=1:d
  if X(i) == 1
    phi(1)= phi(1)+1;
  end
end

for j=1:n
  site=0;
  while site==0
    site = round(rand*d);
  end
  % Compute new energy (with new X(site) = -X(site)):
  if site == 1
    Hnew = H + 2*X(1)*X(2);
  elseif site == d
    Hnew = H + 2*X(d)*X(d-1);
  else
    Hnew = H + 2*X(site)*(X(site-1) + X(site+1));
  end

  % Change to new state in two ways:
  % with probability 1/2, flip the switch
  if rand < .5
    if (Hnew <= H) | (rand < exp(-beta*(Hnew-H)))
      X(site) = -X(site);
      H = Hnew;
      % Compute new value of observable
      if X(site) == 1
        phi(j+1) = phi(j) + 1;
      else
        phi(j+1) = phi(j) -1;
      end
    else phi(j+1)=phi(j);
    end
  else phi(j+1)=phi(j);
  end
  Energy(j) = H;
end
plot(Energy)

% A second observable
phi2 = zeros(n+1,1);
for i=1:n+1
  phi2(i) = phi(i)*phi(i);
end

% Write first observable to phi.dat:
fid = fopen('phi.dat','w');
fprintf(fid,'%f\n',phi);
fclose(fid);

% Write second observable to phi2.dat:
fid = fopen('phi2.dat','w');
fprintf(fid,'%f\n',phi2);
fc1ose(fid);

% Compute a few Lanczos coefficients by new method
START=1001;
V1=zeros(2);
C1=zeros(2);
C2=zeros(2);
C3=zeros(2);

V1 = cov(phi(START:n+1),phi(START:n+1));
V1 = V1(1,1);
C1 = cov(phi(START:n),phi(START+1:n+1));
C1 = C1(1,2);
C2 = cov(phi(START:n-1),phi(START+2:n+1));
C2 = C2(1,2);
C3 = cov(phi(START:n-2),phi(START+3:n+1));
C3 = C3(1,2);

alpha1 = C1/V1;
beta1 = sqrt(C2/V1 -(C1/V1)^2);
alpha2 = (1/(beta1^2))*(C3/V1 - 2*alpha1*(C2/V1) + alpha1^3);

T1 = [alphal beta1; beta1 alpha2]
V1 = cov(phi2(START:n+1),phi2(START:n+1));
V1 = V1(1,1);
C1 = cov(phi2(START:n),phi2(START+1:n+1));
C1 = C1(1,2);
C2 = cov(phi2(START:n-1),phi2(START+2:n+1));
C2 = C2(1,2);
C3 = cov(phi2(START:n-2),phi2(START+3:n+1));
C3 = C3(1,2);

alpha1 = C1/V1;
beta1 = sqrt(C2/V1 - (C1/V1)^2);
alpha2 = (1/(beta1^2))#(C3/V1 - 2*alpha1*(C2/V1) + alpha1^3);

T2 = [alpha1 beta1; beta1 alpha2]

fid = fopen('Tmats.dat','v');
fprintf(fid,'%f\n',T1);
fprintf(fid,'%f\n',T2);
fclose(fid);

%%% end ising.m

\end{verbatim}}

\vskip2cm
%
%
% -------------- 45.txt ------------------------------------------
%
%
{\small 
\begin{verbatim}
% Matlab code tpm.m for generating transition probability matrix
% and computing its eigenvalues directly
%
% Written by William J. DeMeo on 1/10/98
%
% Inputs:
%        d = number of nodes of the ising lattice (e.g. d=10)
%     beta = Annealing schedule (e.g. beta - 2,
%            when beta -> 0 will always go to new state
%            when beta -> infty will never go to state of higher energy)

disp('bui1ding proposition matrix...')

states = 2^d;
A = 0;
for i=0:(d-1)
  E=eye(2^i);
  A = [A E; E A];
end

B = .5*eye(states);
A = B + .5*(1/d)*A;
% our modified Glauber dynamics requires the B and the .5*(1/d)

disp('...done')

% display pattern of nonzero entries
% spy(A)

% check that all row sums are 1
check=O;
for i=1:states
  check = check + (not(sum(A(i,1:states))<.99));
end
% if not all rows sum to 1, print check = (# of rows with sum=1)
if not(check==states)
  check
  error('Row sums are not all 1')
end

% construct matrix of states
E = zeros(d,states);
flip=-1;
for i=1:d
  for j=1:2^(i-1):states
    flip = -1*flip;
    for k=0:2^(i-1)
      E(i,j+k) = flip;
      end
  end
end
E=E';

% display first 64 states
% E(1:64,:)

% compute energy of each state
H = zeros(states,1);
for i=1:states
  for j=1:d-1
    H(i) = H(i) - E(i,j)*E(i,j+1);
  end
end

disp('building Hetropolis transition matrix...')

for i=1:states
  for j=i+1:states
    if not(A(i,j)==0)
      if(H(j)>H(i))
        alt = A(i,j)*(1-exp(-beta*(H(j)-H(i))));
        A(i,i) = A(i,i)+a1t;
        A(i,j) = A(i,j)-alt;
      elseif(H(j)<H(i))
        alt = A(j,i)*(1-exp(-beta*(H(i)-H(j))));
        A(j,j) = A(j,j)+a1t:
        A(j,i) = A(j,i)-alt;
      end
    end
  end
end

disp('...done')

check=0;
for i=1:states
  check = check + (not(sum(A(i,1:states))<.99));
end

% if not all rows sum to 1, print check = (# of rows with aum=1)

if not(check==states)
  check
  error('Row sums are not all 1')
end

disp('computing eigenvalues of tpm...')
cput = cputime;
evals = eig(A);
ecput = tputime - cput;
disp('the CPU time (in secs) for computing eigenvalues of tpm: ')
ecput

%%% end tpm.m
\end{verbatim}}

\vskip2cm

% 50.txt


{\small 
\begin{verbatim}




/*************************************************************
 * lanczos.c main program for computing Lanczos coefficients *
 *                                                           *
 * Created by William J. DeMeo on 1/7/98                     *
 * Last modified 2013.10.19                                  *
 ************************************************************/

#include <stdlib.h>
#include <stdio.h>
#include <math.h>
#include "prototypes.h"

/* Machine constants on dino (Sun Ultra 1 at Courant) */
#define MACHEPS 1.15828708535533604981e-16
#define SQRTEPS 1.07623746699106147048e-08
#define MAX_NAME 100

void read_name(char *);

/* Prototypes from the file functions.c */
double alpha(int j);
double betasq(int j);
double form1(int j, int k);
double form2(int j, int k);
double moment(double *data, long n, 
              double *ave, double *var, double *cov, long k);

/* external variables to be used by functions */
double *phi,*cov, *var, *ave;

main()
{

  char *filename;
  FILE *ofp;
  double temp=O, *T;
  long i, j, nrow, nlanc, START;
  int flag=O;

  filename = cmalloc(MAX_NAME);

  printf("\nName of file containing observed values: ");
  read_name(filename);
  printf("\nTota1 number of observations in file (iterations): ");
  scanf("%u",&nrow);
  printf("\nNumber of leading observations to discard: ");
  scanf("%u",&START);
  printf("\nNumber of Lanczos coefficients desired: ");
  scanf("%u",&nlanc);
  while(2*nlanc > nrow-START)
    {
      printf("\nNot enough data for that many coefficients.\n");
      printf("\nEnter a smaller number of Lanczos coefficients: ");
      scanf("%u",&nlanc);
    }

  phi = dmalloc(nrow);
  cov=dmalloc(2*nlanc);
  var=dmalloc(1);
  ave=dmalloc(1);
  T = dmalloc(nlanc*nlanc); 
  for(i=0;i<nlanc;i++) 
    T[i]=(double) 0;

  /* observable phi is stored contiguously column-wise by MATLAB */
  matlabread(phi, nrow, 1, filename);

  /* send the observable, offset by START */
  moment(phi+START,nrow-START,ave, var, cov,2*nlanc-1);

  /* First column of T */
  T[0] = alpha(1);
  if((temp=betasq(1))>MACHEPS*10)
    {
      T[1]=sqrt(temp);

      /* General column of T */
      j=2:
      for(i=1; i<nlanc-1, flag==O;)
	{
	  T[i*nlanc+i-1]=sqrt(betasq(i));
	  T[i*nlanc+i] = alpha(i+1);
	  if((temp = betasq(i+1))>MACHEPS*10)
	    {
	      T[i*nlanc+i+1] = sqrt(temp);
	      i++;
	    }
	  else
	    flag=1;
	}

      /* Last column */
      if(flag!=1 && ((temp= betasq(nlanc-1))>MACHEPS*10))
	{
	  T[nlanc*nlanc-2] = sqrt(temp);
	  T[nlanc*nlanc-1] = alpha(nlanc);
	  printf("\nbeta(1) = %1f\nbeta(%d) = %lf (last beta)",
		 T[1],(nlanc-1),T[nlanc*nlanc-2]);
	}
      else
	{
	  printf("\nApproximate invariant space reached at step %d.",i);
	  printf("\nbeta(1) = %lf\nbeta(%d) = %lf (last accurate beta)".
		 T[1],i,T[i*nlanc+i-1]);
	  printf("\nbeta(%d)^2: %lf (first spurious resu1t)",i+1,temp);
	}
      printf("\nThe matrix T is: \n");
      matprint(T,nlanc,nlanc);
    }
  else
    {
      printf("\nmain(): Approximate invariant space reached at first step.");
      printf("\nalpha(1) - %lf\nbeta(1)^2: %lf (first spurious result)", 
             T[O], temp);
    }
  ofp = fopen("Tmat.m","v");
  check(ofp);
  matlabwrite(T,nlanc,nlanc,ofp);
  fclose(ofp);
}

void read_name(char *name)
{
  int c, i = 0;
  while ((c = getchar()) != EOF && c != ' ' && c != '\n')
    name[i++] = c;
  name[i] = '\0';
}

/** end lanczos.c **/


\end{verbatim}}

\vskip2cm


{\small 
\begin{verbatim}


/***************************************************
 * functions.c -- functions required by lanczos.c  *
 *                                                 *
 * Created by William J. DeMeo on 1/7/98           *
 * Last modified 2013/10/19                        *
 ***************************************************/

#include <math.h>
#define START 1000
#define ITER 10000

/* Machine constants on dino (Sun Ultra 1 at Courant) */
#define MACHEPS 1.15828708535533604981e-16
#define SQRTEPS 1.07623746699106147048e-O8

double alpha(int j);
double betasq(int j);
double form1(int j, int k);
double form2(int j, int k);
double moment(double *data, long n, 
              double *ave, double *var, double *cov, int k);

/* external variables to be used by functions */
double *phi,*cov, *var, *ave;

double alpha(int j)
{
  /* alpha is never called with j < 1 */
  if(j==1)
    return form1(1,1);
  else if(j>1)
    return
      ((double)1/betasq(j-1)) * (form1(j-1,3) - pow(alpha(j-1),3)
       - 2 * (alpha(j-1)*betasq(j-1) + sqrt(betasq(j-2))*form2(j-1,2))
       + betasq(j-2));
}


double betasq(int j)
{
  if(j==0)
    return (double)O;
  else if(j>0)
    return (form1(j,2) - pow(alpha(j),2) - betasq(j-1));
}


double form1(int j, int k)
{
  double form14, alpha1, alpha1sq, form12, form13;
  if(j==0)
    return (double)0;
  else if(j==1)
    {
      /* printf("\nvar = %lf, cov(%d) = %lf \n",*var,k,cov[k]); */
      return (cov[k])/(*var); /* the only real value */
    }
  else if(j>1)
    {
      return 
	((double)1/betasq(j-1))
	* ( form1(j-1,k+2) + pow(alpha(j-1),2) * form1(j-1,k)
	    + 2*(alpha(j-1) * sqrt(betasq(j-2)) * form2(j-1,k)
		 - alpha(j-1)*form1(j-1,k+1)
		 - sqrt(betasq(j—2))*form2(j-1,k+1) )
	    + betasq(j-2)*form1(j-2,k));
    }
}


double form2(int j, int k)
{
/* form2 is never called with j < 1 */
  if(j==1)
    return (double)0;
  else if(j>1)
    return
      (pov(betasq(j-1),-.5)) 3
      (form1(j-1,k+1) - alpha(j-1)*form1(j-1,k)
       - sqrt(betasq(j-2)) * form2(j-1,k));
}

/* moment() function for computing var and cov(k)
   arguments:
   data = a nxi array of doubles
   n = length of data[]
   ave =(on exit)= the average of data[]
   var =(on exit)= the variance of data[]
   cov =(on exit)= the covariance of data[i] and data[i+j] for i-1,...,k
   k = the max lag for cov above
*/
double moment(double *data, long n, 
              double *ave, double *var, double *cov, int k)
{
  /* Centered about data[0] algorithm: */
  long i,j;
  double ave1, ave2;
  
  *ave=O;*var=0; ave1=ave2=O;
  for(j=0;j<=k;j++)
    cov[j]=0;

  for(i=1;i<n;i++)
    {
      *ave += (data[i] - data[0]);
      *var += (data[i] - data[0])*(data[i] - data[0]);
    }
  *var /= (double)(n-1);
  *var -= (((*ave)/(double)n) * ((*ave)/(double)(n-1)));
  /* *ave = ((*ave)/(double)n) + data[0]; (the true average; not needed)*/

  for(j=0;j<=k;j++)
    {
      for(i=0;i<(n-j);i++)
	{
	  ave1 += (data[i] - data[0]);
	  ave2 += (data[i+j] - data[0]);
	  cov[j] += (data[i] - data[0])*(data[i+j] - data[0]);

	}

      ave1/=(double)(n-j); ave2/=(double)(n-j);
      cov[j] = ((cov[j] - (double)(n-j)*ave1*ave2)/(double)(n-j-1));
    }

}

/*** end funccions.c ***/

\end{verbatim}
}



%%%-------------------------------------------------------------------
%%%
%%% BIBLIOGRAPHY
%%%


%% Note: If your thesis has more than one appendix, NYU requires a "list of
%% appendices" page before the body of the thesis. I don't provide the tools
%% to create that here, so you're on your own for that one... Sorry.
%\input{app2}
%%%% Input bibliography file %%%%%%%%%%%%%%%
%\input{biblio}

%% \bibliographystyle{plainurl}
%% \bibliography{wjd}
\def\cprime{$'$} \def\cprime{$'$}
  \def\ocirc#1{\ifmmode\setbox0=\hbox{$#1$}\dimen0=\ht0 \advance\dimen0
  by1pt\rlap{\hbox to\wd0{\hss\raise\dimen0
  \hbox{\hskip.2em$\scriptscriptstyle\circ$}\hss}}#1\else {\accent"17 #1}\fi}
\begin{thebibliography}{1}

\bibitem{Demmel:1997}
J.~Demmel.
\newblock {\em Applied Numerical Linear Algebra}.
\newblock Society for Industrial and Applied Mathematics, 1997.

\bibitem{Diaconis:1996}
P.~Diaconis and L.~Saloff-Coste.
\newblock Logarithmic {S}obolev inequalities for finite {M}arkov chains.
\newblock {\em The Annals of Applied Probability}, 6:695--750, 1996.

\bibitem{Durret:1996}
R.~Durrett.
\newblock {\em Probability: Theory and Examples}.
\newblock Duxbury Press, second edition, 1996.

\bibitem{Geman:1984}
S.~Geman and D.~Geman.
\newblock Stochastic relaxation. {G}ibbs distributions and the {B}ayesian
  restoration of images.
\newblock {\em IEEE Transactions on Pattern Analysis and Machine Intelligence},
  6:721--741, 1984.

\bibitem{Golub:1996}
G.~Golub and Charles Van~Loan.
\newblock {\em Matrix Computations}.
\newblock Johns Hopkins University Press, third edition, 1996.

\bibitem{GoodmanSokal:1989}
G.~Goodman and A.~Sokal.
\newblock Multigrid {M}onte {C}arlo method. conceptual foundations.
\newblock {\em Physical Review D}, 40:2035--2071, 1989.

\bibitem{Hammersley:1964}
J.~M. Hammersley and D.~C. Handscomb.
\newblock {\em Monte Carlo Methods}.
\newblock Chapman and Hall, 1964.

\bibitem{Metropolis:1953}
N.~Metropolis, A.~W. Rosenbluth, M.~N. Rosenbluth, A.~H. Teller, and E.~Teller.
\newblock Equations of state calculations by fast computing machines.
\newblock {\em J. Chem. Phys.}, 21:1087--1092, 1953.

\bibitem{Rosenthal:1995}
J.~Rosenthal.
\newblock Convergence rates for {M}arkov chains.
\newblock {\em Siam Review}, 37:387--405, 1995.

\end{thebibliography}

\end{document}

\end{document}
