
\singlespacing

{\small 
\begin{verbatim}
% Matlab code ising.m
%
% Written by William J. DeMeo on 12/15/97
% last modified 2013.10.19
% Inputs:
%         d = number of nodes of the ising lattice (e.g. d=100)
%         n = number of iterations (e.g. n=1000)
%         beta = Annealing schedule (e.g. beta = 2,
%                when beta -> 0 will always go to nee state
%                when beta -> infty will never go to state of higher energy)
%
X = zeros(d,1);
Energy = zeros(n,1);
Ave=0;

% Initialize using hot start
for i=1:d,
  if rand < .5
    X(i) = -1;
  else
    X(i) = 1;
  end
end

% Initialize at state of high energy (not used)
if 0
  X(1:2:d-1)=1;
  X(2:2:d)=—1;
end

% Initialize energy
H=0;
for i=1:(d-1)
  H = H - X(i)*X(i+1);
end

% Initialize observable
phi = zeros(n+1,1);

for i=1:d
  if X(i) == 1
    phi(1)= phi(1)+1;
  end
end

for j=1:n
  site=0;
  while site==0
    site = round(rand*d);
  end
  % Compute new energy (with new X(site) = -X(site)):
  if site == 1
    Hnew = H + 2*X(1)*X(2);
  elseif site == d
    Hnew = H + 2*X(d)*X(d-1);
  else
    Hnew = H + 2*X(site)*(X(site-1) + X(site+1));
  end

  % Change to new state in two ways:
  % with probability 1/2, flip the switch
  if rand < .5
    if (Hnew <= H) | (rand < exp(-beta*(Hnew-H)))
      X(site) = -X(site);
      H = Hnew;
      % Compute new value of observable
      if X(site) == 1
        phi(j+1) = phi(j) + 1;
      else
        phi(j+1) = phi(j) -1;
      end
    else phi(j+1)=phi(j);
    end
  else phi(j+1)=phi(j);
  end
  Energy(j) = H;
end
plot(Energy)

% A second observable
phi2 = zeros(n+1,1);
for i=1:n+1
  phi2(i) = phi(i)*phi(i);
end

% Write first observable to phi.dat:
fid = fopen('phi.dat','w');
fprintf(fid,'%f\n',phi);
fclose(fid);

% Write second observable to phi2.dat:
fid = fopen('phi2.dat','w');
fprintf(fid,'%f\n',phi2);
fc1ose(fid);

% Compute a few Lanczos coefficients by new method
START=1001;
V1=zeros(2);
C1=zeros(2);
C2=zeros(2);
C3=zeros(2);

V1 = cov(phi(START:n+1),phi(START:n+1));
V1 = V1(1,1);
C1 = cov(phi(START:n),phi(START+1:n+1));
C1 = C1(1,2);
C2 = cov(phi(START:n-1),phi(START+2:n+1));
C2 = C2(1,2);
C3 = cov(phi(START:n-2),phi(START+3:n+1));
C3 = C3(1,2);

alpha1 = C1/V1;
beta1 = sqrt(C2/V1 -(C1/V1)^2);
alpha2 = (1/(beta1^2))*(C3/V1 - 2*alpha1*(C2/V1) + alpha1^3);

T1 = [alphal beta1; beta1 alpha2]
V1 = cov(phi2(START:n+1),phi2(START:n+1));
V1 = V1(1,1);
C1 = cov(phi2(START:n),phi2(START+1:n+1));
C1 = C1(1,2);
C2 = cov(phi2(START:n-1),phi2(START+2:n+1));
C2 = C2(1,2);
C3 = cov(phi2(START:n-2),phi2(START+3:n+1));
C3 = C3(1,2);

alpha1 = C1/V1;
beta1 = sqrt(C2/V1 - (C1/V1)^2);
alpha2 = (1/(beta1^2))#(C3/V1 - 2*alpha1*(C2/V1) + alpha1^3);

T2 = [alpha1 beta1; beta1 alpha2]

fid = fopen('Tmats.dat','v');
fprintf(fid,'%f\n',T1);
fprintf(fid,'%f\n',T2);
fclose(fid);

%%% end ising.m

\end{verbatim}}

\vskip2cm
%
%
% -------------- 45.txt ------------------------------------------
%
%
{\small 
\begin{verbatim}
% Matlab code tpm.m for generating transition probability matrix
% and computing its eigenvalues directly
%
% Written by William J. DeMeo on 1/10/98
%
% Inputs:
%        d = number of nodes of the ising lattice (e.g. d=10)
%     beta = Annealing schedule (e.g. beta - 2,
%            when beta -> 0 will always go to new state
%            when beta -> infty will never go to state of higher energy)

disp('bui1ding proposition matrix...')

states = 2^d;
A = 0;
for i=0:(d-1)
  E=eye(2^i);
  A = [A E; E A];
end

B = .5*eye(states);
A = B + .5*(1/d)*A;
% our modified Glauber dynamics requires the B and the .5*(1/d)

disp('...done')

% display pattern of nonzero entries
% spy(A)

% check that all row sums are 1
check=O;
for i=1:states
  check = check + (not(sum(A(i,1:states))<.99));
end
% if not all rows sum to 1, print check = (# of rows with sum=1)
if not(check==states)
  check
  error('Row sums are not all 1')
end

% construct matrix of states
E = zeros(d,states);
flip=-1;
for i=1:d
  for j=1:2^(i-1):states
    flip = -1*flip;
    for k=0:2^(i-1)
      E(i,j+k) = flip;
      end
  end
end
E=E';

% display first 64 states
% E(1:64,:)

% compute energy of each state
H = zeros(states,1);
for i=1:states
  for j=1:d-1
    H(i) = H(i) - E(i,j)*E(i,j+1);
  end
end

disp('building Hetropolis transition matrix...')

for i=1:states
  for j=i+1:states
    if not(A(i,j)==0)
      if(H(j)>H(i))
        alt = A(i,j)*(1-exp(-beta*(H(j)-H(i))));
        A(i,i) = A(i,i)+a1t;
        A(i,j) = A(i,j)-alt;
      elseif(H(j)<H(i))
        alt = A(j,i)*(1-exp(-beta*(H(i)-H(j))));
        A(j,j) = A(j,j)+a1t:
        A(j,i) = A(j,i)-alt;
      end
    end
  end
end

disp('...done')

check=0;
for i=1:states
  check = check + (not(sum(A(i,1:states))<.99));
end

% if not all rows sum to 1, print check = (# of rows with aum=1)

if not(check==states)
  check
  error('Row sums are not all 1')
end

disp('computing eigenvalues of tpm...')
cput = cputime;
evals = eig(A);
ecput = tputime - cput;
disp('the CPU time (in secs) for computing eigenvalues of tpm: ')
ecput

%%% end tpm.m
\end{verbatim}}

\vskip2cm

% 50.txt


{\small 
\begin{verbatim}




/*************************************************************
 * lanczos.c main program for computing Lanczos coefficients *
 *                                                           *
 * Created by William J. DeMeo on 1/7/98                     *
 * Last modified 2013.10.19                                  *
 ************************************************************/

#include <stdlib.h>
#include <stdio.h>
#include <math.h>
#include "prototypes.h"

/* Machine constants on dino (Sun Ultra 1 at Courant) */
#define MACHEPS 1.15828708535533604981e-16
#define SQRTEPS 1.07623746699106147048e-08
#define MAX_NAME 100

void read_name(char *);

/* Prototypes from the file functions.c */
double alpha(int j);
double betasq(int j);
double form1(int j, int k);
double form2(int j, int k);
double moment(double *data, long n, double *ave, double *var, double *cov, long k);

/* external variables to be used by functions */
double *phi,*cov, *var, *ave;

main()
{

  char *filename;
  FILE *ofp;
  double temp=O, *T;
  long i, j, nrow, nlanc, START;
  int flag=O;

  filename = cmalloc(MAX_NAME);

  printf("\nName of file containing observed values: ");
  read_name(filename);
  printf("\nTota1 number of observations in file (iterations): ");
  scanf("%u",&nrow);
  printf("\nNumber of leading observations to discard: ");
  scanf("%u",&START);
  printf("\nNumber of Lanczos coefficients desired: ");
  scanf("%u",&nlanc);
  while(2*nlanc > nrow-START)
    {
      printf("\nNot enough data for that many coefficients.\n");
      printf("\nEnter a smaller number of Lanczos coefficients: ");
      scanf("%u",&nlanc);
    }

  phi = dmalloc(nrow);
  cov=dmalloc(2*nlanc);
  var=dmalloc(1);
  ave=dmalloc(1);
  T = dmalloc(nlanc*nlanc); 
  for(i=0;i<nlanc;i++) 
    T[i]=(double) 0;

  /* observable phi is stored contiguously column-wise (e.g. by MATLAB) */
  matlabread(phi, nrow, 1, filename);

  /* send the observable, offset by START */
  moment(phi+START,nrow-START,ave, var, cov,2*nlanc-1);

  /* First column of T */
  T[0] = alpha(1);
  if((temp=betasq(1))>MACHEPS*10)
    {
      T[1]=sqrt(temp);

      /* General column of T */
      j=2:
      for(i=1; i<nlanc-1, flag==O;)
	{
	  T[i*nlanc+i-1]=sqrt(betasq(i));
	  T[i*nlanc+i] = alpha(i+1);
	  if((temp = betasq(i+1))>MACHEPS*10)
	    {
	      T[i*nlanc+i+1] = sqrt(temp);
	      i++;
	    }
	  else
	    flag=1;
	}

      /* Last column */
      if(flag!=1 && ((temp= betasq(nlanc-1))>MACHEPS*10))
	{
	  T[nlanc*nlanc-2] = sqrt(temp);
	  T[nlanc*nlanc-1] = alpha(nlanc);
	  printf("\nbeta(1) = %1f\nbeta(%d) = %lf (last beta)",
		 T[1],(nlanc-1),T[nlanc*nlanc-2]);
	}
      else
	{
	  printf("\nApproximate invariant space reached at step %d.",i);
	  printf("\nbeta(1) = %lf\nbeta(%d) = %lf (last accurate beta)".
		 T[1],i,T[i*nlanc+i-1]);
	  printf("\nbeta(%d)^2: %lf (first spurious resu1t)",i+1,temp);
	}
      printf("\nThe matrix T is: \n");
      matprint(T,nlanc,nlanc);
    }
  else
    {
      printf("\nmain(): Approximate invariant space reached at first step.");
      printf("\nalpha(1) - %1f\nbeta(1)^2: %lf (first spurious result)", T[O], temp);
    }
  ofp = fopen("Tmat.m","v");
  check(ofp);
  matlabwrite(T,nlanc,nlanc,ofp);
  fclose(ofp);
}

void read_name(char *name)
{
  int c, i = 0;
  while ((c = getchar()) != EOF && c != ' ' && c != '\n')
    name[i++] = c;
  name[i] = '\0';
}

/** end lanczos.c **/


\end{verbatim}}

\vskip2cm


{\small 
\begin{verbatim}


/***************************************************
 * functions.c -- functions required by lanczos.c  *
 *                                                 *
 * Created by William J. DeMeo on 1/7/98           *
 * Last modified 2013/10/19                        *
 ***************************************************/

#include <math.h>
#define START 1000
#define ITER 10000

/* Machine constants on dino (Sun Ultra 1 at Courant) */
#define MACHEPS 1.15828708535533604981e-16
#define SQRTEPS 1.07623746699106147048e-O8

double alpha(int j);
double betasq(int j);
double form1(int j, int k);
double form2(int j, int k);
double moment(double *data, long n, double *ave, double *var, double *cov, int k);

/* external variables to be used by functions */
double *phi,*cov, *var, *ave;

double alpha(int j)
{
  /* alpha is never called with j < 1 */
  if(j==1)
    return form1(1,1);
  else if(j>1)
    return
      ((double)1/betasq(j-1)) * (form1(j-1,3) - pow(alpha(j-1),3)
       - 2 * (alpha(j-1)*betasq(j-1) + sqrt(betasq(j-2))*form2(j-1,2))
       + betasq(j-2));
}


double betasq(int j)
{
  if(j==0)
    return (double)O;
  else if(j>0)
    return (form1(j,2) - pow(alpha(j),2) - betasq(j-1));
}


double form1(int j, int k)
{
  double form14, alpha1, alpha1sq, form12, form13;
  if(j==0)
    return (double)0;
  else if(j==1)
    {
      /* printf("\nvar = %lf, cov(%d) = %lf \n",*var,k,cov[k]); */
      return (cov[k])/(*var); /* the only real value */
    }
  else if(j>1)
    {
      return 
	((double)1/betasq(j-1))
	* ( form1(j-1,k+2) + pow(alpha(j-1),2) * form1(j-1,k)
	    + 2*(alpha(j-1) * sqrt(betasq(j-2)) * form2(j-1,k)
		 - alpha(j-1)*form1(j-1,k+1)
		 - sqrt(betasq(j—2))*form2(j-1,k+1) )
	    + betasq(j-2)*form1(j-2,k)):
    }
}


double form2(int j, int k)
{
/* form2 is never called with j < 1 */
  if(j==1)
    return (double)0;
  else if(j>1)
    return
      (pov(betasq(j-1),-.5)) 3
      (form1(j-1,k+1) - alpha(j-1)*form1(j-1,k)
       - sqrt(betasq(j-2)) * form2(j—1,k));
}

/* moment() function for computing var and cov(k)
   arguments:
   data = a nxi array of doubles
   n = length of data[]
   ave =(on exit)= the average of data[]
   var =(on exit)= the variance of data[]
   cov =(on exit)= the covariance of data[i] and data[i+j] for i-1,...,k
   k = the max lag for cov above
*/
double moment(double *data, long n, double *ave, double *var, double *cov, int k)
{
  /* Centered about data[0] algorithm: */
  long i,j;
  double ave1, ave2;
  
  *ave=O;*var=0; ave1=ave2=O;
  for(j=0;j<=k;j++)
    cov[j]=0;

  for(i=1;i<n;i++)
    {
      *ave += (data[i] - data[0]);
      *var += (data[i] - data[0])*(data[i] - data[0]);
    }
  *var /= (double)(n-1);
  *var -= (((*ave)/(double)n) * ((*ave)/(double)(n-1)));
  /* *ave = ((*ave)/(double)n) + data[0]; <- that's the true average (not needed)*/

  for(j=0;j<=k;j++)
    {
      for(i=0;i<(n-j);i++)
	{
	  ave1 += (data[i] - data[0]);
	  ave2 += (data[i+j] - data[0]);
	  cov[j] += (data[i] - data[0])*(data[i+j] - data[0]);

	}

      ave1/=(double)(n-j); ave2/=(double)(n-j);
      cov[j] = ((cov[j] - (double)(n-j)*ave1*ave2)/(double)(n-j-1));
    }

}

/*** end funccions.c ***/




\end{verbatim}}

