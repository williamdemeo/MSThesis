\chapter{Conclusion}
% Chapter 6
\label{cha:conclusion}

We conclude by noting a few aspects of the special cases to which the foregoing applies.
We are interested in the subdominant eigenvalue of a large transition probability matrix. The

foregoing can be used to approximate these eigenvalues when the following conditions are satisfied:
o The Markov chain is reversible (detailed balance condition).

0 We can easily simulate realizations of the chain as well as functions (observables) of these

realizations.

When these conditions hold, we can side step the traditional Lanczos algorithm, and get at the
values of the Lanczos coeflicients via the variational properties of observables on the state space.
There are many ideas in this paper that should be expanded upon in future studies. Per-
haps most obvious is the desire to find analogous techniques for non-reversible (i.e. non-symmetric)
Markov chains. This is a considerable problem since the Lanczos methods described above are no
longer applicable. Perhaps the Amoldi algorithm ([1], algorithm 6.9), or the non-symmetric Lanczos

algorithm can also be usefully specialized for stochastic matrices.

% 41.txt
41

When applying the Lanczos algorithm, we usually choose a single starting vector (in the
present context, an observable), perform the algorithm for a number of steps, and gather results.
Then we do the same with a new starting vector and compare the results to those obtained with
the first vector. Proceeding in this way for a number of starting vectors, we can provide statistical
evidence that we have, indeed, located the eigenvalues of interest. However, when there are two
observables 451, :15; of particular interest, perhaps we would benefit by considering both observables
simultaneously. Future studies might develop an algorithm which incorporates both observables,

again via ¢.- = H1/2(¢.' - E,.¢.-). In this case, a Lanczos algorithm would be based on the space

Ic*(M,¢1,¢»2.p) 2 ran{«/»mz»2,M«/»1,M«/:2, . . .,MP—‘¢1.M"—‘«/22}

The Lanczos coefiicients would then involve not only the autocovariances C, (¢,<(n),¢,-(n + m)),

but also the cross-covariances C, (¢.~(n), ¢j(n + m)) (defined in (2.6)).
