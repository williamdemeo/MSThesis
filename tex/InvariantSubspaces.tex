\chapter{Invariant and Approximate Invariant Subspaces}

\section{Invariant Subspaces}
Definition 3.1.1 A subspace S Q R" with the property that

1:ES'=Ma:€S

is said to be invariant for M.

Recall that, having chosen observables <I> = ((151, . . . , qfip), we constructed the covariance
matrix C(s) = \II‘M"\II. If the column space of \II, which we denote by ran(\II), is an invariant
subspace for M, the definition implies Mifij E  That is, for each 111,- there exists a vector
t of coeflicients such that M1/Jj = f=1t,-1/)5,-. This is true for all j and, putting each vector of
coefficients into a matrix T, we see that M\II = \IIT. Conversely, M\II = IIIT implies that M1/Jj is a

linear combination of columns of \II, so ran(\II) is invariant. We have thus proved the following
% 15.txt
Fact 3.1.1 The subspace ran(\II) is invariant for M if and only if there exists T E R”? such that

M\II = \IIT.

Consequently,

Fact 3.1.2 )\(T) Q ).(M)

Proof of 3.1.2:

z\E)\(T) _=_ 3v€R"3Tv=).v
<=> \IITv = A\IIv
<=> M\II-u = /\\IIv
< /\ E /\(M)

The second ¢> follows from fact 3.1.1.
Facts 3.1.1 and 3.1.2 are theoretically useful. To see why, write the equation of fact 3.1.1

as follows:

\IIT = M\II
\II‘\IIT = \II‘M\II
T = (\1:'~1r)-‘\1:'M\1:

In this form, we recognize that T = C‘1(0)C(1). That is, using only the covariance of observables
on the state space, we can generate a matrix T with the property A(T) (_Z ).(M). Recalling that
M = H1/2Pl'I‘1/2, we see that M is similar to our original stochastic matrix P, and thus ).(M) =

).(P).

\section{Approximate Invariant Subspace}

We qualified the foregoing by stating that the facts are only theoretically useful. This

is because when choosing observables of the state space we aren't sure that they will satisfy the
% 16.txt
primary assumption underlying facts 3.1.1 and 3.1.2. Recall the assumption: ran(\II) is an invariant
subspace for M, where ‘II = H1/2(<I> —  It may well be the case that there is a 1p; 6 ran(\II) such
that M1/zj ¢ ran(\II), thereby violating the assumption. However, lacking an invariant subspace, we

might hope that our observables at least provide an approximate invariant subspace.

Definition 3.2.1 If the columns of ‘II E R“? are independent and the residual matrix E E

M\II —— \IIT is small for some T 6 R7”, then ran(\II) defines an approximate invariant subspace.

To see how an approximate invariant subspace can be useful in approximating eigenvalues of M,
we extract a theorem from Golub and Van Loan 
% Theorem 3.2.1
\begin{theorem}
\label{thm-3.2.1}  
Suppose M 6 Rd” and T 6 RP"? are symmetric and let E E MQ — QT, where Q 6 Rd"? satisfies

Q‘Q = I. Then there exist p1,... ,;z,, E A(T) and A1,. . . , A,, E A(M) such that
ll‘k - Akl S \/§|lE"

fork=1,...,p.
\end{theorem}

If the subspace ran(Q) is an approximate invariant subspace, then the definition implies that
there is a choice T rendering the error IIEII2 small, and thus, the eigenvalues of T provide a good
approximation to those of M.

When considering the preceding ideas, it becomes apparent that their application will
present new — but hopefully less prohibitive — problems. As these problems summarize the rt of
the paper, now is a good time to examine them. First, the preceding theorem assumes a matrix Q
whose columns form an orthonormal basis for the approximate invariant subspace. For our problem,
derivation of such a Q is tricky, and we must prepare for this. Next, having an approximate invariant

subspace at our disposal merely tells us that there exists some matrix T which makes the error
% 17.txt
"EH2 small. We must discover the form of such a T. Moreover, it is natural to seek that T which
minimizes IIEII2 for a given approximate invariant subspace. Finally, in order to apply these ideas
to real examples of our eigenvalue problem, we must find a practical way to generate the most
appropriate approximate invariant subspace. We now address each of these issues in the order
raised.

Recall the matrix I1 E III/2(<I> — (<I>)). The columns of this matrix, though independent
(by choice of independent observables), are not necessarily orthonormal. However, consider the
polar decomposition ‘II = QZ where Q‘Q = I and Z2 = \Il‘\II is symmetric positive semidefinite.1

Note that Z = (\II‘\II)l/2 is nonsingular, so Q has the form
Q = ~1:z-1 = \II(\II‘\II)“/2

and it is clear that ran(Q) = ran(\II). Perhaps ra.n(Q) is a. useful approximation to the invariant
subspace for M. If ran(Q) is not itself invariant, we have the error E = MQ — QT. It is easy to
show, as we do in the following proof, that the T which minimizes HE“; is T = Q‘MQ. This yields

the following,
Theorem 3.2.2

If \I1 = QZ is the polar decomposition of \II, then the matrix
T = Q‘MQ = (\1:*\1:)-1/’\1:‘M\I:(\Ir‘\I:)-1/2
minimizes IIEII2 = IIMQ - QTII2

Proof:
We prove the result by establishing the following

(Hal: Given M 6 Rd” suppose Q 6 Rd"? satisfies Q'Q =1. Then,

min ||MQ — QTII2 = H(1- QQ')MQll2

TERI‘?

‘Recall that the polar decomposition is derived from the SVD, \II = UZJV‘, by letting Q = UV‘ and Z = VEV‘.

% 18.txt
and T = Q’MQ is the minimizer.
To verify the claim, observe the following application of the Pythagorean theorem:

For any T E R.”"” we have

IIMQ — QT||§ = IIMQ - QQ'MQ + QQ'MQ - QT|l§
||(I - QQ’)MQ + Q(Q'MQ - T) H3

||(1 - QQ‘)MQ||§ + |lQ(Q'MQ - T)||§ (3-1)

2 l|(I — QQ')MQ||§

Equality (3.1) holds since I — QQ' projects MQ onto the subspace orthogonal to ran(Q). Thus,
the two terms in the expression on the right are orthogonal, and the Pythagorean theorem yields
equality. The concluding inequality establishes that the minimizing T is that which annihilates the
second term in (3.1); i.e. T = Q‘MQ.

The second equality in Theorem 3.2.2 is a consequence of the polar decomposition, in

which Q = \IIZ“ = \I:(~I:‘\I:)-1/2. Therefore,
T = (x1z'x1:)-1/2x1:‘M~1:(\1:*~1z)—1/2

is the minimizer.

3.3 The Krylov Subspace

Suppose the columns of a matrix Q E R“? provide an orthonormal basis for an approxi-

mate invariant subspace. Then we have seen,
1. T = Q'MQ minimizes ||E||2 = ||MQ — QT||2 and
2. E|p.1,...,p,, E z\(T) and /\1,...,A, 6 »\(M) such that

ll-‘I: - Ml S \/§|lE||2

% 19.txt
19
fork=1,...,p.

Given an approximate invariant subspace ran(Q) of dimension 1), these facts tell us what matrix we
should use to approximate p elements of M’s spectrum. Now all we lack is a. description of ran(Q).
That is, we have not specified which approximate invariant subspace would best suit our objective

of approximating the subdominant eigenvalue A,,.a2:(M). 'I‘o do so requires the following definition:

Definition 3.3.1 The Raleigh quotient of a symmetric matrix M and a nonzero vector 3 is

p(2:, M) E (:z:‘Ma:)/(z‘a:).

We will also denote this by p(z) when the context makes clear what matrix is involved.

To find the approximate invariant subspace most appropriate for our problem, we choose
each dimension successively, providing justification at each step. Thus, start with one observable
42 on the state space, and let zpl E 111/ 2(¢ — E,,¢). That is, 1/:1 represents a centered (mean zero)
observable whose ith coordinate is weighted by . First notice that the definition (which comes
directly from the definition of ‘II following equation (2.7)) is such that 1/)1 is a constant vector if
and only if :15 is oonstant on the state space, in which case 1/); E 0. This fact provides one reason
observables that are constant on the state space are not interesting. Second, recall that the row
sums of the matrix P are all one and, therefore, the eigenvector corresponding to the eigenvalue
Ao(P) = 1 is the constant vector. By definition of M E 1'11/2Pl'I‘1/2, we see that the constant vector
is also the eigenvector corresponding to A0  This will play an important role in what follows, as
it allows us to focus primarily on the subdominant eigenvalue z\max(M) E max{}q (M), [A44 
rather than on A0  (which we already know is 1).

Now, notice that

lP(¢1,M)l 5 max |P(x,M)l = »\max(M) (3-2)
% 20.txt
where the max is taken over all non-constant vectors. Since our interest centers on An-.3_x(M), we
would like a. 1/11 which makes the left hand side of (3.2) large. This would be achieved if 1/)1 were to
lie in the space spanned by, say, the first two eigenvectors of the Markov chain (sometimes referred
to as the slowest modes of the process). However, this subspace spans only two dimensions of the
entire d-dimensional space, and it is more likely that ¢1, at best, only comes close to lying in the
subspace of interest. Now, given 1/11, a judicious choice for the second dimension l/)2, and hence
W; E [1/)1 1/:2], would be that which makes max,,¢o |p(\II2a)| is as large as possible. To establish that

this is indeed the pertinent objective, note the following:

   

_ a‘\1:5Mx1:2a
I;1;g<Ip(\I'2a,M)l = 13:3: ————a,W§q,2a
= M
mrggf,mIp(z. )1
5 maxlp(z,M)l (3-3)
= ).,,m(M)

Again, the max on the right side of (3.3) is over non-constant vectors. In words, we wish to chose
\II-2 = [1/:1 ¢2] so that there is a vector a 6 R2 making |p(\II2a,M)| close to )«max(M).

Now, p(1/11) changes most rapidly in the direction of the gradient Vp(¢1).

W0  (iiiiiiv---viiiléi)
= ,,;,l (Mam — p<¢.)«m (3.4)

So, to maximize the left hand side of (3.3), ‘I12 should be chosen so that the subspace ra.n(\II2)

contains the gradient vector. That is, we must choose 1/22 so that

VP(¢'1) E F3l1{¢1y¢2} E F-'%!1(‘I’2) (3-5)

Clearly, equation (3.4) implies Vp(1/:1) E ran{z/11,M1,b1}. Therefore, if ran{1/:1,1/)2} = ra.n{¢1,M¢1},

then (3.5) is satisfied.
% 21.txt
In general, having chosen \II,, E [1/)1 - -- 1/11,] so that

ran{'pl9¢21' " 1¢k} = ran{¢1vM1[)la ' ‘ ' vMk—11/’k}

we must chose 1/1;.“ so that, for any non-zero vector u 6 R",

VP(‘I’I:a) 6 ra‘n{¢11 ‘21 - ° - 31/Jk-F1} 
Now,
vp<~I~.a) = gmm — P(‘I’kfl)‘I’k“)

and therefore,

Vp(‘II;,a.) E ran(M\II;,) U ra.n(\IIk) = ra.n{¢1, M1/)1, . . . , Mktfil}

Thus, requirement (3.6) is met when ran{1/:1, 1/:2, . . . , 1/4+1} = ran{1/)1, M1/:1, . . . , Mk:/11}.

In conclusion, the p-dimensional approximate invariant subspace which is most suitable

for our objective is

/C(Mn/mp) 2 ran{«/:1,M¢1,... ,M"“«/21}

This is known as the K rylov subspace. Therefore, in answer to the problem posed at the outset
of this section, if we take the columns of Q to be an orthonormal basis for IC(M,z/)1, p), then the

eigenvalues of T = Q‘MQ should provide good estimates of p extremal eigenvalues of M.

% 22.txt
