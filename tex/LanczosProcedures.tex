\chapter{Lanczos Procedures}

The following will review the usual procedure for generating an orthonormal basis for
$\sK(\M, \psi_1, p)$, 
and conclude by showing that, for our problem, we can find the matrix 
$\Q^t\M\Q$
without actually carrying out this procedure. Note that this conclusion is essential since the procedure
requires matrix vector multiplication---an operation we have assumed impossible for large enough $d$.

\section{The General Lanczos Algorithm}
% Sec 4.1
\label{sec:gener-lancz-algor}
Let $\psip{k} =  \M^k \psi$ and consider the $d\times d$ matrix
\[
\Psi = [\psi, \M\psi, \dots, \M^{d-1}\psi] = [\psip{0}, \psip{1}, \dots, \psip{d-1}]
\]
Notice that $\M\Psi = [\psip{1}, \psip{2}, \dots, \psip{d-1}, \M^d \psi]$,
and assuming for the moment that $\Psi$ is nonsingular,
we can compute the vector $h = -\Psi^{-1}\M^d \psi$. Thus,
\begin{equation}
\label{eq:1}
\M\Psi = \Psi [e_2,\dots,e_d,-h],  
\end{equation}
where $e_j$ is the column vector with 1 in
the $j$th position and zeros elsewhere. We define $H = [e_2,\dots,e_d,-h]$,
so~(\ref{eq:1}) becomes $\M\Psi = \Psi H$.
%
%
% ----------------- 23.txt ----------------------------------------------
%
%
Equivalently,
\begin{equation}
  \label{eq:4.1}
H = 
\begin{pmatrix}
0 & 0      & \cdots & 0 & -h_1 \\
1 & 0      & \cdots & 0 & -h_2\\
  & \ddots &        &   & \vdots \\
  &        & \ddots & 0 & -h_{d-1}\\
  &        &        & 1 & -h_{d}
\end{pmatrix} = \Psi^{-1} \M \Psi.
\end{equation}
$\H$ is a \emph{companion matrix} which means that its characteristic polynomial
is $p(x) = x^d + \sum_{i=1}^d h_i x^{i-1}$.
Since $\H$ is similar to $\M$, finding the eigenvalues of $\M$ is equivalent to
finding the roots of $p(x)$. However, this is of little practical use since
finding $h$, constructing $p(x)$, and finding its roots is 
probably a harder problem than the one we started with. Instead, the value of decomposition (4.1)
derives from its \emph{upper Hessenberg form}. We exploit this property below.

Let $\Psi = \Q\R$ be the QR decomposition of $\Psi$. Since $\Psi$ is assumed nonsingular,
\[
\Psi_d^{-1}\M \Psi_d = (\R^{-1}\Q^t)\M(\Q\R) = \H
\]
Therefore, $\Q^t\M\Q = \R\H\R^{-1}$.  Let $\T \R\H\R^{-1}$.  Then, since $\R$
and $\R^{-1}$ are both upper triangular and $\H$ is upper Hessenberg $\T = \R\H\R^{-1}$ is also upper
Hessenberg. Furthermore, since $\M$ is symmetric, it is clear that 
$\T^t =\Q^t\M\Q = T$. Thus, $\T$ is both upper Hessenberg and symmetric. Therefore,
$\T$ is tridiagonal and we can write it as follows:
\begin{equation}
  \label{eq:4.2}
T = 
\begin{pmatrix}
\alpha_1 & \beta_1 &      0 & \cdots & 0 \\
\beta_1  & \alpha_2 & \ddots &        & \vdots\\
         & \ddots & \ddots & \ddots   &\\
\vdots   &       &  \ddots & \ddots & \beta_{d-1}\\
0  &   \cdots    &   & \beta_{d-1} &\alpha_d
\end{pmatrix}.
\end{equation}
%
%
% ------------ 24.txt ----------------------------------------------
%
%
Equating columns $j$ on both sides of $\M\Q = \Q\T$ yields
\begin{equation}
\label{eq:4.3}
\M q_j = \beta_{j-1}q_{j-1}+\alpha_j q_j + \beta_j q_{j+1}
\end{equation}
Since the columns of $\Q$ are orthonormal, multiplying both sides 
of~(\ref{eq:4.3})
by $q_j$ and $q_{j+1}$ yields
$\alpha_j = q_j^t \M q_j$ and $\beta_j = q_{j+1}^t \M q_j$.
The foregoing justifies what is called the \emph{Lanczos algorithm}, which is performed as follows:

\emph{The Lanczos algorithm for (partial) reduction to symmetric tridiaganal form}

91 = ¢/||¢||2, [30 = 0, 40 = 0

for j=1top

z=Mq_,-

5:^t = ||Z||2
if 5, = 0, quit
¢Ij+1 = 2/51

end for

The qj computed by the Lanczos algorithm are often called the Lanczos vectors. If the
loop in the algorithm is terminated because fig = 0, this indicates that an exact invariant subspace
has been computed, and is given by ran{q1 - - - qk}. Otherwise, we usually halt the algorithm after
p steps, in which case the algorithm converg to approximations of at most 12 eigenvalues.^t

Note that equation (4.3) results from the equation MQ = QT. The latter equation holds
since, starting with d vectors in our subspace, Q is a d Xd matrix whose columns form an orthonormal

basis for all of R^t, which is clearly an invariant “sub” space. The Lanczos algorithm above, however,

^tA detailed discussion of such convergence (and misconvergence) is given in 

% 25.txt
25

proceeds for only p steps, producing a d x p matrix Q, whose columns form an orthonormal basis
for the approximate invariant subspace IC(M, ¢, p), and a p X p matrix T, = Q'MQ. In that case,
\PsiMQ = QTp|[g =  is non-zero and gives the error bound described in Theorem 3.2.1. Now,

writing the full d x :1 matrix T of (4.2) as

01 51
/71

(4.4)

   

allows us to describe the error bound ||E||-2 in terms of the submatrix Tm, and, hence, in terms of

fizz-
Theorem 4.1.1

HT, and T,“ are the matrices appearing in (4.4), and if the p columns of Q are computed by the

Lanczos algorithm, then there exist pl, . . . ,;t,, E /\(T,) and A1, . . . , A, E A(M) such that

ill» - /\I=i S iiTP'4”2 = 512

fork=1,...,p.

For a proof, see page 365 of [1].

% 26.txt
26

4.2 A Lanczos Procedure for Markov Chains

Notice that the Lanczos algorithm requires the computation of Mqj. Throughout this
paper we have assumed that M 6 Rd“, and that d is so large as to make the matrix vector
multiplication Mqj impossible. Now, the Lanczos algorithm constructs an orthonormal basis for
the Krylov subspace. For our purposes, this construction is not essential. What is essential is that
we find a way to generate the Lanczos coejficients oz,-, fij, (j = 1, . . . , p), and whence the matrix
T, without performing the operation Mqj required by the Lanczos algorithm. We now address this
problem.

Begin with the centered and weighted observable 1/1 = 111/ 2 (qS—E,,¢), and let q1 = 11;/||1/JH2.
To simplify notation, let ¢(n) E ¢(X,.). The form of the first Lanczos coeflicient a1 is straight

forward:

01 = qiMq1

I/1^tM1/I
¢’¢

and now the special properties of the Markov chain setting play an important role. For, regard the

following

¢^tM«/2 = (¢ — E.¢)^t\Psi%^tM\Psi%(¢ - Em)

Cw (d>(n), ¢(" + 1)) (4-5)

and

W1!’ = (<15 - E«¢)'H(¢ - E«¢)

Varn (¢(")) (4-6)

Recall that the definition of M implies H%^tMI'I% = HP, and this justifies equation (4.5). Putting

% 27.txt
27

these equations together, we have a. nice form for :11:

_ c, (run), an + 1))
"^t ' Ta:,(¢(n))

The next Lanczos coeflicient, B1, is only slightly more complicated.

[31 = ||M91-0t1tI1||2

[(Mq1 — 0l1¢11)i(M¢]1 - 0101)] 1/2

[t1iM2¢I1 — ail 1/2

(4.7)

Again recalling the definition M E 111/ 2P1'I^t1/2, a. simple manipulation verifies

1'li^tM’l'[% = HP’, whereby it follows that

(«s — E,,¢)^tHP”(¢ — Em)

"^tM2"^t (¢ — E,¢)*n<¢ — Em)

C: (¢("), ¢(n + 2))
V3-Ur 

Inserting the last expression into equation (4.7) gives

_ c,(¢(n).¢<n+2)) c,(¢<n>,¢(n+1)> ’ "’
"“ jvar,<¢<n)> "(jvar,<¢<n)) ^t)

Notice that we can estimate these expressions of oz; and fll by running simulations of the Markov
chain. To do so, we run the Markov chain starting from a random initial state, compute the value of
the observable at each step of the chain and, from the data, compute estimates of the covariances
and variance of the observable.

Before generalizing this for the jth Lanczos coefficient, we consider one more coeflicient
in detail. Recall that the second basis element q; in the Lanczos algorithm is given by 1;; =

(Mql — oz1q1)/fil. From this we can derive an expression for (12 which, though not as neat as those

% 28.txt
28

for oz; and [31, is nonetheless tractable.

I12 ^tIiMq2

1
[7(M91 — 0t1¢I1)tM(M¢I1 - 11191)
1

1
= f,§(qiM3q1 - 2a1qlM2<11 + a’9iM¢11)
1
1
= —2(<1iM3q1 - 2015i ^t all (4-8)

1.
The last equality follows from equation (4.7), which can be written as q{M2q1 = off’ + (if. The
preceding paragraphs define all but one of the quantities in the expression (4.8). The one remaining
is q§M3q1, which can also be expressed in terms of covariances of <15, and the analysis is almost

identical to that for q{M2q1:

(¢ — E,¢)'\PsiP3(¢ — Em)

"lM3"^t (¢ — E,¢s)^tH(¢ - Em»)

Cu (¢(")a ¢(n + 3))
Varn(¢(n))

To generalize the foregoing for higher order Lanczos coeflicients requires a recurrence
relation between (aj+1, flj) and (aj, fl,~_1). As we will see, such a relation involves higher order
Lanczos vectors q,~ — vectors we do not wish to compute. Recall that the only “vector” we can
handle is our initial observable. Therefore, we must also establish a recurrence among the q,-’s, thus
making them available inductively from our observable vector.

To begin with the flfs, recall that fij E ||Mqj — ajq_.,- — ,H_,~_1q,-_;|[2. Whence,

Bf (M411 — °^tj^tIj - fij-1qj—1)’(MlIj - Otjqj - 51-19141)

q;-M291^t - 0? - B,z_1 (4-9)

Some simple algebra and the definitions aj E q§Mqj and fiJ~_1 E q;-Mq,--1 prove equality (4.9). The
% 29.txt
recurrence for 0114.1 is equally straight forward, though the final form is not as neat:

2
0tj+1 ¢Ij+1M'1j+1

1
§(M¢15 - 11111;^t - flj—1<Ij—1)'M(Mqj - Otjqj - /3j—1qj—1)
J

1
§(q;-M-“*4; — 2a,~q;M*q,- — 2x3;+’_1q;--1M*q,- + «I? + 2a,-5,51 + am (4.10)
J
From (4.9), q;~M2q,~ = oz? + flf + fl}_1 and, upon plugging this into (4.10), we arrive at
1
a,~+1 = E(q§»M3q_,,- — (1? — 2ozJ~fiJ2 — 2flj_1q;»_1M2qj + fi2_1) (4.11)
J

The expressions above involve the higher order Lanczos vectors qj. In the present context,
these vectors will not be readily available. However, another recursion — this time relating forms
involving q,-+1 to forms involving qj and qj__1 — will make these expressions accessible by induction
on our initial observable. The general forms of interest are q;+1M"qj+1 and qj-+1M"qJ~. Again, we

begin with the simpler of the two:

¢I;'+1Mk9j = ij(Mqj - Otjqj - flj—1t1j—1)'M"¢1j
= i(<1§M"“¢Ij - C^tj^tI_;'Mk^tIj - flj—1q§M"qj-1)
The second is,
^tI_;'+1Mkqj+1 = fli?(M<Ij - Otjqj - .3j—1<Ij—1)^tMk(Mqj — Orjqj - flj—1¢1j—1)

1
fi(^t1§M'°"<1:^t + °^t:2"I§M'°<J:' + fi?—z<Ii—1M"<I:^t—1
J
—2a,-q;-M*+^tq,~ + 2am,--1q;-M*q,--1 — zza,--1q;-M*+^tq]--1) (4.12)

The point of all this is that each pair of coeflicients (oz,-+1, /3,-) can be expressed as a
function of the covariance between ¢(n) and ¢(n+ m), for m = 0, . . . , 21' + 1. Since the expressions
are based on four recurrence relations, they become quite complicated as 1' gets large. However,
defining these recursions in a. computer program is trivial, and we use this fact in Section 5.4 when

applying the foregoing to a particular problem.

% 30.txt
30

