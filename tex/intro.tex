The rate at which a Markov chain converges to a given probability distribution
has long been an active area of research. This is not surprising considering
this problem's relevance to the areas of statistics, statistical mechanics, and
computer science. Markov Chain Monte Carlo (MCMC) algorithms provide important
examples. These algorithms come in handy when we encounter a complicated
probability distribution from which we want to draw random samples. In
statistical mechanics, we might wish to estimate the phase average of a function
on the state space. Goodman and Sokal~\cite{GoodmanSokal:1989} examine Monte
Carlo methods in this context. Examples from statistics occur in the Bayesian
paradigm when we are forced to simulate an unwieldy posterior distribution (see,
e.g., Geman and Geman~\cite{Geman:1984}.  

To implement the MCMC algorithm, we invent a Markov chain that converges to the
desired distribution (this is often accomplished using the \emph{Metropolis algorithm}
described in Chapter~\ref{cha:an-appl-conv}). Realizations of the chain will
eventually represent samples from this distribution. Sometimes
``eventually''---meaning all but finitely many terms of the chain---is just not 
enough. We need more practical results. In particular, we want to know how many
terms of the chain should be discarded before we are sampling from a
distribution that is close (in total variation distance) to 
%
%
% ------------- 03.txt -----------------------------------------------------
%
%
the distribution of interest. This is the purpose of bounding convergence rates
for Markov chains. 

Often the Markov chains encountered in this context satisfy a condition known in
the physics literature as \emph{detailed balance}. Probabilists call chains with
this property \emph{reversible}. This simply means that the chain has the same
probability law whether time moves forward or backward.\footnote{This is not a
precise definition. In particular the chain must have started from its
stationary distribution. Full rigor is postponed until
Section~\ref{sec:general-theory}.} 
In this paper, we consider the rate at which such chains converge to
a \emph{stationary distribution}. This and other italicized terms are
defined in Section~\ref{sec:general-theory}.

There are a number of different methods in common use for bounding convergence
rates of Markov chains, and a good review of these methods with many references
can be found in~\cite{Rosenthal:1995}.  More recently developed methods
employing logarithmic Sobolev inequalities are reviewed in~\cite{Diaconis:1996}.  
Most of the bounds in common use involve the subdominant eigenvalue of the
Markov chain's transition probability matrix, and thus require good
approximations to such eigenvalues. In many applications, however, the
transition probability matrix is so large that it becomes impossible to store
even a single vector of the matrix in conventional computer memory. These so
called \emph{out-of-core} problems are not amenable to traditional eigenvalue
algorithms without modification.\footnote{By ``traditional 
  eigenvalue algorithms'' we mean those found, for example, in Golub and Van
  Loan~\cite{Golub:1996}. See also the book by Demmel\cite{Demmel:1997} for more 
  recent treatment.}
In this paper we develop such a modification for the Markov chain eigenvalue
problem. In particular we develop a method for approximating the first few
eigenvalues of a transition probability matrix when we know the general
structure of the underlying Markov chain. The method does not require storage
of large matrices or vectors. Instead we need only simulate the Markov chain,
and conduct a statistical analysis of the simulation.

Here is a brief summary of the paper. Section~\ref{sec:general-theory} contains
a review of the relevant Markov chain theory. Readers who already know the
basics of asymptotic theory of Markov chains might wish to skim 
Section\ref{sec:general-theory} if only to familiarize themselves with our
notation. Section~\ref{sec:funct-state-space} describes functions on the state 
% 04.txt
space of the Markov process. This section and Chapter~\ref{cha:invar-appr-invar}
develop the context in which we formulate the new ideas of the paper. In the
last section of Chapter~\ref{cha:invar-appr-invar},
Section~\ref{sec:krylov-subspace}, we describe the \emph{Krylov subspace} and
explain why this represents our best approximation to a subspace containing 
extremal eigenvectors of the transition probability matrix (more
  precisely, a similarity transformation of this matrix). The first section of
Chapter~\ref{cha:lanczos-procedures} describes the \emph{Lanczos algorithm} for
generating an orthonormal basis for the Krylov subspace. As it stands, this
algorithm is useless for an out-of-core problem since, by
definition of such problems, it requires too much data movement; all the
computing time is spent swapping data between slow and fast memory (disk to ram
to cache and back). We develop alternatives to the Lanczos algorithm and
demonstrate that the \emph{Lanczos coefficients} of the algorithm can be
obtained by simulations of the Markov chain, and this allows us to avoid
the standard algorithm altogether. Following this is a chapter describing the
Metropolis algorithm used to produce a reversible stochastic matrix. It is here
that we experiment with the procedure described in
Section~\ref{sec:lancz-proc-mark} and approximate the extremal eigenvalues of
the matrix, without storing any of its vectors. Finally,
Chapter~\ref{cha:conclusion} concludes the paper. 
% 05.txt
