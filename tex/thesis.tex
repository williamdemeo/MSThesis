%% NYU PhD thesis format. Created by Jos� Koiller 2007--2008.

%% Use the first of the following lines during production to
%% easily spot "overfull boxes" in the output. Use the second
%% line for the final version.
%\documentclass[12pt,draft,letterpaper]{report}
\documentclass[12pt,letterpaper]{report}

%% Replace the title, name, advisor name, graduation date and dedication below with
%% your own. Graduation months must be January, May or September.
\newcommand{\thesistitle}{Proof of the Riemann Hypothesis}
\newcommand{\thesisauthor}{Jane Doe}
\newcommand{\thesisadvisor}{Professor Fulana de Tal}
\newcommand{\graddate}{May 2032}
%% If you do not want a dedication, scroll down and comment out
%% the appropriate lines in this file.
\newcommand{\thesisdedication}{To my dog Weierstra\ss, with affection.}

%% The following makes chapters and sections, but not subsections,
%% appear in the TOC (table of contents). Increase to 2 or 3 to
%% make subsections or subsubsections appear, respectively. It seems
%% to be usual to use the "1" setting, however.
\setcounter{tocdepth}{1}

%% Sectional units up to subsubsections are numbered. To number
%% subsections, but not subsubsections, decrease this counter to 2.
\setcounter{secnumdepth}{3}

%% Page layout (customized to letter paper and NYU requirements):
\setlength{\oddsidemargin}{.6in}
\setlength{\textwidth}{5.8in}
\setlength{\topmargin}{.1in}
\setlength{\headheight}{0in}
\setlength{\headsep}{0in}
\setlength{\textheight}{8.3in}
\setlength{\footskip}{.5in}

%% Use the following commands, if desired, during production.
%% Comment them out for final version.
%\usepackage{layout} % defines the \layout command, see below
%\setlength{\hoffset}{-.75in} % creates a large right margin for notes and \showlabels

%% Controls spacing between lines (\doublespacing, \onehalfspacing, etc.):
\usepackage{setspace}

%% Use the line below for official NYU version, which requires
%% double line spacing. For all other uses, this is unnecessary,
%% so the line can be commented out.
\doublespacing % requires package setspace, invoked above

%% Each of the following lines defines the \com command, which produces
%% a comment (notes for yourself, for instance) in the output file.
%% Example:    \com{this will appear as a comment in the output}
%% Choose (uncomment) only one of the three forms:
%\newcommand{\com}[1]{[/// {#1} ///]}       % between [/// and ///].
\newcommand{\com}[1]{\marginpar{\tiny #1}} % as (tiny) margin notes
%\newcommand{\com}[1]{}                     % suppress all comments.

%% This inputs your auxiliary file with \usepackage's and \newcommand's:
%% It is assumed that that file is called "definitions.tex".
\input{definitions}

%% Cross-referencing utilities. Use one or the other--whichever you prefer--
%% but comment out both lines for final version.
%\usepackage{showlabels}
%\usepackage{showkeys}


\begin{document}
%% Produces a test "layout" page, for "debugging" purposes only.
%% Comment out for final version.
%\layout % requires package layout (see above, on this same file)

%%%%%% Title page %%%%%%%%%%%
%% Sets page numbering to "roman style" i, ii, iii, iv, etc:
\pagenumbering{roman}
%
%% No numbering in the title page:
\thispagestyle{empty}
%
\begin{center}
  {\large\textbf{\thesistitle}}
  \vspace{.7in}

  by
  \vspace{.7in}

  \thesisauthor
  \vfill

\begin{doublespace}
  A dissertation submitted in partial fulfillment\\
  of the requirements for the degree of\\
  Doctor of Philosophy\\
  Department of Mathematics\\
  New York University\\
  \graddate
\end{doublespace}
\end{center}
\vfill

\noindent\makebox[\textwidth]{\hfill\makebox[2.5in]{\hrulefill}}\\
\makebox[\textwidth]{\hfill\makebox[2.5in]{\hfill\thesisadvisor\hfill}}
\newpage
%%%%%%%%%%%%% Blank page %%%%%%%%%%%%%%%%%%
\thispagestyle{empty}
\vspace*{0in}
\newpage

%%%%%%%%%%%%%% Dedication %%%%%%%%%%%%%%%%%
%% Comment out the following lines if you do not want to dedicate
%% this to anyone...
\vspace*{\fill}
\begin{center}
  \thesisdedication\addcontentsline{toc}{section}{Dedication}
\end{center}
\vfill
\newpage
%%%%%%%%%%%%%% Acknowledgements %%%%%%%%%%%%
%% Comment out the following lines if you do not want to acknowledge
%% anyone's help...
\section*{Acknowledgements}\addcontentsline{toc}{section}{Acknowledgements}
First, I thank my advisor, Professor Jonathan Goodman, for giving me the opportunity to work on
this problem, and helping me arrive at the following exposition. Next, I wish to thank Professor
Helena Frydman for first sparking my interest in Markov chains by giving lucid descriptions of their
many interesting properties and applications. I thank Professors James Demmel and Beresford
Parlett, for answering many questions pertaining to this problem, and Professor Leslie Greengard
for agreeing to review this paper.

Finally, I would like to thank my family, for their support of my interests in research (and
everything else!), and especially my parents, Barbara and Ted Terry, and Benita and Bill De Meo.

Without them, this paper would not have been written.


\newpage
%%%% Abstract %%%%%%%%%%%%%%%%%%
\section*{Abstract}\addcontentsline{toc}{section}{Abstract}
\begin{center}
\newcommand\skipsize{6pt}
A Lanczos Procedure for Approximating Eigenvalues of Large Stochastic Matrices\\[\skipsize]
by\\[\skipsize]
William J. DeMeo\\[\skipsize]
Master of Science in Mathematics\\[\skipsize]
New York University\\[\skipsize]
Professor Jonathan Goodman, Chair
\end{center}

The rate at which a Markov chain converges to a given probability distribution
has long been an active area of research. Well known bounds on this rate of
convergence involve the subdominant eigenvalue of the chain's underlying
transition probability matrix. However, many transition probability matrices are
so large that we are unable to store even a vector of the matrix in fast
computer memory. Thus, traditional methods for approximating eigenvalues are
rendered useless. 

In this paper we demonstrate that, if the Markov chain is reversible, and we
understand the structure of the chain, we can derive the coefficients of the
traditional Lanczos algorithm without storing a single vector. We do this by
considering the variational properties of observables on the chain's state
space. In the process we present the classical theory which relates the
information contained in the Lanczos coefficients to the eigenvalues of the
Markov chain. 

\newpage
%%%% Table of Contents %%%%%%%%%%%%
\tableofcontents

%%%%% List of Figures %%%%%%%%%%%%%
%% Comment out the following two lines if your thesis does not
%% contain any figures. The list of figures contains only
%% those figures included withing the "figure" environment.
\listoffigures\addcontentsline{toc}{section}{List of Figures}
\newpage

%%%%% List of Tables %%%%%%%%%%%%%
%% Comment out the following two lines if your thesis does not
%% contain any tables. The list of tables contains only
%% those tables included withing the "table" environment.
\listoftables\addcontentsline{toc}{section}{List of Tables}
\newpage

%%%%% Body of thesis starts %%%%%%%%%%%%
\pagenumbering{arabic} % switches page numbering to arabic: 1, 2, 3, etc.
%% Introduction. If your thesis has no introduction, or chapter 1 is
%% meant to be the introduction, then comment out the lines below.
\section*{Introduction}\addcontentsline{toc}{section}{Introduction}
The rate at which a Markov chain converges to a given probability distribution has long
been an active area of research. This is not surprising considering this problem’s relevance to the ar-
eas of statistics, statistical mechanics, and computer science. Markov Chain Monte Carlo (MCMC)
algorithms provide important examples. These algorithms come in handy when we encounter a
complicated probability distribution from which we want to draw random samples. In statistical
mechanics, we might wish to estimate the phase average of a function on the state space. Goodman
and Sokal [6] examine Monte Carlo methods in this context. Examples from statistics occur in the
Bayesian paradigm when we are forced to simulate an unwieldy posterior distribution (see, e.g.,
Geman and Geman 

To implement the MCMC algorithm, we invent a Markov chain that converges to the
desired distribution (this is often accomplished using the Metropolis algorithm
described in Chapter 5). Realizations of the chain will eventually represent
samples from this distribution. Sometimes ``eventually'' -- meaning all but
finitely many terms of the chain -- is just not enough. We need more practical
results. In particular, we want to know how many terms of the chain should be 
discarded before we are sampling from a distribution that is close (in total variation distance) to
% 03.txt
the distribution of interest. This is the purpose of bounding convergence rates for Markov chains.

Often the Markov chains encountered in this context satisfy a condition known in the
physics literature as detailed balance. Probabilists call chains with this property reversible. This
simply means that the chain has the same probability law whether time moves
forward or 
backward.\footnote{This is not a precise definition. In particular the chain must
  have started from its stationary distribution. Full rigor is postponed until Section 2.1.}
In this paper, we consider the rate at which such chains converge to a \emph{stationary distribution}.\footnote{This and other italicized terms are defined in Section 2.1.}

There are a number of different methods in common use for bounding convergence rates of
Markov chains, and a good review of these methods with many references can be found in  More
recently developed methods, employing logarithmic Sobolev inequalities, are reviewed in  Most
of the bounds in common use involve the sub—dominant eigenvalue of the Markov chain's transition
probability matrix, and thus require good approximations to such eigenvalues. In many applications,
however, the transition probability matrix is so large that it becomes impossible to store even a
single vector of the matrix in conventional computer memory. These so called out-of-core problems
are not amenable to traditional eigenvalue algorithms\footnote{By ``traditional
  eigenvalue algorithms'' we refer to those found, for example, in Golub and Van
  Loan[5]. See also the book by Demmel [1] for a more recent discussion.}
without modification. This paper develops 
such a modification for the Markov chain eigenvalue problem. In particular it develops a method
for approximating the first few eigenvalues of a transition probability matrix when we know the
general structure of the underlying Markov chain. The method does not require storage of large
matrices or vectors. Instead we need only simulate the Markov chain, and conduct a statistical
analysis of the simulation.

Here is a look at what follows. Section 2.1 contains a review of the relevant Markov chain
theory. Readers conversant in the asymptotic theory of Markov chains might wish to at least skim
Section 2.1, if only to become familiar with our notation. Section 2.2 describes functions on the state
% 04.txt
space of the Markov process. This section and Chapter 3 develop the context in which we formulate
the new ideas of the paper. In the last section of Chapter 3, Section 3.3, we present the familiar
Krylov subspace and explain why this represents our best approximation to a subspace containing
extremal eigenvectors of the transition probability matrix.\footnote{or, more
  precisely, a similarity transformation of this matrix.} The first section of
Chapter 4 describes the \emph{Lanczos algorithm} for generating an orthonormal basis
for the Krylov subspace. As it stands, this algorithm is useless for an
out-of-core problem such as ours since, by definition of such problems, 
it requires too much data movement; all the computing time is spent swapping data between slow
and fast memory (e.g. between the hard disk and cache). Therefore, we discuss alternatives to
Lanczos and demonstrate that the \emph{Lanczos coeficients} are readily available through simulations of
the Markov chain, which fact allows us to avoid the standard algorithm altogether. Following this
is a chapter describing the Metropolis algorithm used to produce a reversible stochastic matrix.
It is here that we experiment with the procedure described in Section 4.2 and approximate the
extremal eigenvalues of the matrix, without storing any of its vectors. Finally, Chapter 6 concludes
the paper.
% 05.txt

%% If your thesis has different "Parts", use commands such as the following:
%\part{First Part\label{part:one}}%
\input{chap1}
%\input{chap2} % further chapters -- change file names to meaningful things...
%\input{chap3}
%\part{Second Part\label{part:two}}%
%\input{chap4}
%\input{chap5}
%\input{chap6}
%%%%% Appendices start %%%%%%%%%%%%%%%%
%% Comment out the following line if your thesis has no appendix
\appendix
\input{app1}
%% Note: If your thesis has more than one appendix, NYU requires a "list of
%% appendices" page before the body of the thesis. I don't provide the tools
%% to create that here, so you're on your own for that one... Sorry.
%\input{app2}
%%%% Input bibliography file %%%%%%%%%%%%%%%
\input{biblio}

\end{document}
